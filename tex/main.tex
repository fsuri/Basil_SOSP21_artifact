\documentclass[letterpaper,twocolumn,10pt]{article}
\usepackage{usenix2019_v3}
\usepackage{preamble}


% to be able to draw some self-contained figs
\usepackage{tikz}
\usepackage{amsmath}
\usepackage{xspace}
% inlined bib file
\usepackage{filecontents}
\usepackage{algorithm}
\usepackage{amsthm}
\usepackage[noend]{algpseudocode}
\usepackage{mathtools}
%\usepackage[ruled, vlined, linesnumbered]{algorithm2e}
\usepackage{mdframed}
\usepackage{enumitem}
\usepackage{xcolor}
\setlist{itemsep=0pt,parsep=0pt}             % more compact lists
%\usepackage[belowskip=-5pt,aboveskip=5pt]{caption}
 
%\setlength{\intextsep}{10pt plus 2pt minus 2pt}
%-------------------------------------------------------------------------------
\begin{document}
%-------------------------------------------------------------------------------

%don't want date printed
\date{}

% make title bold and 14 pt font (Latex default is non-bold, 16 pt)
\title{\Large \bf Indicus: Unchaining Byzantine Databases\\
 }

%for single author (just remove % characters)
\author{
{\rm Your N.\ Here}\\
Your Institution
\and
{\rm Second Name}\\
Second Institution
% copy the following lines to add more authors
% \and
% {\rm Name}\\
%Name Institution
} % end author

\maketitle

%-------------------------------------------------------------------------------
\begin{abstract}
%-------------------------------------------------------------------------------
This paper presents \sys{}, the first leaderless Byzantine Fault Tolerant (BFT) key-value store that provides ACID transaction. Unlike traditional BFT approaches that build atop totally ordered SMR protocols, \sys is designed around the premise of \textit{independent operability}. By allowing clients to execute and drive consensus instances for their own transactions in parallel, \sys enables commutative transactions to complete independently while limiting the influence of Byzantine behavior on results. \sys furthermore introduces a novel recovery mechanism to retain liveness under faults.
This design, coupled with an aggressive concurrency control mechanism to maintain Isolation correctness, allows \sys to reasonably compete within non-BFT tolerant transaction systems, while outperforming SMR-based BFT solutions: For the TPC-C OLTP workload \sys{}'s throughput and latency comes within Yx and Zx respectively when compared to Tapir, a state-of-the-art non-BFT key-value store; but improves over a total-order inducing BFT baseline by Sx and Tx respectively.
\end{abstract}


%-------------------------------------------------------------------------------
\section{Introduction} 
%-------------------------------------------------------------------------------


%Safe and efficient online data sharing among mutually distrustful
parties offers exciting opportunities for a variety of applications,
including healthcare~\cite{}, financial services~\cite{}, supply chain
management~~\cite{} and more~\cite{}. It is also challenging, as it
requires building a new breed of systems, which can offer to their
users the functionaalities of a distributed database but do so under
much weaker trust assumptions that the database community has
typically considered.

An increasingly popular answer to this challenge are permissioned
blockchains. These systems use a Byzantine fault tolerant (BFT)
replicated state machine protocol\cite{} to produce a totally ordered
log of client transactions, upon which they layer a database
abstraction.  The obvious appeal of this approach is that it can
immediately benefit from, and  build upon, the existing BFT
literature. However, the total order it imposes on transactions is not
only costly to implement and hard to scale, but it is unnecessary for
many of its target applications. For example, unrelated financial
transactions need not be totally ordered, and supply chains, despite
their name, are actually complex networks that generate and process
many logically concurrent transactions.

This mismatch motivates us to explore in this paper a fundamentally
different design point: instead of layering database functionalities on top
of BFT state machine replication, we aim to build \la{the first (?)} a
{\em Byzantine fault tolerant transactional key-value store}.

In principle, this new approach can yield several benefits.

First, it can overcome the scalability bottleneck of a low-level
implementation based on a totally ordered BFT log. Sharded blockchains
can mitigate this bottleneck, but only to a point, as the concurrency
they can achieve is limited by the degree to which transactions can be
executed on individual shards. In contrast, we aim to leverage
databases' ability, honed over decades of research in concurrency
control, to support highly concurrent transaction processing while
retaining, to ease reasoning about correctness, the capacity to
generate executions equivalent to a total order of commmitted
transaction.


Second, by doing away with an explicit ordering service, it can
sidestep the growing concerns~\cite{} about ensuring that blockchains order
transactions fairly. Some protocols~\cite{} leave that ordering to the
whim of a leader; to limit its discretion, other protocol rotate
that rsponsibility, thus reducing, but not eliminating, the skew that can be
introduced by a malicious or self-interested leader. To make things
worse, there is no rigorous, non-anecdotal definition for what
fairness should mean in this context.

Third, it offers the opportunity to eliminate the redundant efforts of
transactional systems built on top state machine replication, which
enforce strong consistency twice, first running replication and then
running atomic commit~\cite{}.

At the same time, a BFT it also presents obvious challenges


\la{Explain the challenge}\\
\la{Point to our ethos as the way we try to address the challenges}\\
\la{briefly introduce the system, unclear in what detail--but not too much}\\
\la{The describe our experience in building and evaluating: we have implemented... We find...}\\
\la{Conclude with the key contributions}

------------------------------------



This paper introduces Indicus, the first leaderless transactional key-value store that is robust in the Byzantine Fault Model. Indicus aims to mitigate the tension between real-world, highly commutative transaction workloads striving for scalability, and the simplicity that totally ordered ledger abstractions such as Blockchains or State Machine Replication offer.

\fs{this maybe sounds too much like a permissioneless blockchain. It needs to be clear that we target permissioned settings which is by definition not quite "trustless", but we "trust" that only a fraction will try to compromise the system.} Specifically, this paper asks the question: how can we enable \textit{mutually distrustful parties} to consistently, reliably share and scalably share data, while minimizing centralization. \\


The ability to share data online offers exciting opportunities; however, increased datasharing also raises the concern of how to \textit{decentralize trust}. In banking, systems like SWIFT (cite) enable financial institutions to quickly and accurately enable cross-institutional transaction clearance, at the cost of placing their trust in the centralized SWIFT network. In manufacturing, online data sharing can improve accountability and auditing amongst the globally distributed supply chain, but there may not be an identifiable source of trust. Consider the supply chain for the latest iPhone: it spans three continents, and hundreds of different contractors \cite{AppleSup} that may neither trust Apple, nor each other, yet must be willing to share and agree on information concerning the construction of the same product. Even sole entities that distribute their dataceneters globally (i.e. Google, Amazon), may not trust their \textit{own} datacenters located in authoritative domains or legislations out of its control.\\


Recognizing this challenge by both the research and industrial communities, much effort has focused on enabling shared computation \fs{sounds like we will talk about secure Multi party computation} between mutually distrustful parties in the context of Byzantine Fault Tolerance (BFT) and Blockchains. 
Systems proposed in the literature of BFT provide the abstraction of a totally ordered request log; the log is agreed upon by the \textit{n} participants in the system, of which at most \textit{f} can misbehave. In the Blockchain world, Bitcoin and Ethereum have become popular distributed computing platforms providing the same log abstraction while aiming for decentralizing trust and open membership \fs{at much lower throughput, more power consumption, no finality}. At the intersection, systems such as Hyperledger Fabric \cite{Hyperledger} or Ethereum Quorum \cite{EthereumQuorum} aim to cater to Blockchain markets while leveraging traditional BFT approaches for scalability. Efforts to incorporate Blockchain/BFT technologies into market ready infrastructures are pervasive \cite{StateFarmQuorum, AutoInventory, StateFarmQuorum2, HyperledgerTelecom, HyperledgerHealth}. \\


At its core, blockchains are functionally a database \fs{Hyperledger itself claims to be a shared database}; they provide a platform for mutually distrustful participants to jointly replicate transactions.
In this paper however, we argue that existing solutions merely shoehorn the desired functionality of transactional applications rather than catering to the principal requirements of a database. 
While applications in stronger failure domains \fs{(Crash Failure)} are built atop transactional databases whose primary design concern is performance, the design of BFT artifacts has been mainly driven by the fault model rather than the desired database functionality, making scalability a secondary optimization concern. 
Instead, we argue, that the practical \fs{sustainable?} way to service applications requiring BFT is to let the functionality drive the design foundation and build dedicated databases that are co-designed with byzantine resilience. \fs{Have a DB that is BFT, rather than BFT that tries to be DB}\fs{people might disagree}
Concretely, we stipulate that an appropriate Database for the Byzantine Fault Model should 1) maintain the \textit{abstraction} of a sequential execution, 2) be scalable, 3) be resitant to censorship and frontrunning and 4) bound the effects of byzantine components, be it replicas or clients.
We outline how existing approaches fall short of satisfying one or more of these requirements, and finally propose a system to fill this gap.

\fs{paragraphs of shortcomings cut here and moved to section 2 in extended form.}
\iffalse
\fs{CUT from HERE}
First, we argue there exists a fundamental mismatch between the implementation of a totally ordered log and the reality of much large-scale distrubted processing. Many large-scale distributed systems consists primarily of unordered and unrelated operations. For example, a product supply chain consists of many concurrent steps that do not require ordering. Imposing an ordering on non-conflicting operations is not only often unecessary, but costly: participants in the shared computation must vote to order operations and serialize request execution accordingly, thus harming both throughput and end-to-end latency. \fs{people can claim that there exist partially ordered solutions -> Dag blockchains, clairvoyant}\\

Second, while there exists work on mitigating this scalability bottleneck through sharding (cite Omniledger, Chainspace, Callinicos), it remains largely an engineering optimization. At its core, the application of sharding conflates two objectives: Horizontal hardware scaling and transactional concurrency \fs{too vague currently}. When a mapping from data to shards is well chosen (note, that this relies on application specific knowledge) partial ordering between objects on different shards manifests as a side-effect of seperating resources \fs{will others agree to this view?}. In fact, when transactions span multiple shards, the latent total order requirement can introduce redundant coordination overheads, as coordination may need to be performed twice, at the level of individual shards, and across shards \cite{zhang2016operation}. This is especially problematic when workloads are geo-replicated (citation?), or when, as in BFT, the replication factor is high \fs{why? be more specific}. \\

Third, achieving a total order often \fs{What if not? OR: Is it even possible without leader?} necessitates a single point of centralization, in order to propose a sequencing order. This is especially undesirable in trust concerned settings as such a sequencer exposes not only a scalability bottleneck but a fairness vulnerability. A sequencer may be biased, frontrun requests or censor others and is inherently antithetical to the need of decentralized, trustless solutions.\\

Lastly, most BFT/Blockchain systems support transactions under the assumption that their read and write operations are known a priori, which limits the set of applications that they can support. \fs{this is sort of random and not in line with the requirements above}


In summary, prior BFT solutions succumb to one or more of the following fallacies, some of which are correlated: They a) impose a restrictive total order, b) use leader in some form or another, c) assume fixed transaction sets or d) incur redundant coordination overheads when sharding. \\

\fs{this paragraph feels like taking a step back}
Leveraging commutativity between transactions is not a new idea. As a research trend of mitigating the scalability bottleneck, EPaxos (cite), TAPIR (cite) and CURP (cite) only consider the ordering between potentially conflicting operations. However, these systems assume the stronger crash-failure model and are non-trivial to extend to the Byzantine model, so that they cannot directly solve the problem of data sharing among mutually distrustful, malicious or arbitrarily failing parties.
\fs{CUT TILL HERE}
\fi

Existing research, in essence, is either attempting to build concurrency control and sharding functionalities over BFT replication, or integrating these functionalities into a crash-failure replication protocol. In this paper, we will show how to build these desiring functionalities inside a BFT replication protocol. Specifically, our goal is to \textit{provide the illusion of a centralized shared log, rather than the non-scalable reality of a totally ordered log}.\\

\fs{paragraph on CF systems cut and moved to section 2}
\iffalse
\fs{rambling too long}
\fs{CUT FROM HERE}

While copious efforts exist to design decentralized systems that exploit transaction semantics for the crash failure model, few, if any attempts have been made for the Byzantine Fault Model. This naturally raises the question, why so? One explanation is that in Byzantine Systems, some centralization is in fact highly desirable as it simplifies the problem by identifying a single point of accountability. A natural way to improve both scalability and fairness is to avoid this bottleneck, at the cost of paving the path for a wider set of undesirable phenomena. The challenge of a leaderless byzantine system is simple yet daunting: It empowers both malicious users and system components to collude and misbehave in potentially unaccountable ways.
In this paper we will show to overcome this challenge by designing and implementing the leaderless BFT system Indicus that avoids near-all centralization while replicating interactive ACID transactions in a partial order. 
\fs{client driven protocol could be seen as nightmare for byz system. What is our design agenda to control this? -> Byz can only influence itself. Honest users can keep using the system }\\
\fs{CUT TILL HERE}
\fi

In this paper we will show to realize this goal by designing and implementing a leaderless BFT database system called Indicus, that avoids near-all centralization while replicating interactive ACID transactions in a partial order. Distinctively, Indicus does not aim to be the 701th BFT State Machine Replication protocol (cite the next 700 bft), but the first BFT protocol to integrate replication and transaction layer.

Indicus' design is driven by the ethos of \textit{independent operability}; both safety \fs{(integrity, serializability)} and liveness \fs{ability to issue and complete TX} are per-client properties and independent from one client to another.
\fs{Indicus embraces partial order across the entire transaction stack} Unlike State Machine Replication (SMR) based systems that achieve agreement on computation (i.e. state transitions) by imposing a total order on execution across all replicas, Indicus executes transactions speculatively at clients, while validating and replicating \textit{all} transactions in arbitrary order at any given replica. Furthermore, clients coordinate the agreement protocol themselves, thus putting them in charge of their own transactions. Doing so, while tolerating byzantine client behavior, is Indicus' main challenge; In Indicus \textit{the client is king - but rulers have to be regulated}.

Abstractly, Indicus proposes each transaction for a seperate concurrent binary consensus instance and enforces an implicit partial order by aborting transactions when inconistencies arise. In order to maintain a serializable transaction history, Indicus assigns each transaction an optimistic timestamp and uses a variation of Multiversioned Timestamp Ordering (MVTSO) for concurrency control. Ordering conflicts between transactions whose interleavings would violate prescribed Isolation guarantees are broken based on the given timestamps and speculative execution results. \fs{does all of this need to be in an intro? There needs to be a little of key insight, but it feels more appropriate for discussion}\\



Overall, the Indicus design has the following implications:
\begin{itemize}%[noitemsep,topsep=0pt,parsep=0pt,partopsep=0pt]

\item Indicus is robust to censorship and frontrunning as there exist no central authority that admits and orders transactions.
\item Indicus allows commutative/non-conflicting Transactions to both be executed and validated out of order, thus maximizing parallelism. \fs{ (This should increase throughput and reduce latency, because clients arent waiting for all previously sequenced tx to finish. Consequently the tail latency does not dominate throughput as much.)}
\item Indicus minimizes the state, communication and computation load on replicas, as Clients serve as both execution hub and broadcast channel for their own Transactions. \fs{optional:}(This avoids quadratic communication communication complexity in the normal case.) \fs{(Theoretically always, but we keep some for practicality in the fallback)}
\item As any Quorum system, Indicus is inherently load-balanced as there exists no leader bottleneck and all replicas share equal responsibilities.
\item In Indicus liveness is a client local property. Unlike common SMR protocols, where the entire system halts during view changes, Byzantine participants may stall system progress only for the objects their transactions touch. Hence, the system appears live to any non-conflicting Transaction.
\end{itemize}

To achieve these properties Indicus incurs two main tradeoffs. First, shifting responsibility from replicas to clients comes at the cost of higher computational requirements for clients, which may not be tolerable for all applications. In practice we envision Indicus clients to be dedicated transaction managers, rather than end users. Second, like all optimistic concurrency control schemes, Indicus is vulnerable to congestion. When contention is high, abort rate soars; Indicus is not designed for such applications. Exploring tradeoffs between client/replica responsibilities as well as pessimistic concurrency control mechanisms are potential avenues for future work.

Results:  ...

To summarize, this paper makes three main contributions:
\begin{enumerate}
\item It offers definitions for transactional Isolation guarantees for the Byzantine Fault Model
\item It presents the design, implementation, and evaluation of the first leaderless ACID transactional system that tolerates Byzantine Failures
\item It presents two variations (Indicus3 and Indicus5) of a client centric agreement mechanism that trade-off performance for replication degree.  
\item \fs{our MVTSO?}
\end{enumerate}

\fs{perhaps unecessary}
The rest of the paper is structured as follows. In section 2 we elaborate the shortcomings of existing systems and challenges that the proposed design faces. In section 3 we delineate the system model and offer general purpose definitions for reasoning about byzantine databases. In section 4 and 5 we outline the Indicus architecture and protocols respectively. Section 6 describes our evaluation, while section 7 discusses limitations of the Indicus design. Finally, section 7 concludes the paper. 

\section{NC's introduction}

This paper presents the design and evaluation of \sys{}, the
first leaderless and scalable key-value store that is robust
to byzantine faults.

Safe and efficient online data sharing among mutually distrustful
parties offers exciting opportunities for a variety of applications,
including healthcare~\cite{}, financial services~\cite{}, supply chain
management~~\cite{} and more~\cite{}. \nc{I understand that you want to move 
to the main point quickly but I think that one example might be useful here still, just
one}. Byzantine fault tolerant (BFT)
systems~\cite{} and permissioned blockchains~\cite{}
are at the center of these new services. These protocols are guaranteed to
produce the same totally ordered log of operations across mutually distrustful
participants. 

Maintaining a totally ordered log of requests, inspite of its appealing simplicity, is problematic. Such an approach is hard to scale and costly to implement. It is, moreover, often unnecessary. Modern distributed applications primarily consist of independent operations that could be executed concurrently. Supply chains for instance, despite their name, are actually complex networks that generate and process logically concurrent transactions.

Existing research recognises this opportunity and attempt to mitigate the aforementioned scalability bottleneck through a combination of sharding~\cite{omniledger,chainspace,callinicos}, cross-shard atomic commit protocols and centralised ordering services~\cite{}. These systems suffer from three primary drawbacks 1) they limit the expressivity of the transactions tthey support~\cite{} 2) they introduce unnecessary coordination across replicas~\cite{} which affects throughput and finally, 3) they often centralises the ordering of requests at a single leader, which raises fairness concerns. These limitations all stem from the same fallacy: the mistaken belief that building out a total order of operations is necessary for correctness.
 
In reality, building \textit{the abstraction} of a total order is sufficient, and distributed databases have long recognised this fact. Serializability, the gold-standard database criterion, defines a correct execution as one that is \textit{equivalent} to a correct schedule. This definition allows concurrent operations to execute in parallel, restricting any ordering to conflicting operations only.  Byzantine systems need be no different.

This paper consequently argues for a paradigm "flip". The right way to build data-sharing systems for distrustful parties is not to layer database-like transactions and sharding on top of the reality of byzantine fault tolerant log, but instead to build out the abstraction of a BFT log on top of a partially ordered distributed database.
To this effect, we design \sys{}, the first leaderless and scalable BFT key-value store
and provide the first formal definition of byzantine serializabililty. \sys{} provides three main benefits.

 First, it overcomes the scalability bottleneck of a low-level
implementation based on a totally ordered BFT log. \sys uses a
client-driven protocol that bypasses the ordering bottleneck and
leverages databases' ability, honed over decades of research in
concurrency control, to support highly concurrent transaction
processing over a sharded database while retaining, to ease reasoning
about correctness, the capacity to generate executions equivalent to a
total order of commmitted transaction.

% an explicit ordering service 

Second, by doing away with an underlying totally ordered log, \sys{} can
sidestep the growing concerns~\cite{} about whether blockchains
order transactions fairly. Though the notion of fairness in this
context has not been rigorously defined, leaving the ordering to the
whim of a potentially Byzantine leader, as some protocols do~\cite{}
is intuitively problematic; to limit its discretion, other protocols
rotate the leader's responsibility, thus reducing, but not
eliminating, the skew it can introduce. In contrast, \sys leaves the
responsibility of driving the replication and distributed commit of a
transaction to the client that proposes it, and allows different
shards to order transactions in different orders, as long asways, as
long as serializability can be maintained.

Third, it avoids the duplicate ordering costs of transactional systems
built on top of state machine replication. As Tapir noted~\cite{},
these systems require, for replication, a consistent ordering of
operations within each shard, and additionally enforce, for
distributed commit, a serial order of transactions across
shards~\cite{}. In contrast, \sys{} integrates distributed commit and
replication through an optimistic concurrency control (OCC)~\cite{} protocol, to
limit coordination only to the validation step of the distributed commit
protocol---and does so in in a setting where clients, as well as a
fraction of the replicas in each shard, can behave maliciously.

In practice, reaping these benefits requires us to address several
technical obstacles. As in any OCC-based protocol, \sys
is vulnerable to aborts if transactions interleave unfavorably during
validation---but these concerns are compounded in a Byzantine
setting. First, since reading from a single local replica can no
longer guarantee integrity, it becomes necessary to read remotely,
thus increasing the window of time during which unfortunate
interleavings may occur. Second, Byzantine clients and replicas may
collude to actively sabotage the commit chances of transactions issued
by correct clients. 

Byzantine behaviour in effect introduces a tension that is not present in non-byzantine
distributed databases: optimistic and aggressive concurrency control mechanisms known
to improve performance (~\cite{occ,mvtso,pipeline,tapir}) in failure-free executions also increase the system's vulnerability to byzantine faults \nc{give example?}. To mitigate this tension, we design \sys{} to follow the ethos of \textit{independent operability}: both safety (namely, serializability) (the ability to issue and complete tranactions)
are \textit{local} properties. They remain, at all times, per clients and per-object.
 
\nc{Things to mention: we design a version of mvtso that still allows writes to be made visible early but allows clients to finish each other's transactions if a byz client tries to stall the system. If a Byz client tries to stall the system, the fallback is per objet only.
We are leaderless, we have linear cost everywhere}

In summary, we make the following three contributions: 
\begin{itemize}
\item definition of byzantine serializability
\item  byzantine mvtso + per-object concurrent view change?
\item system?
\end{itemize}

Nonetheless, our system has the following limitations ...

The rest of the paper is structured as follows ...


%


This paper introduces Indicus, the first leaderless transactional key-value store that is robust in the Byzantine Fault Model. Indicus aims to mitigate the tension between real-world, highly commutative transaction workloads striving for scalability, and the simplicity that totally ordered ledger abstractions such as Blockchains or State Machine Replication offer. Specifically, this paper asks the question: how can we enable \textit{mutually distrustful parties} to consistently and reliably share data, while minimizing centralization. \fs{this maybe sounds too much like a permissioneless blockchain}

The ability to share data online offers exciting opportunities, however, increased datasharing also raises the concern of how to \textit{decentralize trust}. In banking, systems like SWIFT (cite) enable financial institutions to quickly and accurately enable cross-institutional transaction clearance, at the cost of placing their trust in the centralized SWIFT network. In manufacturing, online data sharing can improve accountability and auditing amongst the globally distributed supply chain, but there may not be an identifiable source of trust. Consider the supply chain for the latest iPhone: it spans three continents, and hundreds of different contractors \cite{AppleSup} that may neither trust Apple, nor each other, yet must be willing to share and agree on information concerning the construction of the same product. Moreover, single authorities that distribute their dataceneters globally (i.e. Google, Amazon), may not even trust their \textit{own} datacenters located in authoritative domains or legislations out of its control.

Recognizing this challenge by both the research and industrial communities, much effort has focused on enabling shared computation between mutually distrustful parties in the context of Byzantine Fault Tolerance (BFT) and Blockchains. Systems proposed in the literature of BFT 

%\subsection{Intro-Notes}

\subsection{Header}

 - This paper introduces Indicus\\
 - interactive, scalable DB in the byz fault model\\
 - first to be fully client driven\\
 \nc{this might be personal taste, but I don't think being
 client-driven is particular novel or important. Being leaderless
 on the other hand seems like something we should emphasise}
 - therefore: fair and can scale to replica bandwidth/processing limits
 
 \subsection{Context}
 - people/entities would like to share data and/or split replication costs. Keep joint records of relevant data (i.e. for cross Audit): Imagine a transaction clearance between banks  (Libra as example: consortium of parties to manage a currency), a supply chain, different medical providers sharing patient data (for easy migration) - any system where users would like to easily migrate between providers.
 - Parties no longer from the same authority, potentially untrusted. --> byzantine fault tolerance not for increased software failure resistance (bugs etc), but to tolerate some level of misbehavior
 - While BFT was not warranted in single entity replication (extra cost for some more FT), it should be the standard for heterogenous replication. This is the basis for Blockchain where BFT is the norm.\\
\nc{I don't think we need to be defensive about BFT. It's the norm
in blockchain and blockchain is "hot"}
 -Alternatively, big corporations that use transactional systems that are geo-replicated might be worried about administrative domains: I.e. google spanner having replicas located in authoritarian regimes or out of its own control. does not "trust" its "own" replicas.
 
 -
 - SMR powerful fault tolerant technique. Creates illusion of single Database, thus simplifying application design. Replication is masked.
 \nc{this comes out of no where} \\
 
- ACID transactions: Absolute data integrity. Simplifies concurrency control. Intuitive data access logic.
Concurrent access to retail inventories,  bank balance, stocks is unavoidable. I.e. serializable isolation makes reasoning simple for application developers.

 Interactive TX: Most general for any kind of application.
\fs{ Any system that provides Interactive TX can handle  One shot for example.}\\
 - Goal: Make BFT accessible to the mainstream. Requires scalability and low latency. Cost needs to be tolerable - more reasonable if replication is done as consortium (each party pays for subset).
\nc{I don't think we particularly need to discuss the benefits of transactions or the benefits of interactive transactions yet. I think it's cleaner to say that blockchain, backed up by BFT, has the problem of being totally ordered. Moves the text more quickly.}

\subsection{Existing systems shortcomings}
- systems were designed for different assumptions: single authority low replication (historical use cases: aviation, space, nuclear power) vs high replication degree in a consoritum of untrusted parties\\
- non-interactive transactions\\
- totally ordered\\
- leader based: fairness and scalability bottleneck \\
- Crash Failure model\\
- single sharded\\

\subsection{Key insights}
- ?? \\
- mismatch between requirements of a transactional store and implementation of totally ordered log\\
- implicit partial order suffices for TX 
\\
-EVERYTHING is a partial order: Entire Life cycle is parallel if non-conflicting\\

- flip the problem of trying to add scalability to existing BFT systems. add BFT to efficient CF DBs\\

- put clients in charge: responsible for their own liveness;  standard way to scale a system\\
\nc{is this the "standard" way to scale a system?}
- do single slot binary consensus\\
-
\paragraph{Challenge} Empowering byzantine Clients is complicated and dangerous

\subsection{Positive Implications of Insight/Design}
- robust to censorship and frontrunning (if network not adversarial) as there exist no central authority \\
-  allows commutative/non-conflicting Transactions to both be executed and validated out of order, thus maximizing parallelism. \\
-  minimizes the state, communication and computation load on replicas, as Clients serve as both execution hub and broadcast channel for their own Transactions. \\
\nc{I don't think everyone would agree that this is a good thing,
replicas are big beefy machines at the datacenter, whereas clients
might be lightweight devices where you explicitly don't want to run much computation}
- As any Quorum system, Indicus is inherently load-balanced as there exists no leader bottleneck and all replicas have equal responsibility\\
- In Indicus liveness is a client local property. Unlike SMR, where the entire system halts during view changes, Byzantine participants may stall system progress only for the objects their transactions touch. Hence, the system appears live to any non-conflicting Transaction. View changes are parallel to one another and are only triggered when a participant is interested, which is inline with our "insight" that everything is a partial order.

\subsection{Results discussion}
- latency should not be that much higher compared to Tapir\\
- throughput better than an atomic broadcast version\\
- abort rate doesnt rise too fast; mvtso improves abort rater over OCC (if not, then its useless overhead and we can use OCC)


\subsection{Limitations}
 - Like any speculative system, Indicus is vulnerable to high contention in which case it needs to resort to aborts\\
 - This can especially be exploited by byzantine participants if the network is adversarial.\\
 - 

\subsection{Contribution summary}
 - Present definitions for what it means to provide ACID (specifically Isolation) guarantees in a byzantine setting\\
 - We present design, implementation and evaluation of Indicus, a fully client-driven BFT ACID transactional system \\
 - Design of a leaderless, client-driven execution and replication protocol that does not enforce a total order\\
 - Present 2 flavors of Indicus: Indicus3 and Indicus5 that explore the trade off between replication degree and performance (latency and computational overhead)
 - The design of a more aggressive optimistic concurrency control scheme that is robust to byzantine behaviors
%
There is an increasing tension between the desire to share data online, and the security concerns it entails.
This paper asks the question: how can we enable \textit{mutually distrustful parties} to consistently and reliably
share data, while minimizing centralization?

The ability to share data online offers exciting opportunities.
\iffalse %% the medical record example is for privacy, more related to Obladi, but not BFT Tapir
In the medical domain, for
instance, cloud-based solutions for managing health record offer doctors increased fault-tolerance at lower
cost, and offer patient an easier path to share their medical
history with their entire treatment teams, even when on the road. Opportunities abound in other areas too.
\fi
In banking, systems like SWIFT enable financial institutions to quickly and accurately receive information
such as money transfer instructions; and in manufacturing, online data sharing can improve accountability
and auditing amongst the globally distributed supply chain.

Increased data sharing, however, raises questions of how to \textit{decentralize trust}.
\iffalse
Even when medical records are
encrypted or anonymized, cloud providers or dishonest applications may be able to acquire
sensitive information: for example, tracking the charts accessed by an oncologist can reveal not only whether
a patient has cancer, but also, depending on the frequency of accesses (e.g., the frequency of chemotherapy
appointments), indicate the cancers type and severity.
\fi
Banking institutions must currently place their trust
in the centralized SWIFT’s network to issue payment orders. Sometimes there is even no identifiable source of
trust. Consider the supply chain for the latest iPhone: it spans three continents, and hundreds of different
contractors~\cite{}; neither Apple nor these contractors trust each other, yet all must be willing to agree and share
information about the construction of the same product.


\iffalse %% Yunhao has removed the medical record example
\Yunhao{If I understand correctly, the medical record example is for privacy,
  the banking and the manufacturing example are for trust.
  I can see how BFT solves the later by decentralizing trust, but I cannot
  see how BFT solves the privacy problem. The access pattern of each transaction is visible to all parties.}
\fi

Recognizing this challenge by both the research and industry communities,
much effort has focused
on enabling shared computation between mutually distrustful parties, in the context of byzantine
fault tolerance (BFT), and blockchains.
Systems proposed in the literature of BFT[][] provide the abstraction of
a totally ordered log; the log is agreed upon by the $n$ participants in the system, of which at most $f$ can misbehave.
Each participant executes operations that may touch one to multiple objects in the log.
In the blockchain world,
Bitcoin and Ethureum have become popular distributed computing platforms
providing the same log abstraction and
aiming for decentralizing trust.
Furthermore, Microsoft Azure has launched projects[] that leverage these blockchain platforms and extend the digital transformation
beyond the companies' four walls in a supply chain.

%\Yunhao{This is the first time we mention transactions. I think we should distinguish 2 groups of research: BFT consensus and BFT transactions.}

This paper argues that there exists a fundamental mismatch between the implementation of %the abstraction of
a totally ordered log and the reality of much large-scale distributed processing. Many large-scale distributed
systems consist primarily of unordered and unrelated operations.
For example,
%Alice’s surgery need not be ordered with respect to Bob’s X-rays; likewise,
a product supply chain consists of many concurrent steps
that do not require ordering. Imposing an ordering on non-conflicting operations is not only often
unnecessary, but costly: participants in the shared computation must vote to order operations, store the full state of
the system, and replay the full log for auditing.

While there exists work on mitigating this scalability bottleneck
through sharding~\cite{}, the latent total order requirement introduces unnecessary coordination overhead, as
coordination is performed twice, at the level of individual shards, and across shards. Callinicos~\cite{} and
Omniledger~\cite{}, for instance, runs a full BFT protocol for every operation. This is especially problematic
when workloads are geo-replicated~\cite{}, or when, as in BFT, the replication factor is high.
%\Yunhao{I can see these two are examples of imposing ordering, but cannot see whether they are examples of twice coordination.}
Further, these systems
support transactions under the assumption that their read and write operations are known a priori, which
limits the set of applications that they can support.

As another research trend of mitigating the scalability bottleneck, 
EPaxos[], TAPIR[] and CURP[] only consider the ordering between potentially conflicting operations,
instead of commutative operations.
However, these systems assume the crash-failure model and are non-trivial to be extended to the Byzantine model,
so that they cannot directly solve the problem of data sharing among mutually distrustful parties.

Existing research, in essence, is either attempting to build concurrency control and sharding
functionalities over BFT replication, or integrating these functionalities into a crash-failure replication protocol.
%Existing research, in essence, is attempting to add database-like transactions and sharding
%functionality to a byzantine fault tolerant totally ordered log. We propose instead to flip the problem on its
%head by adding BFT to an efficiently shardable replicated database.
In this paper, we will show how to build these desiring functionality inside a BFT replication protocol.
Specifically, our goal is to \textit{provide the illusion of a centralized shared
log, rather than the non-scalable reality of a totally ordered log.}

%Looking beyond treating transactions as black box requests and leveraging existing commutativity is not a new idea. However, while copious efforts exist to design and build more decentralized systems in order to exploit transaction knowledge for the Crash Failure model, few, if any attempts have been made for the Byzantine Fault Model. This naturally begs the question why so? One explanation is that for BFT systems centralization is in fact highly desirable as it simplifies the problem by pin-pointing a single point of accountability. Since historically, BFT systems have taken a rather niche role for applications that require "additional" safety, the principal design concern has always been maintaining consistency with efficiency and scalability being secondary concerns. Replication was traditionally done by a single authority and thus it is reasonable to assume a primary backup scheme, since neither fairness, nor total ordering were major concerns. As Byzantine Fault Tolerance moves into the mainstream with the popularity ascent of Blockchains these considerations are being revistited. For example, when operating a database as a consortium of mutually distrustful parties it is no longer desirable to grant priviledges to a leader and impose centralization for the sake of simplicity.
Rather than trying to make a traditional BFT system more scalable we ask the question whether we can take the scalability lessons from Crash Failure settings and improve upon their robustness. While this is an attractive avenue, it is far from being straightforward. A natural way to scale a system is to move state and responsabilites to the clients.  
However, in a byzantine setting, giving up the comfort of a bounded and accountable set of replicas opens up the system to a wider set undesirable phenomena. Yet, this is exactly what we will do: Concretely, in this paper we will show how to design a BFT system that is almost entirely client driven, and all the while both safe and live. 


// Unlike SMR that achieves agreement on a result by imposing a total order to evaluate the question, we just evaluate the result directly. Instead of operating multi consensus, I.e. a ledger where reaching consensus implies reaching consensus on every prior decision (like a blockchain), we just do binary consensus for each transaction in a seperate slot. Because its unecessary to reach agreement on this total order, we only need to respect an order to the set of conflicting transactions, a (implicit) partial order suffices.
 Transactions are still able to be totally ordered according to a serializable execution they correspond to by assigning a Timestamp and evaluation Isolation according to it.

Doing so has the following positive outcomes:
\begin{itemize}

\item Indicus is robust to censorship and frontrunning as there exist no central authority (in SMR traditionally a leader) that decides what Transactions enter the system and decides on the ordering of Transactions
\item Indicus allows commutative/non-conflicting Transactions to both be executed and validated out of order, thus maximizing parallelism. (This should increase throughput and reduce latency, because clients arent waiting for all previously sequenced tx to finish. Consequently the tail latency does not dominate throughput as much.)
\item Indicus minimizes the state, communication and computation load on replicas, as Clients serve as both execution hub and broadcast channel for their own Transactions. This avoids quadratic communication communication complexity in the normal case. (Theoretically always, but we keep some for practicality in the fallback)
\item As any Quorum system, Indicus is inherently load-balanced as there exists no leader bottleneck and all replicas have equal responsibility
\item In Indicus liveness is a client local property. Unlike SMR, where the entire system halts during view changes, Byzantine participants may stall system progress only for the objects their transactions touch. Hence, the system appears live to any non-conflicting Transaction
\end{itemize}

Since we do not sequence Transaction execution we require a concurrency control mechanism in order to maintain Isolation between Transactions.
For this purpose, we design and implement a byzantine replicated MVTSO scheme, an aggressive version of Optimistic Concurrency control that empowers Read Transactions in the hope of reducing abort rates. Specifically, we allow to read old version, we allow to read uncommitted writes and we acquire read leases.

Additionally, we propose definitions for what it means to enforce an Isolation level in a transactional system with byzantine participants. These are general purpose formalizations and can serve as guideline for future byzantine Database systems.

In section X, we discuss Limitations of this approach. 



%\section{On the quest for scalable fault tolerance}


\subsection{In light of partial ordering}
Talk more about total order. 
And how we really would like to enforce it when necessary only.
How we abstractly treat it as seperate registers on which we can run independent consensus

Databases do it right: They do a partial order and only show the equivalence. 
Hyperledger is a database in its on way: Exec, order, validate architecture. Uses CF consensus however - could be swapped out.\\


Imposing a shared total order across different machines solves the problem of State Machine Replication, i.e. the objective of maintaining consistent data replicas. It does so however, by a reduction of problem, rather than solving the original goal: providing consistent output. While execution in a common order implies agreement on a common result, this additional step adds coordination overhead, when possibly unecessary, as each transaction is declared to be co-dependent with all other transactions.
Concretely, when requiring agreement on a totally ordered ledger of transactions, achieving agreement on each new request implies achieving agreement on the entire history prefix. This is desirable in settings relying on probabilistic finality (such as permissionless blockchains), however unecessary in traditional BFT/permissioned blockchain settings. In fact, when all transactions are commutative, all requests could reach agreement in parallel and out of order. Thus, in practice, where only a subset of transactions conflict, a partial order suffices. 
This is the bread and butter of decades of database research. Transactions inherently are run concurrently and may interleave arbitrarily, while Concurrency Control mechanisms enforce an implicit partial order by resolving conflicts only when necessary. The premise of such an approach is simple: Databases allow for a natural, partially ordered execution, that merely maintains the equivalence to a sequential execution.
However, traditional databases assume the stronger Crash-Failure model and cannot be modified straightforwardly to tolerate Byzantine Faults. Existing blockchain technologies such as Hyperledger Fabric, consider themselves shared databases, yet fall back to restrictive total order broadcast primitives and hence fundamentally limiting scalability. 


\subsection{A shard of truth}
To combat this scalability bottleneck many systems rely on sharding. Orignially a database technique, sharding partitions the datastore into smaller and faster \textit{shards} that are easier to manage. While sharding approaches in practice (often) induce partial ordering and increase parallelism, this is not a principled property as in this case the application of sharding conflates two objectives:
At its core, sharding horizontally scales hardware resources, in order reduce memory overheads such as storage cost or lookup time. When there exist exploitable spacial locality within the data, i.e. transaction access patterns are limited to a subset of shards
and a suitable partition data mapping is chosen


"Sharding is a type of database partitioning that separates very large databases the into smaller, faster, more easily managed parts called data shards"

Cost and speed for lookups.

How it is more of a hack and conflates objectives.

Partial order is a byproduct of "share nothing". Works, when there is spacial locality in the data. Shards could be co-located with certain demographic etc.  Otherwise coordination is required across shards.


Redundancy can be avoided by integrating replicaiton and transaction layer with concurrency control. See tapir: they show that isolation and replication are not orthogonal.

\subsection{Decentralizing decentralized systems}
Talk about leader based. How PBFT is bottlenecked scalability wise. Reference to BFT Mir here. Multi leaders. When view changes happen the system is not down fully.

Also elaborate how systems that scale to multi-leaders (i.e. MIR, or a hypothetical BFT EPaxos) would still have inherent fairness struggles. Even with rotating leaders the problem does not go away, it is just mitigated. It always gives every byzantine replica the "chance" for damage. With a leaderless approach this does not arrive. Byzantine replicas can only try to equivocate or vote unfavorably.
Push scalability from 1 to n leaders to c clients.
Our crux becomes how to deal with byzantine clients.
What requirements does this put on the system? Byz should not be able to interfere with honest, or at least in a bounded way.
The ideal mechanism: Honest and byz are fully independent, that we cannot guarantee. But we can give a client the keys to its own liveness. Bounded amount of conflicts to pick up (that are not due to concurrency)

\subsection{Linearity}
dont use all to all, avoid large signature overheads if possible?

Maybe dont have this section. 

\subsection{A byzantine empire}
Crash Failure is not enough. Does not extend trivially to weaker and more complex failure domains.

\subsection{Indicus' heel}
Our achilles heel, under congestion you can always abort, and byantine clients can create that congestion if they have the access control. But you expect this to be limited. 
Replicas can enforce exponential backoff on clients.
%\section{On the quest for scalable, decentralized, fault tolerance}
\fs{some intro sentence.} 
In the following we discuss both the limitations of existing solutions and desirable properties that drive Indicus' design. 

\subsection{In light of partial ordering}
Imposing a shared total order across different machines solves the problem of State Machine Replication, by a reduction of problem, rather than solving the original goal: providing consistent output. While execution in a common order implies agreement on a common result, this additional step adds coordination overhead, when possibly unecessary, as each transaction is declared to be co-dependent with all other transactions. 
Concretely, when requiring agreement on a totally ordered ledger of transactions, achieving agreement on each new request implies achieving agreement on the entire history prefix. Many large-scale distributed systems consist primarliy of unordered and unrelated operations. For example, a product supply chain consists of many concurrent steps that do not require ordering and can reach agreement in parallel.  Thus, in practice, where only a subset of transactions conflict, a partial order suffices. 
Existing blockchain technologies \fs{such as Hyperledger Fabric} often consider themselves shared databases, yet fall back to restrictive total order broadcast primitives, hence fundamentally limiting scalability. 


\subsection{A shard of truth}
To combat this scalability bottleneck many systems rely on sharding. While sharding approaches in practice \fs{(often)} can induce partial ordering and increase parallelism, this is not a principled property as the application of sharding conflates two objectives:
At its core, sharding horizontally scales hardware resources, reducing memory overheads such as storage cost or lookup time \fs{, and general CPU load, such as messages and agreement instances to manage}. In order to maintain a consistenct database, sharding requires coordination across shards, typically in the form of a joint two-phase commit and concurrency control protocol. When transactions span multiple shards, the latent total order requirment within each shard introduces redundant coordination overheads, as coordination may need to be performed twice, at the level of individual shards, and across shards. This is especially problematic when workloads are geo-replicated, or when, as in BFT, the replication factor is high. 
We argue, that there exist no optimal calibration for the degree of sharding: When sharding is coarse (few large shards), scalabiliy suffers under the per-shard total order, wheras when sharding is fine-grained (a lot of small shards), cross-shard overheads begin to dominate. 
Note moreover, that chosing a suitable partition data mapping is highly workload dependent and hence \fs{from a system design perspective} is not agnostic to the application.

Zhang et al \cite{zhang2016operation, zhang2015tapir} were the first to point out that by integrating both replication and transaction layer, and consequently relaxing the total order requirement within shards, redundant coordination overheads can be reconciled.
In Indicus we adopt this rationale by co-designing concurrency control and replication, thus naturally exposing existing commutativty between transactions.



\subsection{Decentralizing decentralized systems}
Blockchain technologies are often advertised as a shared databases that decentralize trust (i.e. Hyperledger). In order to to do so, permissioned Blockchains rely on traditional BFT SMR, such as the seminal PBFT protocol \cite{castro1999practical} and its descendants \cite{castro1999practical, kotla2007zyzzyva,  gueta2018sbft, clement2009making, buchman2016tendermint, yin2019hotstuff}, that provide a totally ordered log abstraction.
In practice, most of these solutions intertwine the requirement for a total order with the existance of a dedicated leader replica that acts as centralized sequencer, thus exposing two obvious concerns: scalability and fairness.

A single sequencer exposes a proposal bottleneck, as load is focused entirely on the leader while other replicas sit idle. Moreover, when the leader is faulty, i.e. crashes or misbehaves, the system comes to a halt, requiring expensive re-election schemes to re-gain liveness while maintaining consistency. 

Moreover, a dedicated sequencer exposes a fairness vulnerability. A malicious, or simply rationally biased proposer may censor, delay or frontrun transactions with only limited accountability. Such authority is antithetical to the ethos of decentralized trust and thus highly undesirable. 

These restrictions motivate us to design a leaderless protocol. Concretely, we avoid proposal delegation to an untrusted replica by declaring each client to be its own transaction coordinator.
By shifting responsibility from replicas to clients, the proposal bottleneck becomes the max-capacity of incoming requests replicas can process, while fairness is entirely up to the network delivey. In doing so, the crux of the design becomes how to tolerate byzantine client behavior. In particular, byzantine participants should not be able to interfere with honest participants in an unbounded manner and without being held accountable for their participation. \\


In this paper we show how to complete this Quest, resulting in our design and implementation of Indicus. In the following we define our system model as well as general properties for reasoning about a byzantine database.

%%%%%%%%%%%%%%%%%%%%%%%%%%%5
%-------------------------------------------------------------------------------
\section{Model and Definitions}
\label{section:model}
%-------------------------------------------------------------------------------

%\section{Definition- notes}

\paragraph{Goals:}

- Define what liveness guarantees there should be in a byz system\\
- Want to offer interactive ACID TX in a byz setting \\
- be robust to byz contention\\


\paragraph{Challenges}
- Clients can intertwine their fate with byz participants. When can/should liveness be guaranteed and when should it not. Under which network assumptions?\\
-  Unclear what correct behavior should be. Thus, we need to define what ACID (specifically Isolation) means in a byzantine context. byzantine Clients can execute arbitrary transactions, might not care for guarantees. Begs the question: Should one enforce "safety" for them too?\\
- What does it mean to be robust to byz contention? To what extent is independence possible
%-------------------------------------------------------------------------------
\section{Model and Definitions}
%-------------------------------------------------------------------------------


\subsection{System Model}
We assume the Byzantine Fault Model, in which faulty participants (we distinguish replicas and clients) can (mis-) behave arbitrarily. We denote any participant that follows prescribed protocols as \textit{honest} \fs{lorenzo prefers "correct"}, while faulty participants are dubbed \textit{byzantine}. The database may be partitioned into several replication groups (shards), each of which tolerates a fixed fraction of byzantine replicas. We further assume that there may exist a finite but unbounded number of byzantine clients, and the existance of a strong, but static adversary that can freely coordinate all byzantine participants. We do assume however, the existance of sufficiently hard cryptographic primitives that allow for private/public key signatures and collision-resistant hashes that cannot be compromised by byzantine participants. We denote a message $m$, signed by principal $p$ as $\langle m \rangle_p$. 
 
We make no assumption on network synchrony in order to maintain safety, but in some cases, consistent with known impossibility results \cite{fischer1985impossibility}, may provide liveness only when the network is synchronous and messages are delayed by no more than a fixed but potentially unknown time frame. 
Unlike traditional State Machine Replication protocols in which the liveness of all Clients is correlated with the fate of the system (or often more specifically a leader), our system guarantees liveness not on a system basis, but on a per client basis. Concretely, we only guarantee liveness to clients that follow the protocol. Conversely, an honest client only loses liveness when it intertwines its fate with byzantine clients.\\

Application services may restrict the influence of byzantine clients by enforcing authentication and access control. While potential damage to the database state through authenticated clients cannot be avoided, it is re-traceable by auditing transaction logs.\\

\subsection{System properties}
Indicus offers clients an interactive transaction interface that maintains the view of an ACID compliant state to all honest participants. Byzantine clients instead, may choose whether to experience isolation guarantees, but cannot tamper with "honest observable" state.
Concretely, Indicus defines and implements \textit{byzantine Serializability}. Intuitively this Isolation level guarantees that all honest clients perceive the database as if there serializable. In order to formally capture this we lay some ground work: \footnote{We modify and expand on definitions for BFT-linearizability from \cite{liskov2006tolerating}.} 

Let $Op =  \{r, w\} \times K \times V $ and $Dec = \{Commit, \,Abort\}$ be the sets of possible read/write operations and decisions respectively, where $K$ is the set of existing data items (keys) and $V$ the range of possible values. A \textit{request} $req \in (Op \cup Dec) \times C$ maps any such operation or decision to the issuing Client from set $C$. We denote with $Hon \subseteq C$ the subset of honest Clients. 
%\fs{could alternatively define ops as functions r[x] and w[x], and dec as c/a}

\fs{NEW ALTERNATIVE: Def from \cite{bernstein1987concurrency}: (formal details commented in tex)}

A \textbf{transaction T}, $T \coloneqq (REQ, <_T)$, consists of a set $REQ \coloneqq \{req_1, \dots, req_t \}$ of requests issued by some client $c$, and a partial order relation $<_T$ such that a) $REQ$ contains a finite number of operations, and exactly one decision $req_t = (dec \in Dec, c)$, b) $<_T$ induces a total order for every read and write operation on a shared key $k$. 
%\fs{, either $(r, k, v_r) <_T (w, k, v_w)$ or $(w, k, v_w) <_T (r, k, v_r)$}, and c) all read/write operations precede the decision. 
%\fs{for all read and write operations $(\{r,w\} \times k \times v) <_T dec$.}

\textbf{History H.} A History represents the interleaving on concurrently executed transactions. Concretely, we define a \textit{history} $H \coloneqq (R, <)$ as partial order where $R$ and $<$ are supersets of a finite set of transactions, i.e. $R \supseteq \bigcup_i T_i$, and $< \supseteq \bigcup_i <_{T_i}$ .
Let $Committed(H)$ be the subsequence of all Transactions with $req_{t} = Commit$. A History H is legal if every read request $(r, k, v)$ is preceeded by a matching write request $(w, k, v)$
and there is no other write $(w, k, v')$ inbetween.
% \fs{could formalize this: $\forall (r, k, v).\exists (w, k v). (w,k,v) < (r,k,v) \land \neg \exists (w, k, v'). v\neq v' \land (w, k,v) < (w,k, v') < (r, k, v)$}
For two transactions $T_a$ and $T_b$, we denote $T_a < T_b$ iff $\forall req^a_i, req^b_j \in T_a, T_b. req^a_i < req^b_j$. A history is serial if all transactions are executed sequentially: $\forall T_a, T_b: T_a < T_b \lor T_b < T_a$.

\textbf{Serializable} 
A history H is serializable if there exists a serial permutation H' of Committed(H) such that H' is legal.


\fs{Old TX def in comments. It is shorter... but lorenzo prefers the berstein def}

\iffalse
\textbf{History H.} We define a \textit{history H} as a finite sequence of requests. Informally, a \textit{H} contains the operations (read/write) and decisions (commit/abort) of every transaction issued in the system.

We define a projection $H|_c$ as the subsequence of requests in $H$ that were issued by Client $c$. A sequence of requests $s = req_i \dots req_{i+t}$ in $H|_c$ form a \textit{Transaction} if $req_i$ is the first request by Client $c$ or $req_{i-1} \in Dec \times c$, and if $req_{i+t} \in Dec \times c$. Let $Committed(H)$ be the subsequence of all Transactions with $req_{i+t} = Commit$. A History H is legal if every read request $(r, k, v)$ is preceded by a matching write request $(w, k, v)$ and there is no other write $(w, k, v')$ inbetween.\\
\textbf{Serializable}  
A history H is serializable if there exists a serial permutation H' of Committed(H) such that H' is legal.
\fi

We define a projection $H|_c$ as the subsequence of requests in $H$ that were issued by Client $c$.
We further define:\\
\textbf{Honest History H(P).} Given protocol P, A \textit{history H} is \textit{honest} if it was generated by participants who all follow P, i.e: $H(P) \equiv H = H|_{Hon}$.\\
\textbf{Honest-View Equivalent.} A \textit{history H} is honest-view equivalent to a \textit{history H'} if the Operations and Decisions of all honest Clients are the same and if the final writes are the same.\\
\textbf{Byz-I} Given a protocol $P$ and an isolation level $I$:
A history H is \textit{byzantine-I} if there exists an honest history \textit{H'} such that H is honest-view equivalent to H' and H' satisfies I. \\

This definition captures the requirements for any byzantine tolerant protocol that strives to maintain byzantine Isolation level I.
Informally, a byzantine Isolation level states that the state that honest clients experience must be explicable by an execution in which all participants were honest. Note, that we make no assumptions on the state a byzantine client \textit{chooses} to experience; Byzantine client reads may be arbitrary, i.e. integrity \fs{based on real commit} and legality \fs{based on latest write} of both read values and versions need not be maintained. Indicus maintains byzantine Isolation level Byz-Serializability.\\

\fs{cut the next part about byzantine Atomicity: Need to mention somewhere that only writes experience atomicity. Reads from byz clients are not required to be included}
\fs{Look at tex comments.}
\iffalse
We further define \textit{byzantine Atomicity}. Intuitively, only honest client transactions are guaranteed to experience Atomicity, i.e. all of its operations succeed, or fail jointly. It follows straightforward:\\
\textbf{Byz-A} Given a protocol $P$, a history $H$ and a set of honest client $Hon$. All \textit{Transactions} in $H|_{Hon}$ experience Atomicity, i.e. either all requests are committed, or not a single request is committed.\\
\fs{This may not be ok if validation maintains Invariants based on the assumption that a Tx is atomic or not: I.e. a bank transfer is net 0. In this case we need to enforce atomicity on writes - cannot on reads since they may not be included - replicas must include all shards for its writes (currently i made it optional so that shards reject it themselves- this makes shards oblivious to what items are in other shards).}

Indicus maintains both \textit{Byz-I} for Isolation level Serializable and \textit{Byz-A}. 
In more pragmatic terms, all honest clients experience the ACID properties, whereas byzantine clients may \textbf{choose} whether to experience Atomicity and Isolation for their transactions.
\fi

Next, we define an ideal progress property to limit the influence byzantine participants have on honest clients.  \fs{Byz Indep is the liveness property, whereas Byz Serializability is the safety property. Without Byz Indep any protocol could be trivially Byz serial by just aborting all. }

\fs{this paragraph might perhaps have to go somewhere later, after the protocol is explained and we discuss limitations}


\textbf{Byzantine Independence}
Given a protocol $P$ and an honest Client $c$. The result of a request $r$ issued by $c$ cannot be determinsistically decided by byzantine participants. \fs{in 5f+1 we can strengthen this to hold for byz client c too.} \\

Intuitively, byzantine independence implies, that honest clients' transactions cannot be reliably, strategically aborted by byzantine influence. If the network is controlled by an adversary, this property is unattainable for Indicus.\footnote{We point out, that all leader-based protocols suffer the same fallacy: A byzantine leader may always frontrun/inject requests that influence successive transaction results. In fact, even with strengthened network assumptions, such a a system could not offer Byzantine Independence. } 
In order to offer this property we must strengthen our assumption on the network. Concretely, while the network may be asynchronous, we assume the adversary does not control the network, and hence, may not reliably impact results \footnote{An exception to this are wide-ranged flooding attacks (ddos) which are beyond the scope of this work.}. \fs{Not only can byz not influence results, but they cannot perform front-running - i.e. submit tx at earlier time based on knowledge of later tx. Maybe this distinction needs to be clearer in the definition}

\iffalse
\fs{omit this next part. gracious potentially useful to talk about Fast Path. Uncivil not really}
We adapt and define gracious and uncivil executions based on cite(aardvark) to match our model. (i.e. network not sync either).
\textbf{Gracious Execution}
An execution is gracious iff (a) the execution is synchronous with some
implementation-dependent short bound on message delay (b) all clients and servers behave correctly and (c) there is no contention on the objects relevant to the execution.
\textbf{Uncivil}
An execution is uncivil iff (a) there is no bound on message delay (asynchrony) and (b) up to f servers and an arbitrary number of clients are Byzantine 
\fi



\subsection{Practical Assumptions on Behaviors}
While we allow clients to act arbitrarily and maintain safety unconditionally, in practice, we do not expect clients to intentfully attempt to circumvent other clients progress via targeted congestion. We deem this assumption reasonable as a byzantine client requires both prior knowledge on  honest clients' transaction patterns as well as relevant access control in order to artificially increase object contention. In Indicus, we assume network neutrality in order to defend against explicit \textit{reactive} artificial contention and maintain Byzantine Independence.

We assume that actively detectable byzantine behavior (such as equivocation or other protocol incoherence) is rare as clients are incentivised to maintain standing in a permissioned system.
Moreover, we exect an appropriate level of client performance, from both honest and byzantine clients, to qualify for system participation. Untimely behavior may qualify as misbehavior and elicit client expulsion. 

%%%

\fs{Replicas can enforce client expulsion via access lists, i.e. via timeouts of blacklists. A strawman protocol to agree on client expulsion can look as follows: 1) If a replica has a PoM (i.e. a equivocation proof, false abort without cert) it may forward this evidence and ignore the client immediately. 2) If a client frequently times out on a client (i.e. either its read timestmap expires - we discuss lighter penalties in section Personalized Read Leases - or a fallback election is started) a replica submits a vote of untimeliness by broadcasting to all replicas. Upon reception of 2f+1 votes (this implies that there might be no progress) a replica blacklists the client and forwards the vote, guaranteeing that every correct replica will do the same. (Alternatively we could forward once f+1 are received, but only blacklist when 2f+1 are received. But this means that the client could have been making progress overall, so it is overly conservative). A honest replica may remove all tentative client associated transaction state upon blacklisting, as it is guaranteed that all other honest replicas will do the same. Waiting dependencies must be aborted.}

Given these client expectancies, we design Indicus to minimize costs during gracious execution while defending and bounding the overhead accordingly when misbehavior does occur.



%-------------------------------------------------------------------------------
\section{System Overview}
\label{section:arch}
%-------------------------------------------------------------------------------

%-------------------------------------------------------------------------------
%\section{Protocol}
%-------------------------------------------------------------------------------
\begin{figure*}[!th]
\begin{center}
\includegraphics[width= \textwidth]{./figures/Archi.png}
\end{center}
\caption{{\em Transaction Lifecycle}. Clients execute remote reads (1) and buffer writes (2). For Committment, all involved shards verify isolation (3). If there are conflicting transactions (TX'), replicas in a shard (B) vote to Abort. A client persists a decision (4) that serves as Two-Phase-Commit Vote for each shard (5), and Commits a transaction if all shards vote to commit (6).}
\label{fig:Figure1}
\end{figure*}
\fs{cut figure if it is not useful}
\sys is designed to be scalable and leaderless. Our architecture reflects this ethos. We briefly summarise it here before going into more detail in the later sections. 
In \sys, clients drive the entire transaction life cycle which can be broken down into three stages as shown in Figure \ref{fig:Figure1}: i) Execution, ii) Validation, and iii) Writeback. 
i) Clients \textit{speculatively execute} transactions themselves, invoking only remote read procedure calls (1) and buffering writes locally (2). \two Clients validate initiate a two-phase commit vote to validate speculative execution results for byzantine-serializability (3).
For every involved shard, a client queries potentially inconsistent replicas for their commitment vote, reconciles divergent votes into a single per-shard decision that maintains Isolation, and make this decision durable to avoid replay of contradictory decisions (4). 
iii) Lastly, clients aggregate all shard decisions (5) for atomic commit, return to the application and asynchronously, \textit{writes back} decisions and database updates (6).

Next, we outline the protocols for Execution, Validation and Writeback respectively. 


\fs{longer version in Architecture.tex }






\iffalse
\fs{since this only affect validation, maybe move it there.}
\sys comes in two different flavors, \sys{}3 and \sys{}5 respectively, that rely on varying replication degrees, but implement the same design. \sys{}3 requires $n=3f+1$ replicas \fs{, the minimum bound necessary for BFT SMR?,} per shard to guarantee consistency in the presence of $\leq f$ byzantine replicas. \sys{}5 reduces both latencies during failure free execution and complexity during recovery. \footnote{We believe that consortiums with high performance requirements or high replication degrees are respectively comfortable with paying for additional replicas or tolerating a lower fraction (1/3 vs 1/5) of failures}
For the simplicity of exposition we discuss \sys{}5 for the remainder of the paper, and defer to section X \fs{and/or TR} to describe differences in \sys{}3. In the following, we outline \sys 's execution (unaffected by replication degree), validation and writeback protocols.

\fi



%%-------------------------------------------------------------------------------
\subsection{Execution}
%-------------------------------------------------------------------------------
The goals of \sys{}'s execution protocol are threefold: 1) to offer clients an interactive transaction interface, 2) to be scalable, and 3) to offer honest clients independent operability.

\sys{} achieves all three of these goals though the combination of optimistic client-side execution, and an aggressive, but byzantine resilient, concurrency control (CC) scheme. Rather than submitting stored procedures that must be ordered and executed by replica servers, \sys clients may execute arbitrary transactions by relying on RPC's to conduct transactions themselves. This places computational responsibility with the clients, and allows commutative transactions to execute in parallel. Since \sys clients execute optimistically concurrently, \sys must enforce CC in order to reconcile Isolation conflicts might arise.
By relying on optimistic rather than pessimistic CC schemes such as 2-phase-locking (2PL), \sys{} forgoes costly coordination to acquire locks, and sidesteps the concern of byzantine clients refusing to relinquish locks. 

Before we detail \sys{}'s transaction processing and validation mechanism, we first discuss the CC that \sys implements as the latter are functions of the precise CC requirements. 
 
\subsubsection{Concurrency Control}

%%% Explain MVTSO itself only briefly. Then explain how we make it for for byz
% 1) bound timestamps, 2) make writes visible late only, 3) only allow f+1 matchin uncommitted writes, 4) read timestamps
%Afterwards: explain execution interface. After that explain validation check. Then have overarching example
\iffalse
% we need to change a few things:
0) timestmap bound
1) Read validity
2) deferred writes
3) read uncommitted
4) byz read timestamps (read lock implies you had access control, at that point it is no different than issuing a tx and completing. But we do not want to give the power to abort for no reason. It should be traceable.
\fi

Figure \ref{fig:MVTSOEX} shows an example trace of \sys operational behaviors under different request processing orders (assume f=1) that we use to guide the remaining outline.

\begin{figure*}
\begin{center}
\includegraphics[width= \textwidth]{./figures/MVTSOLargeFont.png}
\end{center}
\caption{\emph{MVTSO behavior for different replica processing orders}. $r_x(C : a_y ,P : a_z)$ denotes that transaction $T_x$ (x being the timestamp of a transaction in this example) reads the version $y$ of object a written by committed transaction $T_y$ and version $z$ from tentative prepared transaction $T_z$. $P_x(RS,WS,DEP)$ denotes a transaction $T_x$'s prepare request (i.e. the validation check) for respective ReadSet (RS), WriteSet (WS) and Dependenc Set (DEP), and $\rightarrow C / A$ denotes the local replica validation outcome (Commit/Abort).} 
\label{fig:MVTSOEX}
\end{figure*}

Our starting point is a Multiversioned Timestamp Ordering scheme (MVTSO) which prescribes a serialization order by assigning a speculative timestamp \textit{before} Execution \cite{bernstein1983multiversion, reed1983implementing, su2017tebaldi}. This allows us to a) only classify concurent transactions as conflicting if their execution results violate the timestamp order (Fig. \ref{fig:MVTSOEX}: 4 bottom), and b) avoid write/write conflicts alltogether by applying them in order. 
When speculative execution results match the pre-defined timestamp order, no aborts are necessary (Fig. \ref{fig:MVTSOEX}: 4 top). \sys challenge is therefore to a) assign appropriate timestamps and b) coordinate execution in a way that maximizes such coherence; the presence of byzantine participants (clients/replicas) complicates this.


In MVTSO, reads operations return the latest written version smaller than the readers' timestamps, while writers attempt to create new versions at their timestamp, but must abort if a higher timestamped read would have "missed the write" by reading a prior version. A transaction may commit only, if all read-write dependencies have committed respectively.

Read operations in \sys have the following sub-goals: \one Honest clients should read valid data, i.e. experience read integrity, \two Honest clients should read fresh data, i.e. minimize staleness and hence maximize commit chance, and \three Reads must provide the context necessary to potentially complete observed write state. 
Clients validate the integrity of reads by requiring replicas to provide a proof of validity, i.e. a set of signatures confirming the committment, or if not yet existant, trusting  only $f+1$ matching replies from discrete replicas. Moreover, in order to guarantee that clients do not read maliciously stale data from their local replica, \sys encourages clients to always read from at least $f+1$ replicas and choose the freshest version (Fig. \ref{fig:MVTSOEX}: 1,2).
Avoiding byzantine influence comes at a cost: Read operations require a synchronous, potentially WAN, rountrip. 

MVTSO intuitively synergizes well with \sys{}'s potenially WAN remote reads as reading from a fixed timestamp helps speculative readers observe consistent snapshots, even when execution is long and consequently interleavings are frequent (Fig. \ref{fig:MVTSOEX}: 6). \fs{at the price of experiencing serializability instead of strict serializability - that is only the case if your timestamps are outdated}.

While we assume that clocks are loosely synchronized across honest participants, byzantine participants may diverge arbitrarily in undesirable fashion. To side-step the additional overhead that a dedicated, bounded timestamping phase \cite{Clairvoyant} incurs, we compromise by allowing clients to optimistically select their own timestamp, but rejecting request timestamps above a Threshold at all honest replicas, thus incentivising clients to select real-time timestamps. In order to facilitate a total serialization order across all clients, we define Timestamps to be tuples of the form $(Time, CID)$. 

In traditional MVTSO, writes become visible to successive reads with higher timestamps immediately. In the presence of byzantine clients this however, is undesirable, as it allows clients to issue singular write operations without the intention of ever committing a transaction. Thus, any honest clients' read that observes, and consequently depends on, a byzantine clients write may be blocked indefinitely, thus compromising its liveness.
Allowing other clients to preemtively commit outstanding write operations is infeasible, as the remaining transaction procedure is known only to the issuing client, while conceiding pessimistic abort permission empowers byzantine participants to obstruct any honest clients' writes. 
To reconcile this dilemma, \sys{} defers all database updates until execution has finalized. Concretely, \sys clients buffer all write operations, and submit them only when attempting to commit the transaction, thereby enabling other \sys client to orderly complete Validation and Writeback steps. 

Clients in \sys distinguish explicitly between \textit{committed} and tentatively, but unconfirmed \textit{prepared} writes: While clients trust and accept every verified \textit{committed} write versions, they accept tentative \textit{prepared} versions only when observing $f+1$ matching results. This ensures that a) at least one honest replica believes that the write can commit, and hence is worth observing, and b) that byzantine replicas cannot violate byzantine independence by reactively inventing transactions that are guaranteed to abort.

Finally, \sys replicas abort all writes that higher timestamped reads "miss" by storing a Read Timestamps (RTS) for each locally processed read (Fig. \ref{fig:MVTSOEX}: 3). While this allows in-execution reads to elicit external effects, much like outstanding write operations discussed above, these are only limited to concurrent transactions with smaller timestamps and hence are straightforward to bypass by re-trying a transaction. To nonetheless limit repeatable abuse , we discuss a practical mechanism to limit byzantine influence in section Y.z (Optional Modifc. personalized read leases).



\subsubsection{Execution interface}
Client applications execute transactions via the following interface. A TX object \textit{TXObj $\coloneqq$ (SeqNo, ClientID, InvolvedShards, ReadSet, WriteSet, Dependencies)}, records the state necessary for Validation.

\iffalse
\begin{figure}
\begin{center}
\includegraphics[width= 0.5\textwidth]{./figures/TxState.png}
\end{center}
\caption{Transaction execution state}
\label{fig:Txstate}
\end{figure}
\fi

\textbf{Begin()} A client begins a transaction by optimistically choosing a timestamp \textit{TS $\coloneqq$ (Time, ClientID)}. \\
\textbf{Write(key, value)}. A client buffers the write: \textit{WriteSet = WriteSet $\cup$ (key, value)}\\
%%%%%%%%%Read protocol%%%%%%%%%%%
\textbf{Read(key, TS, RQS)} 
If \textit{key $\in$ WriteSet} a client returns the buffered write value. Otherwise, the client conducts a remote read:

\fbox{\begin{minipage}{22em}

\textbf{1: C} $\rightarrow$ \textbf{R}: Client sends read request to Replicas
\end{minipage}}\\
Given a hyperparameter Read Quorum Size (RQS), a client sends $m = (Read, key, TS)_c$  to $RQS$ different replicas. Note, that in order to guarantee $\geq RQS$ replies a client might need to send up to $f$ additional requests to compensate for unresponsive/faulty participants ($max(|Replies|) \leq n-f$). 

\fbox{\begin{minipage}{22em}
\textbf{2: R} $\rightarrow$ \textbf{C}: Replica processes client read and replies
\end{minipage}}\\
A replica authenticates the client\footnote{Byzantine replica may ignore read access control. Solving this problem is beyond the scope of this work; we defer to existing solutions \cite{basu2019efficient}.}, and whether the timestamp is within a local highwater mark ($HW = localClock + \delta$). It returns a signed message \text{$\langle \textit{ReadReply, Committed, Prepared} \rangle _R$}, where $Committed \coloneqq (value, version, proof)$ represents the value-version pair with largest committed write version smaller than TS and a proof of commitment,  and $Prepared \coloneqq (value, version, TxID', deps)$ the respective largest uncommitted value-version pair, the associated transaction ID, and the latter transactions potentially uncommitted read-write dependencies. Moreover, a Replica stores a new read timestamp (RTS) for the key: $RTS(key) = RTS(key) \cup TS$ (Fig. \ref{fig:MVTSOEX}: 3). 
\fs{a replica may include a set of prepared values to increase likelihood of client receiving f+1 matching}

\fbox{\begin{minipage}{22em}
\textbf{3: C} ($\rightarrow$ \textbf{R}): Client receives read replies 
\end{minipage}}\\
A client waits for RQS read replies and chooses the biggest valid result \textit{(value, version) $= max_{valid}$(\{Committed\},\{Prepared\}} (Figure \ref{fig:MVTSOEX}: 1,2) . A \textit{Committed} tuple is valid, if the proof confirms commitment, wheras a \textit{Prepared} is valid iff there exist $f+1$ matching \textit{Prepared$_r$} signed by different replicas. The client adds the version to its read set \textit{ReadSet = ReadSet $\cup$ (key, version)} and additionally claims a dependency if it was a \textit{Prepared} version: \textit{Dep = Dep $\cup$ f+1 $\times$ (Prepared$_r$)} . 

\textbf{Commit()} A Client terminates its execution, and computes a unique transaction identifier based on final execution object and its timestamp: $TxID \coloneqq (H(TxObj, TS)$, thus preculding byzantine participants from equivocating transaction contents. It then begins the Validation Phase by issuing a 2PC-Prepare requests to each relevant shard and waiting for all votes.

\textbf{Abort()} A client terminates execution, and broadcasts a request to release all acquired Read Timestamps (RTS). Since writes in \sys are deferred, no other rollback action is necessary.


We briefly discuss some implications of the choice of Read Quorum Size (RQS). Following cases may be distinguished: \one \textbf{$RQS = 1$} A replica may read from just 1 replica at the risk of reading maliciously stale data. If a Client trusts a local replica \textbf{and} the replica is not lagging behind, this can reduce execution latency and consequently minimize conflicting interleavings. \two \textbf{$RQS \geq f+1$} \sys{}'s recommended mode of operation. While side-stepping maliciously stale reads, it is still possible to read (arbitrarily) stale data, either due to inconsistency caused by asynchrony or a byzantine client not fully replicating its transaction. 
\three \textbf{$RQS \geq \frac{n+f+1}{2}$} When reading from a Quorum of sufficient size to overlap with any Validation Quorum (ref section Validation) in at least one honest replica, a client guarantees, that no more \textbf{additional}, beyond previously observed, conflicting writes can be admitted, since such a Quorum of acquired RTS acts as a read-lock. 

\iffalse
Trusting only $f+1$ matching prepared writes has the beneficiary side effect of allowing us to include only transaction identifiers as dependencies, rather than the full transaction, since it is guaranteed that at least one honest replica has stored the transaction. Thus, in the failure free scenario, where clients need not complete transactions $in Dep$, \sys minimizes the meta-data overhead that the recovery protocol imposes.
\fi

We remark, that a byzantine client is not required to follow any of the protocol steps, with the exception of claiming dependencies. By our definition of Byzantine-Isolation, read integrity is only required for honest clients; Byzantine clients may \textit{choose} whether to read legal, or even real data. We do, however, require that no client can fabricate dependencies in order to preclude byzantine clients from indetectibly stalling their own transactions and consequently obstructing liveness for consecutive descendant (second degree...) dependencies.
Trusting only $f+1$ matching prepared writes allows us to only include only transaction identifiers as dependencies, rather than the full transaction, thereby minimizing the common path overhead. When the full transaction information is required to complete potentially stalled transactions, it is guaranteed to be obtainable from an honest replica. We discuss how to complete stalled or slow transactions in section Y (Granting Liveness).





%
\subsection{Validation Check}

Algorithm \ref{mvtso} shows the necessary validation check to preserve Byzantine-Serializability. 
Given a transactions prepare request the validation check returns an Abort vote if a conflict has been detected, and Commit otherwise. 
When execution results match the timestamp order, there are no conflicts (Fig. \ref{fig:MVTSOEX}: 4 top). For each read, a replica verifies that it has not voted to commit a conflicting write (Algorithm \ref{mvtso}, line 3-7). Conversely, for each write, a replica confirms that there exist no previously accepted reads (line 8-10), and no ongoing read transactions (line 11-12) that conflict. Fig. \ref{fig:MVTSOEX}: 4 shows both non-conflicting and conflicting interleavings.
If there are no conflicts, a replica tentatively \textit{prepares} a transaction, making its writes visible and evaluating future transactions against it for conflicts. Regardless of the outcome, a replica garbage collects all Read Timestmaps (RTS) associated with the transactions reads.
It then waits for necessary dependencies (uncommited writes that were read) to be resolved (Figure \ref{fig:MVTSOEX}: 5). We remark, that the concurrency control check is serialized and executed atomically for each transaction.

\begin{theorem}
The set of transactions for which the MVTSO-Check returns Commit is Byzantine-Serializable. 
\end{theorem}
\begin{proof}
See TR.
\end{proof}

\begin{algorithm}
\caption{MVTSO-Check(TX, TS)}\label{mvtso}
\begin{algorithmic}[1]
\If{\textit{$TS > localClock + \delta$}} %  || $TS < lowWM$ || $\exists d \in dep: d.TS < lowWM$} } dont mention garbage collection part here, it only confuses
\State \Return Abort
\EndIf

\For{\textit{$\forall key,version \in \textit{TX.RS}$}}
        \If{$ \exists TX2 \in Committed \cup Prepared: key \in \textit{TX2.WS} $ \newline
        \hspace*{2em} $\land \, version < \textit{TX2.TS} < TS$}  
          \State  \Return Abort, \textit{TX2, (TX2.CommitProof)}  
         \EndIf  
\EndFor

\For{\textit{$\forall key \in \textit{TX.WS}$}}
        \If{$\exists TX2 \in Committed \cup Prepared:$ \newline 
        \hspace*{2em} $\textit{TX2.RS[key].version} < TS < TX2.TS$} 
          \State  \Return Abort, \textit{TX2, (TX2.CommitProof)}
         
        \EndIf
        \If{$\exists RTS \in key.RTS: RTS > TS$} 
          \State  \Return Abort
       \EndIf
\EndFor
\State Prepared.add(TX) 

\While{$\exists d \in dep: d \notin CommitLog \cup AbortLog $)}
\State Suspend
\EndWhile

%structure it in a way that is better
\For{\textit{$\forall d \in dep$}}
		\If{$ d \in AbortLog $}
		\State	Prepare.remove(TX)
		\State \Return Abort, \textit{(TX2.AbortProof)}
		\EndIf
\EndFor
\State \Return Commit
\end{algorithmic}

\end{algorithm}

In order to perform the MVTSO-check, a replica maintains several data strucutres: \one It stores read timestamps, read versions alongside the multiversioned write stores for committed and tenative transactions respectively, to provide efficient evaluation of conflicts.
\two In order to confirm dependency outcomes, replicas log proofs for completed transactions in respective Commit and Abort Log sets, that together induce the ledger of all processed transactions. 
\three To avoid busy waiting when dependencies are not yet resolved, a replica temporarily suspends the MVTSO-check for the current transaction, allowing it to process other transactions pending validation. To facilitate this, it keeps track of an additonal transaction to dependents mapping that allows to identify and resume all suspended MVTSO-checks associated with dependents of a completing transaction.

\iffalse

\textit{Aside:} Consistent with our definition of Byzantine-Isolation, byzantine clients may issue ficticious Read-Sets comprised of arbitrary read versions and values. However, these have only limited external effect on concurrent writes. We distinguish two extreme cases: 1) $read.version \rightarrow 0$: This case is equivalent to simply reading stale data, and effectively reduces MVTSO to TSO as conlflicts are evaluated only on basis of the transaction timestamps (i.e. Abort write if: write.TS < read.TS). 2) $read.version \rightarrow read.TS$: In this case there are no conflicts as a write is never "missed" by a previous read.
\fi

In the following section we will show how to design a replicated validation scheme that upholds Isolation guarantees and reaches a single shard decision, even when replicas within a shard validate in different orders.


%%-------------------------------------------------------------------------------
\subsection{Validation Phase}
%-------------------------------------------------------------------------------

In order to facilitate atomic committment across all shards involved in a transaction \sys clients invoke a Prepare request, inquiring each shard to validate the transaction for conflicts and cast a 2PC vote. Shards in \sys are replicated for fault tolerance.

\sys comes in two different flavors, \sys{}3 and \sys{}5 respectively, that rely on varying replication degrees, but implement the same design. \sys{}3 requires $n=3f+1$ replicas \fs{, the minimum bound necessary for BFT SMR?,} per shard to guarantee consistency in the presence of $\leq f$ byzantine replicas. \sys{}5 reduces both latencies during failure free execution and complexity during recovery. \footnote{We believe that consortiums with high performance requirements or high replication degrees are respectively comfortable with paying for additional replicas or tolerating a lower fraction (1/3 vs 1/5) of failures}
For the simplicity of exposition we discuss \sys{}5 for the remainder of the paper, and defer to section X \fs{and/or TR} to describe differences in \sys{}3. In the following, we outline \sys 's execution (unaffected by replication degree), validation and writeback protocols.

The goal of the validation phase is straightforward:
Shards need to cast a single durable vote, in the following we refer to this as the shard-decision. It is computed on the basis of different replica votes.
The vote needs to respect isolation guarantees
leaderless and partially ordered (i.e. coordination only when necessary)



The goals of the Validation phase are threefold: \one It must decide on a single, durable vote per-shard that maintains Byzantine Serializability (henceforth we refer to this as a \textit{shard-decision}), \two It should be leaderless, and not enforce unecessary ordering for commutative, non-conflicting transactions, and \three It must preserve independent operability. \fs{needs to refer to the fallback}

Satisfying these goals requires ovecoming several challenges. In order to maximize parallelism and embrace partial ordering, Indicus allows replicas to process requests out of order. Consequently, replicas may temporarily diverge, and hence return different validation results. Such divergence must be reconciled in a way that maintains Isolation, but not overly conservatively in order to maximize the ability to commit successfully and bound the impact of byzantine participants. Further, Indicus designates clients as validation coordinator for its own transactions, thus omitting a dedicated replica leader. The respective protocol must tolerate client failures such as crashes, omission/stalling, equivocation, or replays and allow for consistent recovery. 

The Validation protocol can be broken down into two functionalities: Voting and Logging.
Since \sys replicas may process requests in different order, they must reach consensus on a joint  decision. To do so, they cast a vote for their local validation result, which are subsequently democratically aggregated into a decision. In the case of Indicus, the client acts as the transaction coordinator who aggregates and relays results (along with necessary evidence).
In order to maintain consistency in the presence of failures (no two honest replicas finalize different Commit/Abort decisions), this decision must be durable and unique, guaranteeing that a replay of any Writeback is idempotent. However, since a byzantine coordinator cannot be trusted to durably store a decision, nor could we retain liveness during a crash or partition, \sys demands clients to \textit{log} the decision at replicas before returning.
 

The voting step requires a single round-trip to all replicas, whereas the logging phase requires at most one round-trip \fs{in Indicus5 and at most two round-trips in Indicus3.}. When execution is fault- and contention free transactions can be committed on the \textit{Fast-Path} in a single-round trip as an explicit logging round is not necessary.


%%%%%%%%%% protocol
To describe how the protocol operates in detail we follow a single-shard transaction through the system:

\fbox{\begin{minipage}{23em}
\textbf{(1: C $\rightarrow$ R)}: Client sends Prepare request to all Replicas within the Shard.
\end{minipage}}\\
Upon deciding to Commit in the Execution phase, a Client initiates Validation by sending a message $Phase1 \coloneqq \langle Prepare, TxID, TX \rangle_{\sigma_c}$ to all Replicas.

\fbox{\begin{minipage}{23em}
\textbf{(2: R $\rightarrow$ C)}: Replica receives validation request, processes it and returns vote to Client.
\end{minipage}}\\
A replica validates Timestamp and Dependency integrity of the request. It then evaluates Read and Write Sets for Isolation conflicts against its local state using the MVTSO Concurrency Control Check (CCC), as shown in algorithm 1. It returns a message $Phase1R \coloneqq \langle TxID, vote \rangle_r$, and optionally evidence in case it voted to Abort.

\underline{Additional subtlelties}: 
A replica never changes its Voting decision, because re-execution could leave to different results. Once the MVTSO-Ceck completes (i.e. there are no blocking dependencies), a replica starts a timer to monitor the clients progress.

\fbox{\begin{minipage}{23em}
\textbf{(3: C)}: Client waits for vote replies.
\end{minipage}}\\
A client waits for at least $n-f$ ($4f+1$ in Indicus5) distinct replica votes, or more, up to a system specified timeout. 

\fbox{\begin{minipage}{23em}
\textbf{(3a: C)}: Client receives Threshold of matching votes and returns to application. Proceeds to Writeback
\end{minipage}}\\
In any of the following 3 cases, a client may short-circuit waiting for additional votes and omit a dedicated Logging round:
\begin{enumerate}
\item \textbf{$1$ Abort vote w/ Conflicting TX \& CommitCertificate}: A conflict with a commited transaction. The client validates the integrity of the CommitCertificate and returns the shard decision $(TxID, Abort, \langle ConflTX \rangle_{CC})$. 
\item \textbf{$3f+1$ Abort votes w/ Conflicting TX}: A conflict with a prepared, but not yet committed transaction. The client returns the shard decision $(TxID, Abort, \{\langle AbstainVote\rangle_r\})$. 
\item \textbf{$5f+1$ Commit votes}: No conflicts. The client returns the shard decision $(TxID, Commit, \{\langle CommitVote \rangle_r\}$
\end{enumerate}
Any of such Quorums forms a \textit{Shard-Certificate} and proves the decision. A client uses this Certificate to return to its application and issue the Writeback.

\fbox{\begin{minipage}{23em}
\textbf{(3b: C $\rightarrow$ R)}: Client receives divergent results and suggests a consistent decision to Replicas for Logging
\end{minipage}}\\
If a client does not receive the necessary thresholds of votes to return, it must continue on the \textit{Slow-Path}. To do so, it aggregates the votes according to following decison rule:
If there exists a $CommitQuorum \coloneqq \frac{n+f+1}{2}$ of Commit Votes, the Slow-Path decision is Commit, otherwise it is Abort.
A client broadcasts a message $Phase2 \coloneqq (TxID, decision)_c, \{\langle votes \rangle_r\}$.

\underline{Additional subtlelties}: A client forwards a Quorum of $\geq n-f$ votes to the replicas in order to prove the Slow-Path decision is consistent with Isolation guarantees. Note, that a byzantine client may equivocate the decision by relaying different Quorums.

\fbox{\begin{minipage}{23em}
\textbf{(4: R $\rightarrow$ C)}: Replicas receive, validate and echo decision
\end{minipage}}\\
A replica confirms that the Decision matches the Quorum by evaluating the decision rule itself and adopting the decision. It then returns the decision to the client by sending $Phase2R \coloneqq \langle TxID, decision \rangle_r$. Importantly, a replica never changes its decision.

\fbox{\begin{minipage}{23em}
\textbf{(5: C)}: Client returns shard-decision to application and proceeds to Writeback
\end{minipage}}\\
A client waits for a Quorum of $n-f$ matching $Phase2R$ messages. Such a Quorum forms a \textit{Shard-Certificate} and proves the decision. A client uses this Certificate to return to its application and issue the Writeback.

\underline{Additional subtlelties}: If a client equivocated, it will never receive a Shard-Certificate. An honest client however, is guaranteed to receive matching Phase2 replies. 

We consider a decision (Commit, Abort) to be \textit{logged} when it is possible for some Shard-Certificate to exist, i.e. as soon as the necessary certificate Quorums exists at some Replicas.
Figure \ref{fig:FigureSP} summarizes the relevant nomenclature.

\begin{figure}
\begin{center}
\includegraphics[width= 0.5\textwidth]{./figures/Nom2.png}
\end{center}
\caption{Validation Nomenclature, Slow-Path. Note, that a byzantine client may equivocate Phase2 decisions by including Commit and Abort Quorums respectively. Byantine replicas may store multiple votes and decisions.}
\label{fig:FigureSP}
\end{figure}

\subsubsection{Correctness}
We show, that a \textit{logged} decision is final:
\begin{theorem}[Saf]
A logged decision is durable, and there can ever exist \textbf{at most one} logged decision.
\end{theorem}
\begin{proof}
See TR.
\end{proof} 

Note, that since replicas never change their decision, it is possible for there to never be any logged decision if a byzantine client equivocated its Slow-Path Quorums. In order to reconcile this, we design and discuss a recovery mechanism in section X which relaxes the requirement on persisting a decision.  


\begin{theorem} 
Indicus maintains \textit{Byzantine-Serializability}.
\end{theorem}
To prove that this is the case, we show that for any two conflicting transactions, at most one can be committed.
\begin{proof}
See TR.
\end{proof}

\begin{theorem} 
Indicus maintains Byzantine Independence in the absence of network adversary.
\end{theorem}

We show, that once a Client submits a transaction for validation, the result cannot be unilaterally decided by any byzantine participant, be it client or replica.
\begin{proof}
See TR.
\end{proof}

%%-------------------------------------------------------------------------------
\subsection{Writeback}
%-------------------------------------------------------------------------------

Validation occurs on every Shard that a transaction spans. The goal of the Writeback phase is to aggregate all relevant shard-decisions and to inform replicas of finalized Commit or Abort decisions. This is necessary in order for replicas to be able to garbage collect meta-data of ongoing transactions, and to allow consecutive transactions to reliably observe the updated state. Any finalized decision must respect all shard-decisions relevant to a transaction. Thus, all decisions are aggregated according to standard two-phase commit. \sys follows a simple 2PC protocol: Only if all shards agree that a transaction may commit (i.e. there exist commit certificates for every shard), then a transaction may commit. A single abort shart-decision suffices to abort a transaction respectively.

\textbf{1.} A coordinator waits for all necessary shard decisions, including certificates\\
\textbf{2.} The coordinator broadcasts a $Writeback \coloneqq (TX, decision, \{certificates_S \} )$ message to all replicas in all relevant shards.\\
\textbf{3.} A replica validates whether the certificates match the decision, and commits/aborts the transaction by applying the relevant store updates, resuming potentially suspended MVTSO-Checks, and garbage collecting all ongoing transaction state. Replicas store the Writeback message which is used as Commit proof to service reads and justify conflicts, or Abort proof to justify dependency aborts.

We point out, that the Writeback coordinator need not be the client issuing the transaction, but can in fact be an arbitrary party (client or replica) that is interested in completing the Writeback. This follows straightforwardly from Theorem Y: Any certified shard-decision implies the existence of a logged decision, and hence the Writeback phase is idempotent.
We utilize this to drive the recovery protocol outlined next. 


%\subsection{Multi-sharding}

and Multi-shard 2pc

- Optimization: Single shard logging


%%-------------------------------------------------------------------------------
\subsection{Failures}
%-------------------------------------------------------------------------------
- Fallback: election (only starts if not waiting on another dep to avoid early eviction), views, resolution, subtelties with mvtso (block because of dep), necessity even without dependencies. Interested clients, write-back multishard. garbage collection
- Fallback requires an extra round in order to learn about current views to start viewchange, but thats ok: Its co-function with learning about full TX, and checking for existing certificates. Timeout invocation is concurrent with p1 message.



%%-------------------------------------------------------------------------------
\subsection{Optimizations}
%-------------------------------------------------------------------------------


\fs{the current writeback might be unecessary to explain, since we implement the below anyways... Perhaps merge writeback and this subsection}
\paragraph{Single Shard Logging}
\fs{This does not work for Atomic Broadcast! In AB, the voting only happens AFTER the tx has already been logged (i.e. the order has been durably replicated)}

When transaction execution touches multiple shards validation can incur redundant explicit logging overhead. When a Slow-Path is necessary to arrive at a logged decision on S different shards, bandwith is wasted. Consider an example in which $S-1$ shards attempt to log the decision Commit, while a single shard attempts to log an Abort decision. If the latter shard succeeds, the effort of the remaining shards was in vain. \fs{Moreover, logging is always bottlenecked by the slowest shard. }
The culprit of this phenomenon is the delayal of Two-Phase-Commit (2PC) until the Writeback phase. By preemptively making a 2PC decision \textbf{before} logging we can avoid this redundancy. We remark, that even when when all shards agree on a decision, this saves redundant coordination.
Concretely, we designate \textbf{one} involved shard as \textit{logging Shard}, while all other shards remain responsible only for Validation. The logging Shard can be determined via a determinsitic function over the \textit{involved Shards}. A simple load balanced solution may select $loggingS = min(TxID \% involvedShard)$. \fs{This does not work for Atomic Broadcast! In AB, the voting only happens AFTER the tx has already been logged (i.e. the order has been durably replicated).Figure \ref{fig:SingleShardOpt} shows a comparison and the revised structure. dont use this figure in actual paper} In order to log a decision, voting quorums from all involved shards are required. We modify step 3 of the Validation protocol accordingly:

\fbox{\begin{minipage}{21em}
\textbf{Validation (3: C)}: Client waits for vote replies from all involved Shards.
\end{minipage}}
A client aggregates a per-shard decision for each shard according to the \textit{CommitQuorum} rule. If all shard-decisions are Commit, it attempts to log a Commit decision by sending $Phase2 \coloneqq (TxID, Commit, S \times \{CommitQuorum\}$ to all replicas in the designated logging Shard. If a single shard-decision is Abort, it stops waiting for other shard-decisions and attempts to log an Abort decision by instead sending $Phase2 \coloneqq (TxID, Abort, AbortQuorum)$. 

\underline{Additional subtlelties:} A client can go Fast-Path and return to the Writeback phase immediately only if Fast-Path Quorums were received for all shards. 

The remaining Validation protocol proceeds identically to the multi-shard version. Notice, that when only a single shard is involved, no adjustments were made. The Writeback phase instead, may proceed with just the single certificate from the logging Shard


The Fallback protocol is adjusted accordingly: A fallback replica need (and can) only be elected on the logging Shard, simplifying reconciliation and reducing the cost for interested clients. 
To further reduce unecessary load, a client may attempt to first inquire whether decisions exists at the logging shard (Fallback protocol step 1 \& 2), before sending $Rec-Phase1$ messages to all shards in order to gather votes itself (Fallback protocol step 1). 

\iffalse
\begin{figure*}
\begin{center}
\includegraphics[width= \textwidth]{./figures/SingleShard.png}
\end{center}
\caption{Single Shard Optimization}
\label{fig:SingleShardOpt}
\end{figure*}
\fi

 
%In the following we outline the Architecture and Protocols of Indicus, the first highly scalable (hopefully) database that tolerates both byzantine clients and replicas.
\\


%\section{Arch-Notes}

\paragraph{Goals:}

- Indicus should be a transactional database\\
- It should be replicated and tolerate Byzantine behavior for fault tolerance and trust assumptions\\
- It should offer the most general set of transactions, i.e. Interactive Transactions\\
- It should enforce ACID guarantees, specifically Serialiability as Isolation level\\
- It should not be affected by censorship\\
- It should enforce only a partial order on Transations\\
- Leaderless\\
- Shardable\\

\paragraph{Challenges}
- Byzantine behavior requires additional safety precautions; higher replication degree\\
- Providing interactive Transactions requires offering Clients an interface for execution.\\
- Enforcing only a partial order implies the necessity of conflict checks. Concretely, maintaining serializability requires the use of Concurrency control.\\
- If we want no total order and no censorship then we cannot have a leader. This means empowering Clients which is dangerous in a byzantine system: exposes more opportunities for misbehavior.\\
- No leader means replicas may go out of sync, need to maintain consistency view of the database somehow\\
- Multi-shard transactions require additional coordination\\
%%-------------------------------------------------------------------------------
\subsection{Architecture}
%-------------------------------------------------------------------------------
\begin{figure*}[!th]
\begin{center}
\includegraphics[width= \textwidth]{./figures/Archi.png}
\end{center}
\caption{{\em Transaction Lifecycle}. Clients execute remote reads (1) and buffer writes (2). For Committment, all involved shards verify isolation (3). If there are conflicting transactions (TX'), replicas in a shard (B) vote to Abort. A client persists a decision (4) that serves as Two-Phase-Commit Vote for each shard (5), and Commits a transaction if all shards vote to commit (6).}
\label{fig:Figure1}
\end{figure*}


\sys is designed to be scalable and leaderless. Our architecture reflects this ethos. We briefly summarise it here before going into more detail in the later sections. 
In \sys, clients drive the entire transaction life cycle which can be broken down into three stages as shown in Figure \ref{fig:Figure1}: i) Execution, ii) Validation, and iii) Writeback. 
i) Clients \textit{speculatively execute} transactions themselves, invoking only remote read procedure calls and buffering writes locally. Transactions may read and write arbitrarily and can span several data partitions (shards). Moreover, different clients may execute transactions concurrently, performing reads in different order at (a subset of) replicas within each shard. 
ii) Since execution is speculative, and clients are unaware of potential concurrency, they must \textit{validate} their transaction execution for Isolation correctness in order to be able to commit. Intuitively, if there are no conflicting concurrent transactions, and a clients transaction observed a consistent snapshot of the database state then it may commit, and otherwise must retry or abort its transaction. Validation too, is client driven, and may happen in different order, both across different shards, as well as across replicas within each shard. For every relevant shard, a client must query potentially inconsistent replicas for their commitment vote, reconcile divergent votes into a single per-shard decision that maintains Isolation, and make this decision durable to avoid replay of contradictory decisions. 
iii) Lastly, a client aggregates all shard decisions in atomic commit fashion and returns to the application. Asynchronously, it \textit{writes back} the decision to all involved replicas who apply the transaction to their local database state.

Next, we outline the protocols for Execution, Validation and Writeback respectively. 







%\section{Protocol-Notes}

\subsection{TX execution}
Goal:\\
- clients should be able to read valid and consistent data from the database\\
- clients should see the most recent data\\


Challenge\\
- Replicas can be out of sync and byzantine\\
- There might be concurrent transactions ongoing that are conflicting\\



\subsection{Concurrency control}
Goal: \\
- limit aborts\\

challenges:\\
- concurrent read/writes\\
- geo-distributed so execution and validation takes long --> more interleavings\\
- clocks are not perfectly synced\\
- byzantine clients can create artificial congestion\\


\subsection{Validation}
Goal:\\
- maintain ACID guarantees: Isolation, Durability, Atomicity\\
- Be client driven\\

Challenges:\\
- Maintain Isolation while creating highest chance to commit\\
- Tolerate byzantine replicas voting dishonestly\\
- Tolerate Client failures: equivocation, crashes, stalling, replay\\
--> Answer: make final decisions idempotent
-



\subsection{Writeback}
Goal: \\
- Finalize commit/aborts\\
- Maintain consistency\\
- Enable garbage collection\\

Challenges:\\
- client failures (crash, equivocation)\\
- 


\subsection{Optimizations}
Goals:\\
- reduce write aborts\\
- reduce redundancy in validation\\
- reduce read lock impact (byz clients)\\

Challenges:\\
- defend against byz client abuse\\
- respect all shard results\\
- 

\subsection{Failures}
Goals:\\
- be robust to byzantine client\\


challenges:\\
- byzantine clients can stall or equivocate during Validation/Writeback\\
- Granting honest clients liveness --> ability to "realiably" finish all Transactions.\\


%-------------------------------------------------------------------------------
\section{Transaction Execution}
\label{section:exec}
%-------------------------------------------------------------------------------
%-------------------------------------------------------------------------------
\subsection{Execution}
%-------------------------------------------------------------------------------
The goals of \sys{}'s execution protocol are threefold: 1) to offer clients an interactive transaction interface, 2) to be scalable, and 3) to offer honest clients independent operability.

\sys{} achieves all three of these goals though the combination of optimistic client-side execution, and an aggressive, but byzantine resilient, concurrency control (CC) scheme. Rather than submitting stored procedures that must be ordered and executed by replica servers, \sys clients may execute arbitrary transactions by relying on RPC's to conduct transactions themselves. This places computational responsibility with the clients, and allows commutative transactions to execute in parallel. Since \sys clients execute optimistically concurrently, \sys must enforce CC in order to reconcile Isolation conflicts might arise.
By relying on optimistic rather than pessimistic CC schemes such as 2-phase-locking (2PL), \sys{} forgoes costly coordination to acquire locks, and sidesteps the concern of byzantine clients refusing to relinquish locks. 

Before we detail \sys{}'s transaction processing and validation mechanism, we first discuss the CC that \sys implements as the latter are functions of the precise CC requirements. 
 
\subsubsection{Concurrency Control}

%%% Explain MVTSO itself only briefly. Then explain how we make it for for byz
% 1) bound timestamps, 2) make writes visible late only, 3) only allow f+1 matchin uncommitted writes, 4) read timestamps
%Afterwards: explain execution interface. After that explain validation check. Then have overarching example
\iffalse
% we need to change a few things:
0) timestmap bound
1) Read validity
2) deferred writes
3) read uncommitted
4) byz read timestamps (read lock implies you had access control, at that point it is no different than issuing a tx and completing. But we do not want to give the power to abort for no reason. It should be traceable.
\fi

Figure \ref{fig:MVTSOEX} shows an example trace of \sys operational behaviors under different request processing orders (assume f=1) that we use to guide the remaining outline.

\begin{figure*}
\begin{center}
\includegraphics[width= \textwidth]{./figures/MVTSOLargeFont.png}
\end{center}
\caption{\emph{MVTSO behavior for different replica processing orders}. $r_x(C : a_y ,P : a_z)$ denotes that transaction $T_x$ (x being the timestamp of a transaction in this example) reads the version $y$ of object a written by committed transaction $T_y$ and version $z$ from tentative prepared transaction $T_z$. $P_x(RS,WS,DEP)$ denotes a transaction $T_x$'s prepare request (i.e. the validation check) for respective ReadSet (RS), WriteSet (WS) and Dependenc Set (DEP), and $\rightarrow C / A$ denotes the local replica validation outcome (Commit/Abort).} 
\label{fig:MVTSOEX}
\end{figure*}

Our starting point is a Multiversioned Timestamp Ordering scheme (MVTSO) which prescribes a serialization order by assigning a speculative timestamp \textit{before} Execution \cite{bernstein1983multiversion, reed1983implementing, su2017tebaldi}. This allows us to a) only classify concurent transactions as conflicting if their execution results violate the timestamp order (Fig. \ref{fig:MVTSOEX}: 4 bottom), and b) avoid write/write conflicts alltogether by applying them in order. 
When speculative execution results match the pre-defined timestamp order, no aborts are necessary (Fig. \ref{fig:MVTSOEX}: 4 top). \sys challenge is therefore to a) assign appropriate timestamps and b) coordinate execution in a way that maximizes such coherence; the presence of byzantine participants (clients/replicas) complicates this.


In MVTSO, reads operations return the latest written version smaller than the readers' timestamps, while writers attempt to create new versions at their timestamp, but must abort if a higher timestamped read would have "missed the write" by reading a prior version. A transaction may commit only, if all read-write dependencies have committed respectively.

Read operations in \sys have the following sub-goals: \one Honest clients should read valid data, i.e. experience read integrity, \two Honest clients should read fresh data, i.e. minimize staleness and hence maximize commit chance, and \three Reads must provide the context necessary to potentially complete observed write state. 
Clients validate the integrity of reads by requiring replicas to provide a proof of validity, i.e. a set of signatures confirming the committment, or if not yet existant, trusting  only $f+1$ matching replies from discrete replicas. Moreover, in order to guarantee that clients do not read maliciously stale data from their local replica, \sys encourages clients to always read from at least $f+1$ replicas and choose the freshest version (Fig. \ref{fig:MVTSOEX}: 1,2).
Avoiding byzantine influence comes at a cost: Read operations require a synchronous, potentially WAN, rountrip. 

MVTSO intuitively synergizes well with \sys{}'s potenially WAN remote reads as reading from a fixed timestamp helps speculative readers observe consistent snapshots, even when execution is long and consequently interleavings are frequent (Fig. \ref{fig:MVTSOEX}: 6). \fs{at the price of experiencing serializability instead of strict serializability - that is only the case if your timestamps are outdated}.

While we assume that clocks are loosely synchronized across honest participants, byzantine participants may diverge arbitrarily in undesirable fashion. To side-step the additional overhead that a dedicated, bounded timestamping phase \cite{Clairvoyant} incurs, we compromise by allowing clients to optimistically select their own timestamp, but rejecting request timestamps above a Threshold at all honest replicas, thus incentivising clients to select real-time timestamps. In order to facilitate a total serialization order across all clients, we define Timestamps to be tuples of the form $(Time, CID)$. 

In traditional MVTSO, writes become visible to successive reads with higher timestamps immediately. In the presence of byzantine clients this however, is undesirable, as it allows clients to issue singular write operations without the intention of ever committing a transaction. Thus, any honest clients' read that observes, and consequently depends on, a byzantine clients write may be blocked indefinitely, thus compromising its liveness.
Allowing other clients to preemtively commit outstanding write operations is infeasible, as the remaining transaction procedure is known only to the issuing client, while conceiding pessimistic abort permission empowers byzantine participants to obstruct any honest clients' writes. 
To reconcile this dilemma, \sys{} defers all database updates until execution has finalized. Concretely, \sys clients buffer all write operations, and submit them only when attempting to commit the transaction, thereby enabling other \sys client to orderly complete Validation and Writeback steps. 

Clients in \sys distinguish explicitly between \textit{committed} and tentatively, but unconfirmed \textit{prepared} writes: While clients trust and accept every verified \textit{committed} write versions, they accept tentative \textit{prepared} versions only when observing $f+1$ matching results. This ensures that a) at least one honest replica believes that the write can commit, and hence is worth observing, and b) that byzantine replicas cannot violate byzantine independence by reactively inventing transactions that are guaranteed to abort.

Finally, \sys replicas abort all writes that higher timestamped reads "miss" by storing a Read Timestamps (RTS) for each locally processed read (Fig. \ref{fig:MVTSOEX}: 3). While this allows in-execution reads to elicit external effects, much like outstanding write operations discussed above, these are only limited to concurrent transactions with smaller timestamps and hence are straightforward to bypass by re-trying a transaction. To nonetheless limit repeatable abuse , we discuss a practical mechanism to limit byzantine influence in section Y.z (Optional Modifc. personalized read leases).



\subsubsection{Execution interface}
Client applications execute transactions via the following interface. A TX object \textit{TXObj $\coloneqq$ (SeqNo, ClientID, InvolvedShards, ReadSet, WriteSet, Dependencies)}, records the state necessary for Validation.

\iffalse
\begin{figure}
\begin{center}
\includegraphics[width= 0.5\textwidth]{./figures/TxState.png}
\end{center}
\caption{Transaction execution state}
\label{fig:Txstate}
\end{figure}
\fi

\textbf{Begin()} A client begins a transaction by optimistically choosing a timestamp \textit{TS $\coloneqq$ (Time, ClientID)}. \\
\textbf{Write(key, value)}. A client buffers the write: \textit{WriteSet = WriteSet $\cup$ (key, value)}\\
%%%%%%%%%Read protocol%%%%%%%%%%%
\textbf{Read(key, TS, RQS)} 
If \textit{key $\in$ WriteSet} a client returns the buffered write value. Otherwise, the client conducts a remote read:

\fbox{\begin{minipage}{22em}

\textbf{1: C} $\rightarrow$ \textbf{R}: Client sends read request to Replicas
\end{minipage}}\\
Given a hyperparameter Read Quorum Size (RQS), a client sends $m = (Read, key, TS)_c$  to $RQS$ different replicas. Note, that in order to guarantee $\geq RQS$ replies a client might need to send up to $f$ additional requests to compensate for unresponsive/faulty participants ($max(|Replies|) \leq n-f$). 

\fbox{\begin{minipage}{22em}
\textbf{2: R} $\rightarrow$ \textbf{C}: Replica processes client read and replies
\end{minipage}}\\
A replica authenticates the client\footnote{Byzantine replica may ignore read access control. Solving this problem is beyond the scope of this work; we defer to existing solutions \cite{basu2019efficient}.}, and whether the timestamp is within a local highwater mark ($HW = localClock + \delta$). It returns a signed message \text{$\langle \textit{ReadReply, Committed, Prepared} \rangle _R$}, where $Committed \coloneqq (value, version, proof)$ represents the value-version pair with largest committed write version smaller than TS and a proof of commitment,  and $Prepared \coloneqq (value, version, TxID', deps)$ the respective largest uncommitted value-version pair, the associated transaction ID, and the latter transactions potentially uncommitted read-write dependencies. Moreover, a Replica stores a new read timestamp (RTS) for the key: $RTS(key) = RTS(key) \cup TS$ (Fig. \ref{fig:MVTSOEX}: 3). 
\fs{a replica may include a set of prepared values to increase likelihood of client receiving f+1 matching}

\fbox{\begin{minipage}{22em}
\textbf{3: C} ($\rightarrow$ \textbf{R}): Client receives read replies 
\end{minipage}}\\
A client waits for RQS read replies and chooses the biggest valid result \textit{(value, version) $= max_{valid}$(\{Committed\},\{Prepared\}} (Figure \ref{fig:MVTSOEX}: 1,2) . A \textit{Committed} tuple is valid, if the proof confirms commitment, wheras a \textit{Prepared} is valid iff there exist $f+1$ matching \textit{Prepared$_r$} signed by different replicas. The client adds the version to its read set \textit{ReadSet = ReadSet $\cup$ (key, version)} and additionally claims a dependency if it was a \textit{Prepared} version: \textit{Dep = Dep $\cup$ f+1 $\times$ (Prepared$_r$)} . 

\textbf{Commit()} A Client terminates its execution, and computes a unique transaction identifier based on final execution object and its timestamp: $TxID \coloneqq (H(TxObj, TS)$, thus preculding byzantine participants from equivocating transaction contents. It then begins the Validation Phase by issuing a 2PC-Prepare requests to each relevant shard and waiting for all votes.

\textbf{Abort()} A client terminates execution, and broadcasts a request to release all acquired Read Timestamps (RTS). Since writes in \sys are deferred, no other rollback action is necessary.


We briefly discuss some implications of the choice of Read Quorum Size (RQS). Following cases may be distinguished: \one \textbf{$RQS = 1$} A replica may read from just 1 replica at the risk of reading maliciously stale data. If a Client trusts a local replica \textbf{and} the replica is not lagging behind, this can reduce execution latency and consequently minimize conflicting interleavings. \two \textbf{$RQS \geq f+1$} \sys{}'s recommended mode of operation. While side-stepping maliciously stale reads, it is still possible to read (arbitrarily) stale data, either due to inconsistency caused by asynchrony or a byzantine client not fully replicating its transaction. 
\three \textbf{$RQS \geq \frac{n+f+1}{2}$} When reading from a Quorum of sufficient size to overlap with any Validation Quorum (ref section Validation) in at least one honest replica, a client guarantees, that no more \textbf{additional}, beyond previously observed, conflicting writes can be admitted, since such a Quorum of acquired RTS acts as a read-lock. 

\iffalse
Trusting only $f+1$ matching prepared writes has the beneficiary side effect of allowing us to include only transaction identifiers as dependencies, rather than the full transaction, since it is guaranteed that at least one honest replica has stored the transaction. Thus, in the failure free scenario, where clients need not complete transactions $in Dep$, \sys minimizes the meta-data overhead that the recovery protocol imposes.
\fi

We remark, that a byzantine client is not required to follow any of the protocol steps, with the exception of claiming dependencies. By our definition of Byzantine-Isolation, read integrity is only required for honest clients; Byzantine clients may \textit{choose} whether to read legal, or even real data. We do, however, require that no client can fabricate dependencies in order to preclude byzantine clients from indetectibly stalling their own transactions and consequently obstructing liveness for consecutive descendant (second degree...) dependencies.
Trusting only $f+1$ matching prepared writes allows us to only include only transaction identifiers as dependencies, rather than the full transaction, thereby minimizing the common path overhead. When the full transaction information is required to complete potentially stalled transactions, it is guaranteed to be obtainable from an honest replica. We discuss how to complete stalled or slow transactions in section Y (Granting Liveness).






\subsubsection{Preserving isolation}

Algorithm \ref{mvtso} shows the necessary concurrency control check to preserve Byzantine-Serializability. 
Given a transactions prepare request \la{This transaction prepare
  request comes a bit out of the blue---it is not discussed explictly
  in the previous section
  as one of the messages being exchanged} the validation check returns
an Abort vote if a conflict has been detected, and Commit
otherwise. \la{For my taste, there are a lot of COmmit flying
  around... can we just call these Yes or No votes, like BHG?}
When execution results match the timestamp order, there are no
conflicts (Fig. \ref{fig:MVTSOEX}: 4 top). For each read, a replica
verifies that it has not voted to commit a conflicting write
(Algorithm \ref{mvtso}, line 3-7). Conversely, for each write, a
replica confirms that there exist no previously accepted reads (line
8-10), \la{line numbers are off} and no ongoing read transactions
(line 11-12) \la{line numbers off} that conflict. Fig. \ref{fig:MVTSOEX}: 4 shows both non-conflicting and conflicting interleavings.
If there are no conflicts, a replica tentatively \textit{prepares} a transaction, making its writes visible and evaluating future transactions against it for conflicts. Regardless of the outcome, a replica garbage collects all Read Timestmaps (RTS) associated with the transactions reads.
It then waits for necessary dependencies (uncommited writes that were read) to be resolved (Figure \ref{fig:MVTSOEX}: 5). We remark, that the concurrency control check is serialized and executed atomically for each transaction.

\begin{theorem}
The set of transactions for which the MVTSO-Check returns Commit is Byzantine-Serializable. 
\end{theorem}
\begin{proof}
\tr{Proof in Appendix~\ref{}}{Proof in TR~\ref{}}
\end{proof}

\iffalse
\begin{proof}
We first show, that the MVTSO-Check never returns Commit for two conflicting transactions TX1 and TX2.
WLOG, assume that TX1 is validated before TX2 and MVTSO-Check(TX1, TX1.TS) returns Commit. This implies that all reads and writes of TX1 are \textit{Prepared}.
We show by case distinction that TX2's validation cannot return Commit:
\begin{itemize}

\item \textbf{TX1.TS < TX2.TS} By timestamp order, the excution results of TX1 and TX2 should be equivalent to a serial schedule in which TX1 happened \textit{before} TX2. A conflict only arises, if TX2 performed a read that should have seen TX1's write, i.e. if TX2's $read.version < TX1.TS$. By lines 3-7 TX2 must Abstain/Abort since TX1 $\in Prepared$. 

\item \textbf{TX1.TS > TX2.TS:} By timestamp order, the excution results of TX1 and TX2 should be equivalent to a serial schedule in which TX1 happened \textit{after} TX2. A conflict only arises, if TX2 attempts to write a version that TX1 should have observed, i.e. if TX1's $read.version < TX2.TS$. By lines 8-10 TX2 must Abstain/Abort since TX1 $\in Prepared$. 
\end{itemize}

We now show, that every correct clients transaction is \textit{legal}, i.e. its reads are based on committed writes. This follow straightforward from Execution protocol: A correct client only reads values that are either committed, or optimistically reads by claiming a dependency on uncommitted writes. By lines 14-18, a TX with dependencies only Commits if all of its dependencies Commit, implying that it read committed writes.

Consequently, all correct clients experience Serializability, and thus, the MVTSO-check maintains Byzantine-Serializability.
\end{proof}

\fi

\begin{algorithm}
\caption{MVTSO-Check(TX, TS)}\label{mvtso}
\begin{algorithmic}[1]
\If{\textit{$TS > localClock + \delta$}} %  || $TS < lowWM$ || $\exists d \in dep: d.TS < lowWM$} } dont mention garbage collection part here, it only confuses
\State \Return Abort
\EndIf

\For{\textit{$\forall key,version \in \textit{TX.RS}$}}
	\If{version > TS} \Return PoM
	\EndIf
        \If{$ \exists TX2 \in Committed \cup Prepared: key \in \textit{TX2.WS} $ \newline
        \hspace*{2em} $\land \, version < \textit{TX2.TS} < TS$}  
          \State  \Return Abort, \textit{TX2, (TX2.CommitProof)}  
         \EndIf  
\EndFor

\For{\textit{$\forall key \in \textit{TX.WS}$}}
        \If{$\exists TX2 \in Committed \cup Prepared:$ \newline 
        \hspace*{2em} $\textit{TX2.RS[key].version} < TS < TX2.TS$} 
          \State  \Return Abort, \textit{TX2, (TX2.CommitProof)}
         
        \EndIf
        \If{$\exists RTS \in key.RTS: RTS > TS$} 
          \State  \Return Abort
       \EndIf
\EndFor
\State Prepared.add(TX) 

\If{$\exists invalid d \in dep$} \Return Pom ~~//no $f+1$ matching
\EndIf
\While{$\exists d \in dep: d \notin CommitLog \cup AbortLog $)}
\State Suspend
\EndWhile

%structure it in a way that is better
\For{\textit{$\forall d \in dep$}}
		\If{$ d \in AbortLog $}
		\State	Prepare.remove(TX)
		\State \Return Abort, \textit{(TX2.AbortProof)}
		\EndIf
\EndFor
\State \Return Commit
\end{algorithmic}

\end{algorithm}

In order to perform the MVTSO-check, a replica maintains several data strucutres: \one It stores read timestamps, read versions alongside the multiversioned write stores for committed and tenative transactions respectively, to provide efficient evaluation of conflicts.
\two In order to confirm dependency outcomes, replicas log proofs for completed transactions in respective Commit and Abort Log sets, that together induce the ledger of all processed transactions. 
\three To avoid busy waiting when dependencies are not yet resolved, a replica temporarily suspends the MVTSO-check for the current transaction, allowing it to process other transactions pending validation. To facilitate this, it keeps track of an additonal transaction to dependents mapping that allows to identify and resume all suspended MVTSO-checks associated with dependents of a completing transaction.

\iffalse

\textit{Aside:} Consistent with our definition of Byzantine-Isolation, byzantine clients may issue ficticious Read-Sets comprised of arbitrary read versions and values. However, these have only limited external effect on concurrent writes. We distinguish two extreme cases: 1) $read.version \rightarrow 0$: This case is equivalent to simply reading stale data, and effectively reduces MVTSO to TSO as conlflicts are evaluated only on basis of the transaction timestamps (i.e. Abort write if: write.TS < read.TS). 2) $read.version \rightarrow read.TS$: In this case there are no conflicts as a write is never "missed" by a previous read.
\fi

In the following section we will show how to design a replicated validation scheme that upholds Isolation guarantees and reaches a single shard decision, even when replicas within a shard validate in different orders.

%-------------------------------------------------------------------------------
\subsection{Validation Phase}
%-------------------------------------------------------------------------------

In order to facilitate atomic committment across all shards involved in a transaction \sys clients invoke a Prepare request, inquiring each shard to validate the transaction for conflicts and cast a 2PC vote. Shards in \sys are replicated for fault tolerance.

\sys comes in two different flavors, \sys{}3 and \sys{}5 respectively, that rely on varying replication degrees, but implement the same design. \sys{}3 requires $n=3f+1$ replicas \fs{, the minimum bound necessary for BFT SMR?,} per shard to guarantee consistency in the presence of $\leq f$ byzantine replicas. \sys{}5 reduces both latencies during failure free execution and complexity during recovery. \footnote{We believe that consortiums with high performance requirements or high replication degrees are respectively comfortable with paying for additional replicas or tolerating a lower fraction (1/3 vs 1/5) of failures}
For the simplicity of exposition we discuss \sys{}5 for the remainder of the paper, and defer to section X \fs{and/or TR} to describe differences in \sys{}3. In the following, we outline \sys 's execution (unaffected by replication degree), validation and writeback protocols.

The goal of the validation phase is straightforward:
Shards need to cast a single durable vote, in the following we refer to this as the shard-decision. It is computed on the basis of different replica votes.
The vote needs to respect isolation guarantees
leaderless and partially ordered (i.e. coordination only when necessary)



The goals of the Validation phase are threefold: \one It must decide on a single, durable vote per-shard that maintains Byzantine Serializability (henceforth we refer to this as a \textit{shard-decision}), \two It should be leaderless, and not enforce unecessary ordering for commutative, non-conflicting transactions, and \three It must preserve independent operability. \fs{needs to refer to the fallback}

Satisfying these goals requires ovecoming several challenges. In order to maximize parallelism and embrace partial ordering, Indicus allows replicas to process requests out of order. Consequently, replicas may temporarily diverge, and hence return different validation results. Such divergence must be reconciled in a way that maintains Isolation, but not overly conservatively in order to maximize the ability to commit successfully and bound the impact of byzantine participants. Further, Indicus designates clients as validation coordinator for its own transactions, thus omitting a dedicated replica leader. The respective protocol must tolerate client failures such as crashes, omission/stalling, equivocation, or replays and allow for consistent recovery. 

The Validation protocol can be broken down into two functionalities: Voting and Logging.
Since \sys replicas may process requests in different order, they must reach consensus on a joint  decision. To do so, they cast a vote for their local validation result, which are subsequently democratically aggregated into a decision. In the case of Indicus, the client acts as the transaction coordinator who aggregates and relays results (along with necessary evidence).
In order to maintain consistency in the presence of failures (no two honest replicas finalize different Commit/Abort decisions), this decision must be durable and unique, guaranteeing that a replay of any Writeback is idempotent. However, since a byzantine coordinator cannot be trusted to durably store a decision, nor could we retain liveness during a crash or partition, \sys demands clients to \textit{log} the decision at replicas before returning.
 

The voting step requires a single round-trip to all replicas, whereas the logging phase requires at most one round-trip \fs{in Indicus5 and at most two round-trips in Indicus3.}. When execution is fault- and contention free transactions can be committed on the \textit{Fast-Path} in a single-round trip as an explicit logging round is not necessary.


%%%%%%%%%% protocol
To describe how the protocol operates in detail we follow a single-shard transaction through the system:

\fbox{\begin{minipage}{23em}
\textbf{(1: C $\rightarrow$ R)}: Client sends Prepare request to all Replicas within the Shard.
\end{minipage}}\\
Upon deciding to Commit in the Execution phase, a Client initiates Validation by sending a message $Phase1 \coloneqq \langle Prepare, TxID, TX \rangle_{\sigma_c}$ to all Replicas.

\fbox{\begin{minipage}{23em}
\textbf{(2: R $\rightarrow$ C)}: Replica receives validation request, processes it and returns vote to Client.
\end{minipage}}\\
A replica validates Timestamp and Dependency integrity of the request. It then evaluates Read and Write Sets for Isolation conflicts against its local state using the MVTSO Concurrency Control Check (CCC), as shown in algorithm 1. It returns a message $Phase1R \coloneqq \langle TxID, vote \rangle_r$, and optionally evidence in case it voted to Abort.

\underline{Additional subtlelties}: 
A replica never changes its Voting decision, because re-execution could leave to different results. Once the MVTSO-Ceck completes (i.e. there are no blocking dependencies), a replica starts a timer to monitor the clients progress.

\fbox{\begin{minipage}{23em}
\textbf{(3: C)}: Client waits for vote replies.
\end{minipage}}\\
A client waits for at least $n-f$ ($4f+1$ in Indicus5) distinct replica votes, or more, up to a system specified timeout. 

\fbox{\begin{minipage}{23em}
\textbf{(3a: C)}: Client receives Threshold of matching votes and returns to application. Proceeds to Writeback
\end{minipage}}\\
In any of the following 3 cases, a client may short-circuit waiting for additional votes and omit a dedicated Logging round:
\begin{enumerate}
\item \textbf{$1$ Abort vote w/ Conflicting TX \& CommitCertificate}: A conflict with a commited transaction. The client validates the integrity of the CommitCertificate and returns the shard decision $(TxID, Abort, \langle ConflTX \rangle_{CC})$. 
\item \textbf{$3f+1$ Abort votes w/ Conflicting TX}: A conflict with a prepared, but not yet committed transaction. The client returns the shard decision $(TxID, Abort, \{\langle AbstainVote\rangle_r\})$. 
\item \textbf{$5f+1$ Commit votes}: No conflicts. The client returns the shard decision $(TxID, Commit, \{\langle CommitVote \rangle_r\}$
\end{enumerate}
Any of such Quorums forms a \textit{Shard-Certificate} and proves the decision. A client uses this Certificate to return to its application and issue the Writeback.

\fbox{\begin{minipage}{23em}
\textbf{(3b: C $\rightarrow$ R)}: Client receives divergent results and suggests a consistent decision to Replicas for Logging
\end{minipage}}\\
If a client does not receive the necessary thresholds of votes to return, it must continue on the \textit{Slow-Path}. To do so, it aggregates the votes according to following decison rule:
If there exists a $CommitQuorum \coloneqq \frac{n+f+1}{2}$ of Commit Votes, the Slow-Path decision is Commit, otherwise it is Abort.
A client broadcasts a message $Phase2 \coloneqq (TxID, decision)_c, \{\langle votes \rangle_r\}$.

\underline{Additional subtlelties}: A client forwards a Quorum of $\geq n-f$ votes to the replicas in order to prove the Slow-Path decision is consistent with Isolation guarantees. Note, that a byzantine client may equivocate the decision by relaying different Quorums.

\fbox{\begin{minipage}{23em}
\textbf{(4: R $\rightarrow$ C)}: Replicas receive, validate and echo decision
\end{minipage}}\\
A replica confirms that the Decision matches the Quorum by evaluating the decision rule itself and adopting the decision. It then returns the decision to the client by sending $Phase2R \coloneqq \langle TxID, decision \rangle_r$. Importantly, a replica never changes its decision.

\fbox{\begin{minipage}{23em}
\textbf{(5: C)}: Client returns shard-decision to application and proceeds to Writeback
\end{minipage}}\\
A client waits for a Quorum of $n-f$ matching $Phase2R$ messages. Such a Quorum forms a \textit{Shard-Certificate} and proves the decision. A client uses this Certificate to return to its application and issue the Writeback.

\underline{Additional subtlelties}: If a client equivocated, it will never receive a Shard-Certificate. An honest client however, is guaranteed to receive matching Phase2 replies. 

We consider a decision (Commit, Abort) to be \textit{logged} when it is possible for some Shard-Certificate to exist, i.e. as soon as the necessary certificate Quorums exists at some Replicas.
Figure \ref{fig:FigureSP} summarizes the relevant nomenclature.

\begin{figure}
\begin{center}
\includegraphics[width= 0.5\textwidth]{./figures/Nom2.png}
\end{center}
\caption{Validation Nomenclature, Slow-Path. Note, that a byzantine client may equivocate Phase2 decisions by including Commit and Abort Quorums respectively. Byantine replicas may store multiple votes and decisions.}
\label{fig:FigureSP}
\end{figure}

\subsubsection{Correctness}
We show, that a \textit{logged} decision is final:
\begin{theorem}[Saf]
A logged decision is durable, and there can ever exist \textbf{at most one} logged decision.
\end{theorem}
\begin{proof}
See TR.
\end{proof} 

Note, that since replicas never change their decision, it is possible for there to never be any logged decision if a byzantine client equivocated its Slow-Path Quorums. In order to reconcile this, we design and discuss a recovery mechanism in section X which relaxes the requirement on persisting a decision.  


\begin{theorem} 
Indicus maintains \textit{Byzantine-Serializability}.
\end{theorem}
To prove that this is the case, we show that for any two conflicting transactions, at most one can be committed.
\begin{proof}
See TR.
\end{proof}

\begin{theorem} 
Indicus maintains Byzantine Independence in the absence of network adversary.
\end{theorem}

We show, that once a Client submits a transaction for validation, the result cannot be unilaterally decided by any byzantine participant, be it client or replica.
\begin{proof}
See TR.
\end{proof}
%-------------------------------------------------------------------------------
\subsection{Writeback}
%-------------------------------------------------------------------------------

Validation occurs on every Shard that a transaction spans. The goal of the Writeback phase is to aggregate all relevant shard-decisions and to inform replicas of finalized Commit or Abort decisions. This is necessary in order for replicas to be able to garbage collect meta-data of ongoing transactions, and to allow consecutive transactions to reliably observe the updated state. Any finalized decision must respect all shard-decisions relevant to a transaction. Thus, all decisions are aggregated according to standard two-phase commit. \sys follows a simple 2PC protocol: Only if all shards agree that a transaction may commit (i.e. there exist commit certificates for every shard), then a transaction may commit. A single abort shart-decision suffices to abort a transaction respectively.

\textbf{1.} A coordinator waits for all necessary shard decisions, including certificates\\
\textbf{2.} The coordinator broadcasts a $Writeback \coloneqq (TX, decision, \{certificates_S \} )$ message to all replicas in all relevant shards.\\
\textbf{3.} A replica validates whether the certificates match the decision, and commits/aborts the transaction by applying the relevant store updates, resuming potentially suspended MVTSO-Checks, and garbage collecting all ongoing transaction state. Replicas store the Writeback message which is used as Commit proof to service reads and justify conflicts, or Abort proof to justify dependency aborts.

We point out, that the Writeback coordinator need not be the client issuing the transaction, but can in fact be an arbitrary party (client or replica) that is interested in completing the Writeback. This follows straightforwardly from Theorem Y: Any certified shard-decision implies the existence of a logged decision, and hence the Writeback phase is idempotent.
We utilize this to drive the recovery protocol outlined next. 



%-------------------------------------------------------------------------------
\section{Transaction Recovery}
\label{section:rec}
%-------------------------------------------------------------------------------

%-------------------------------------------------------------------------------
\subsection{Granting Liveness}
%-------------------------------------------------------------------------------
\sys operates under the premise that clients experience progress indepentently - there exists no shared notion of system progress. Byzantine clients, may bring execution, validation or writeback to a halt for their own transactions: they may stall during all phases, or equivocate during validation logging, causing replicas to diverge on decision values (Commit/Abort) and thus prohibiting the generation of shard-certificates. When all transactions are commutative this phenomenon requires no action: Any client \textit{chooses} whether to adhere to the protocol and experience progress, or not. However, when transactions depend on each other, either explicitly by claiming dependencies on uncommitted write), or implicitly through read-write conflicts, liveness is no longer independent. For instance, a dependency that does not progress may cause the dependent to block indefinetely, while an uncommited, but insufficiently replicated, write might cause consecutive read transactions to abort.
In order to allow clients to safely intertwine their fate with concurrent transactions, \sys must provide clients the tools reclaim their own liveness. We define Liv: 
\begin{theorem}[Liv] 
Every transaction that an honest client is \textit{interested} in eventually completes.
\end{theorem}

Intuitively, honest clients enjoy this progress property granted that they can reliably obtain shard-certificates as any client can conduct the writeback phase. To achieve this, we must relax the requirement that replicas may never alter their decision, while preserving Theorem Saf.

A naive solution that allows any client to drive another clients protocol is problematic, since \textit{interested} clients could concurrently make inconsistent decisions, thus inhibiting progress. Likewise, coordinator election across client is undesirable as there may exist an unbounded number of electable byzantine clients that may not constructively aid in reconciliation. In \sys we circumvents this by letting concurrent clients replay non critical sections of the protcol, while delegating critical responsibility to the bounded replica set. Concretely, we design a mechanism to endorse a dedicated \textit{Fallback} replica responsible for reconciling diverged replica decisions. Since the number of faulty replicas is bounded, at most $f+1$ leader elections are necessary to make progress when the network is synchronous. \sys challenge is to guarantee a live round-robin election while keeping clients responsible for their own progress. We remark, that in an asynchronous network, deterministic leader election might not be possible  \cite{fischer1985impossibility}.  \fs{no non-randomized protocol can reach agreement in an async setting. (randomized exceptions: BenOr, Honeybadger, BEAT)}.

On a high level, the recovery mechanism operates as follows: \textit{Interested} clients attempt to run the validation protocol themselves. If a client suspects decision divergence, it issues a \textit{Fallback invocation} request, electing a \textit{Fallback} replica that reconcile decisions. Such recovery protocol is reminiscent of \textit{view-changes} in traditional BFT SMR protocols, but differs in two core aspects: a) While control changes between fallback replicas (leaders), the protocol is still client driven and linear in complexity, and b) The fallback mechanism impacts only the ongoing transaction. Independent concurrent transactions are unaffected. Figure \ref{fig:FigureFBnom} shows an example scenario requiring reconciliation, and gives on overview of the  Fallback protocol message pattern.

\begin{figure}
\begin{center}
\includegraphics[width= 0.5\textwidth]{./figures/FBNom.png}
\end{center}
\caption{Fallback Scenario. 1. A byzantine client equivocates decisions. 2. An interested client invokes a Fallback Replica (FB) election. 3. An honest FB reconciles a single decision.}
\label{fig:FigureFBnom}
\end{figure}


Below, we detail the protocol by following a transactions life cycle. To start the protocol we assume an \textit{interested} client is in possession of the respective transactions $Prepare$ request, signed by the orignial issuer. This information may be obtained during the validation of the clients own transaction, either as full dependency or conflict. 
%%%%%%%%%%%%%%%%%%%%%%%%%%%%%%%%%%%% WRITEBACK HAS ALREADY HAPPENED


%%Cut the protocol re-iterations?

\fbox{\begin{minipage}{23em}
\textbf{(1: C $\rightarrow$ R)}: Client issues Prepare request and inquires current protocol state.
\end{minipage}}\\
An interested client broadcast a message $Backup-Phase1 \coloneqq (TX.Phase1, CID)$ to all replicas in \textit{InvolvedShards} of the transaction. \\
\fbox{\begin{minipage}{23em}
\textbf{(2: R $\rightarrow$ C)}: Replica marks Client as interested and returns its state.
\end{minipage}}\\
Replicas execute an un-seen $Phase1$ (see Validation, step 2), and reply with its vote. Otherwise, a replica replies with its current decision value and the corresponding view ($Phase2R$), or, if not existant, with its origninal vote ($Phase1R$), and lastly, its current view.\\
\underline{Finalized decisions:} If a replica has already received a Writeback message for a transaction it ignores all Fallback related client requests and simply returns the decision and corresponding certificate. An intersted client can return to the Writeback phase immediately and forward the response.\\
\fbox{\begin{minipage}{23em}
\textbf{(3a: C $\rightarrow$ R)}: A client receives the necessary information to proceed to Writeback.
\end{minipage}}\\
A client either a) receives a Fast-Path threshold of $Phase1R$ messages, or b) receives sufficiently many matching (in decision and view) $Phase2R$ messages to assemble a shard-certificate. \\
\fbox{\begin{minipage}{23em}
\textbf{(3b: C $\rightarrow$ R)}: A client needs to go Slow-Path or observes inconsistency and requires assistance for resolution.
\end{minipage}}\\
If a client receives insufficient $Phase2R$ replies to return, the client itself broadcasts $Phase2$ messages using the received $Phase1R$ or $Phase2R$ responses as proof in order to log a decision. If instead, a client receives inconsistent $Phase2R$ replies (either decision or view), it invokes a fallback election, broadcasting a message $InvokeFB \coloneqq (TxID, CID, \{view\}_R$ that includes a Quorum of replicas' current views.
Inconsistent replies from the same view $v$ additionally constitute a Proof of Misbehavior (PoM) for the issuing client or replica. \\
\fbox{\begin{minipage}{23em}
\textbf{(4a: R $\rightarrow$ C)}:  Replicas receive $Phase2$ messages and replies with decision
\end{minipage}}\\
A replica buffers requests until the orginial client timer expires, and, upon expiration, adopts the decision if it has stored none. It returns a corresponding $Phase2R$ message. \\
\fbox{\begin{minipage}{21em}
\textbf{(4b: R $\rightarrow$ R(FB))}: Replicas receives Fallback invocation and starts election
\end{minipage}}\\
If a replica receives an $InvokeFB$ message and the original client has timed-out it attempts to elect a new \textit{Fallback replica}. To do so, it follows the \textbf{View Change Rules} and adopts new view $current.view = v+1$ if $v \geq current.view$. 
Replicas send an $ElectFB \coloneqq (TxID, decision, current.view)_R$ to replica $current.view + TxID \% n$. \\
\underline{\textbf{View Change Rules:}} \one Replicas only adopt a view $v+1$ if the view set includes $3f+1$ votes from view $v$. \textit{Vote subsumtion:} A view $v$ may count as a vote for all $v' \leq v$. \two Replicas that lag behind, may safely skip ahead to the maximum view $v$ present $f+1$ times, since, by induction, $\geq 3f+1$ replicas must have claimed to be in a view $\geq v-1$. Thus, any possible divergence can be reconciled in a single step, as there must exist a view $v' \geq max(honest.views) -1 $ for which at least $f+1$ honest votes exist in any Quorum of $4f+1$ votes.\\
\underline{Additional subtlelties:} Replicas enforce exponential time-outs for new elections. In absence of honest interested clients, byzantine clients may invoke fallback elections at a subset of replicas, thus inhibiting true election and skipping select replicas' terms. To avoid artificially increased timeouts and non-skipping candidates, replicas may forward the $ElectFB$ message to all other replicas. Replicas that receive $f+1$ forwarded messages, adopt larger views, and forward the $ElectFB$ message themselves, thus ensuring election for each view.
\footnote{This optimization is not necessary for "theoretical" liveness. To avoid unecessary all-to-all communication, it may only be enforced for views $v > T$, where T is a system hyperparameter }\\
\fbox{\begin{minipage}{21em}
\textbf{(5: R(FB) $\rightarrow$ R)}:  Fallback aggregates and echos election messages.
\end{minipage}}
The Fallback replica considers itself elected upon receiving a Quorum of $4f+1$ ElectFB messages. It then uses this set of messages to reconcile a decision. It broadcasts a message $FBdec \coloneqq \langle RID, dec, view, \{ElectFB_r\} \rangle_R$, including proof for the new decision. \\
\textbf{Decision Reconciliation Rule:} \textit{$dec_{new}$ $=$ maj(\{Elect.decision\})}. If a logged decision exists (implies that a shard-decision might exist), any Quorum of $4f+1$ Elect messages is guaranteed to contain $2f+1$ matching decisions (a majority). Otherwise, an arbitrary decision qualifies since at least one honest replica must have agreed to it.\\
\underline{Additional subtlelties:} The attentive reader may notice that Elect messages do not include a corresponding view. Decisions need not match in views: By the Decision Reconciliation Rule, if there ever existed a logged decision $d$ with matching views, then all future $dec_{new} = d$.\\
\fbox{\begin{minipage}{21em}
\textbf{(6: R $\rightarrow$ C)}:  Replicas echo decision to interested clients
\end{minipage}}
Replicas receive a $FBdec$ message, adopt $decision = (FBdec.dec, FBdec.view)$ and $current.view = FBdec.view$ if $FBdec.view \geq current.view$, and forward the respective decision to all interested clients. Replicas that time out waiting for a $FBdec$ message proactively forward their current view to all interested clients. \\
\fbox{\begin{minipage}{21em}
\textbf{(7: C}: A client starts Writeback phase or restarts Fallback invocation
\end{minipage}}
An interested client returns to the Writeback phase upon receiving sufficiently matching decisions. If it times out, or receives inconsistent decisions it inquires a new set of current views and re-starts Fallback starts a new broadcast $InvokeFB \coloneqq (TxID, CID, \{view\}_R$.\\
\underline{Additional subtlelties:} An honest client \textit{expects} to receive $4f+1$ matching decisions and will continue to wait for decisions from the same view up to a timeout. Eventually a logged decision must exist and time-outs grow enough to guarantee a client will receive a shard-certificate. \fs{In section X (optimization) we discuss a practical optimization: Async extra phase.}\\
%%%%%%%
Given synchrony, any honest client that follows the Fallback protocol experiences \textit{Liv}. This follows straightforward from the eventual existance of an honest Fallback replica (at most $f+1$ elections) that reconciles all honest replicas. For any transaction, an interested honest client will eventually receive a shard-decision, and hence the transaction will complete.

We point out that steps 1, 2a, 3 and 4a simply correspond to the normal validation protocol. In order to avoid (multiple) \textit{interested} clients interfering too soon and causing unecessary divergence, we grant the original client an initial window of immunity.\footnote{Note, that it is irrelevant which client issues the $Phase1$ message. Therefore, we enforce the timeout window only for explicit logging ($Phase2$).} The orignial client is an interested client by default - an honest client that loses autonomy will, if necessary, elect a \textit{Fallback} itself. 

When depending on slow or byzantine transactions, clients may have to incur addtional latency in order to maintain liveness. On the flipside, when a clients fate is independent from other transactions, it is unaffected by concurrent slowdowns. 

%-------------------------------------------------------------------------------
\subsection{Optimizations}
%-------------------------------------------------------------------------------


\fs{the current writeback might be unecessary to explain, since we implement the below anyways... Perhaps merge writeback and this subsection}
\paragraph{Single Shard Logging}
\fs{This does not work for Atomic Broadcast! In AB, the voting only happens AFTER the tx has already been logged (i.e. the order has been durably replicated)}

When transaction execution touches multiple shards validation can incur redundant explicit logging overhead. When a Slow-Path is necessary to arrive at a logged decision on S different shards, bandwith is wasted. Consider an example in which $S-1$ shards attempt to log the decision Commit, while a single shard attempts to log an Abort decision. If the latter shard succeeds, the effort of the remaining shards was in vain. \fs{Moreover, logging is always bottlenecked by the slowest shard. }
The culprit of this phenomenon is the delayal of Two-Phase-Commit (2PC) until the Writeback phase. By preemptively making a 2PC decision \textbf{before} logging we can avoid this redundancy. We remark, that even when when all shards agree on a decision, this saves redundant coordination.
Concretely, we designate \textbf{one} involved shard as \textit{logging Shard}, while all other shards remain responsible only for Validation. The logging Shard can be determined via a determinsitic function over the \textit{involved Shards}. A simple load balanced solution may select $loggingS = min(TxID \% involvedShard)$. \fs{This does not work for Atomic Broadcast! In AB, the voting only happens AFTER the tx has already been logged (i.e. the order has been durably replicated).Figure \ref{fig:SingleShardOpt} shows a comparison and the revised structure. dont use this figure in actual paper} In order to log a decision, voting quorums from all involved shards are required. We modify step 3 of the Validation protocol accordingly:

\fbox{\begin{minipage}{21em}
\textbf{Validation (3: C)}: Client waits for vote replies from all involved Shards.
\end{minipage}}
A client aggregates a per-shard decision for each shard according to the \textit{CommitQuorum} rule. If all shard-decisions are Commit, it attempts to log a Commit decision by sending $Phase2 \coloneqq (TxID, Commit, S \times \{CommitQuorum\}$ to all replicas in the designated logging Shard. If a single shard-decision is Abort, it stops waiting for other shard-decisions and attempts to log an Abort decision by instead sending $Phase2 \coloneqq (TxID, Abort, AbortQuorum)$. 

\underline{Additional subtlelties:} A client can go Fast-Path and return to the Writeback phase immediately only if Fast-Path Quorums were received for all shards. 

The remaining Validation protocol proceeds identically to the multi-shard version. Notice, that when only a single shard is involved, no adjustments were made. The Writeback phase instead, may proceed with just the single certificate from the logging Shard


The Fallback protocol is adjusted accordingly: A fallback replica need (and can) only be elected on the logging Shard, simplifying reconciliation and reducing the cost for interested clients. 
To further reduce unecessary load, a client may attempt to first inquire whether decisions exists at the logging shard (Fallback protocol step 1 \& 2), before sending $Rec-Phase1$ messages to all shards in order to gather votes itself (Fallback protocol step 1). 

\iffalse
\begin{figure*}
\begin{center}
\includegraphics[width= \textwidth]{./figures/SingleShard.png}
\end{center}
\caption{Single Shard Optimization}
\label{fig:SingleShardOpt}
\end{figure*}
\fi

%%%-------------------------------------------------------------------------------
\subsection{Execution}
%-------------------------------------------------------------------------------
The goals of \sys{}'s execution protocol are threefold: 1) to offer clients an interactive transaction interface, 2) to be scalable, and 3) to offer honest clients independent operability.

\sys{} achieves all three of these goals though the combination of optimistic client-side execution, and an aggressive, but byzantine resilient, concurrency control (CC) scheme. Rather than submitting stored procedures that must be ordered and executed by replica servers, \sys clients may execute arbitrary transactions by relying on RPC's to conduct transactions themselves. This places computational responsibility with the clients, and allows commutative transactions to execute in parallel. Since \sys clients execute optimistically concurrently, \sys must enforce CC in order to reconcile Isolation conflicts might arise.
By relying on optimistic rather than pessimistic CC schemes such as 2-phase-locking (2PL), \sys{} forgoes costly coordination to acquire locks, and sidesteps the concern of byzantine clients refusing to relinquish locks. 

Before we detail \sys{}'s transaction processing and validation mechanism, we first discuss the CC that \sys implements as the latter are functions of the precise CC requirements. 
 
\subsubsection{Concurrency Control}

%%% Explain MVTSO itself only briefly. Then explain how we make it for for byz
% 1) bound timestamps, 2) make writes visible late only, 3) only allow f+1 matchin uncommitted writes, 4) read timestamps
%Afterwards: explain execution interface. After that explain validation check. Then have overarching example
\iffalse
% we need to change a few things:
0) timestmap bound
1) Read validity
2) deferred writes
3) read uncommitted
4) byz read timestamps (read lock implies you had access control, at that point it is no different than issuing a tx and completing. But we do not want to give the power to abort for no reason. It should be traceable.
\fi

Figure \ref{fig:MVTSOEX} shows an example trace of \sys operational behaviors under different request processing orders (assume f=1) that we use to guide the remaining outline.

\begin{figure*}
\begin{center}
\includegraphics[width= \textwidth]{./figures/MVTSOLargeFont.png}
\end{center}
\caption{\emph{MVTSO behavior for different replica processing orders}. $r_x(C : a_y ,P : a_z)$ denotes that transaction $T_x$ (x being the timestamp of a transaction in this example) reads the version $y$ of object a written by committed transaction $T_y$ and version $z$ from tentative prepared transaction $T_z$. $P_x(RS,WS,DEP)$ denotes a transaction $T_x$'s prepare request (i.e. the validation check) for respective ReadSet (RS), WriteSet (WS) and Dependenc Set (DEP), and $\rightarrow C / A$ denotes the local replica validation outcome (Commit/Abort).} 
\label{fig:MVTSOEX}
\end{figure*}

Our starting point is a Multiversioned Timestamp Ordering scheme (MVTSO) which prescribes a serialization order by assigning a speculative timestamp \textit{before} Execution \cite{bernstein1983multiversion, reed1983implementing, su2017tebaldi}. This allows us to a) only classify concurent transactions as conflicting if their execution results violate the timestamp order (Fig. \ref{fig:MVTSOEX}: 4 bottom), and b) avoid write/write conflicts alltogether by applying them in order. 
When speculative execution results match the pre-defined timestamp order, no aborts are necessary (Fig. \ref{fig:MVTSOEX}: 4 top). \sys challenge is therefore to a) assign appropriate timestamps and b) coordinate execution in a way that maximizes such coherence; the presence of byzantine participants (clients/replicas) complicates this.


In MVTSO, reads operations return the latest written version smaller than the readers' timestamps, while writers attempt to create new versions at their timestamp, but must abort if a higher timestamped read would have "missed the write" by reading a prior version. A transaction may commit only, if all read-write dependencies have committed respectively.

Read operations in \sys have the following sub-goals: \one Honest clients should read valid data, i.e. experience read integrity, \two Honest clients should read fresh data, i.e. minimize staleness and hence maximize commit chance, and \three Reads must provide the context necessary to potentially complete observed write state. 
Clients validate the integrity of reads by requiring replicas to provide a proof of validity, i.e. a set of signatures confirming the committment, or if not yet existant, trusting  only $f+1$ matching replies from discrete replicas. Moreover, in order to guarantee that clients do not read maliciously stale data from their local replica, \sys encourages clients to always read from at least $f+1$ replicas and choose the freshest version (Fig. \ref{fig:MVTSOEX}: 1,2).
Avoiding byzantine influence comes at a cost: Read operations require a synchronous, potentially WAN, rountrip. 

MVTSO intuitively synergizes well with \sys{}'s potenially WAN remote reads as reading from a fixed timestamp helps speculative readers observe consistent snapshots, even when execution is long and consequently interleavings are frequent (Fig. \ref{fig:MVTSOEX}: 6). \fs{at the price of experiencing serializability instead of strict serializability - that is only the case if your timestamps are outdated}.

While we assume that clocks are loosely synchronized across honest participants, byzantine participants may diverge arbitrarily in undesirable fashion. To side-step the additional overhead that a dedicated, bounded timestamping phase \cite{Clairvoyant} incurs, we compromise by allowing clients to optimistically select their own timestamp, but rejecting request timestamps above a Threshold at all honest replicas, thus incentivising clients to select real-time timestamps. In order to facilitate a total serialization order across all clients, we define Timestamps to be tuples of the form $(Time, CID)$. 

In traditional MVTSO, writes become visible to successive reads with higher timestamps immediately. In the presence of byzantine clients this however, is undesirable, as it allows clients to issue singular write operations without the intention of ever committing a transaction. Thus, any honest clients' read that observes, and consequently depends on, a byzantine clients write may be blocked indefinitely, thus compromising its liveness.
Allowing other clients to preemtively commit outstanding write operations is infeasible, as the remaining transaction procedure is known only to the issuing client, while conceiding pessimistic abort permission empowers byzantine participants to obstruct any honest clients' writes. 
To reconcile this dilemma, \sys{} defers all database updates until execution has finalized. Concretely, \sys clients buffer all write operations, and submit them only when attempting to commit the transaction, thereby enabling other \sys client to orderly complete Validation and Writeback steps. 

Clients in \sys distinguish explicitly between \textit{committed} and tentatively, but unconfirmed \textit{prepared} writes: While clients trust and accept every verified \textit{committed} write versions, they accept tentative \textit{prepared} versions only when observing $f+1$ matching results. This ensures that a) at least one honest replica believes that the write can commit, and hence is worth observing, and b) that byzantine replicas cannot violate byzantine independence by reactively inventing transactions that are guaranteed to abort.

Finally, \sys replicas abort all writes that higher timestamped reads "miss" by storing a Read Timestamps (RTS) for each locally processed read (Fig. \ref{fig:MVTSOEX}: 3). While this allows in-execution reads to elicit external effects, much like outstanding write operations discussed above, these are only limited to concurrent transactions with smaller timestamps and hence are straightforward to bypass by re-trying a transaction. To nonetheless limit repeatable abuse , we discuss a practical mechanism to limit byzantine influence in section Y.z (Optional Modifc. personalized read leases).



\subsubsection{Execution interface}
Client applications execute transactions via the following interface. A TX object \textit{TXObj $\coloneqq$ (SeqNo, ClientID, InvolvedShards, ReadSet, WriteSet, Dependencies)}, records the state necessary for Validation.

\iffalse
\begin{figure}
\begin{center}
\includegraphics[width= 0.5\textwidth]{./figures/TxState.png}
\end{center}
\caption{Transaction execution state}
\label{fig:Txstate}
\end{figure}
\fi

\textbf{Begin()} A client begins a transaction by optimistically choosing a timestamp \textit{TS $\coloneqq$ (Time, ClientID)}. \\
\textbf{Write(key, value)}. A client buffers the write: \textit{WriteSet = WriteSet $\cup$ (key, value)}\\
%%%%%%%%%Read protocol%%%%%%%%%%%
\textbf{Read(key, TS, RQS)} 
If \textit{key $\in$ WriteSet} a client returns the buffered write value. Otherwise, the client conducts a remote read:

\fbox{\begin{minipage}{22em}

\textbf{1: C} $\rightarrow$ \textbf{R}: Client sends read request to Replicas
\end{minipage}}\\
Given a hyperparameter Read Quorum Size (RQS), a client sends $m = (Read, key, TS)_c$  to $RQS$ different replicas. Note, that in order to guarantee $\geq RQS$ replies a client might need to send up to $f$ additional requests to compensate for unresponsive/faulty participants ($max(|Replies|) \leq n-f$). 

\fbox{\begin{minipage}{22em}
\textbf{2: R} $\rightarrow$ \textbf{C}: Replica processes client read and replies
\end{minipage}}\\
A replica authenticates the client\footnote{Byzantine replica may ignore read access control. Solving this problem is beyond the scope of this work; we defer to existing solutions \cite{basu2019efficient}.}, and whether the timestamp is within a local highwater mark ($HW = localClock + \delta$). It returns a signed message \text{$\langle \textit{ReadReply, Committed, Prepared} \rangle _R$}, where $Committed \coloneqq (value, version, proof)$ represents the value-version pair with largest committed write version smaller than TS and a proof of commitment,  and $Prepared \coloneqq (value, version, TxID', deps)$ the respective largest uncommitted value-version pair, the associated transaction ID, and the latter transactions potentially uncommitted read-write dependencies. Moreover, a Replica stores a new read timestamp (RTS) for the key: $RTS(key) = RTS(key) \cup TS$ (Fig. \ref{fig:MVTSOEX}: 3). 
\fs{a replica may include a set of prepared values to increase likelihood of client receiving f+1 matching}

\fbox{\begin{minipage}{22em}
\textbf{3: C} ($\rightarrow$ \textbf{R}): Client receives read replies 
\end{minipage}}\\
A client waits for RQS read replies and chooses the biggest valid result \textit{(value, version) $= max_{valid}$(\{Committed\},\{Prepared\}} (Figure \ref{fig:MVTSOEX}: 1,2) . A \textit{Committed} tuple is valid, if the proof confirms commitment, wheras a \textit{Prepared} is valid iff there exist $f+1$ matching \textit{Prepared$_r$} signed by different replicas. The client adds the version to its read set \textit{ReadSet = ReadSet $\cup$ (key, version)} and additionally claims a dependency if it was a \textit{Prepared} version: \textit{Dep = Dep $\cup$ f+1 $\times$ (Prepared$_r$)} . 

\textbf{Commit()} A Client terminates its execution, and computes a unique transaction identifier based on final execution object and its timestamp: $TxID \coloneqq (H(TxObj, TS)$, thus preculding byzantine participants from equivocating transaction contents. It then begins the Validation Phase by issuing a 2PC-Prepare requests to each relevant shard and waiting for all votes.

\textbf{Abort()} A client terminates execution, and broadcasts a request to release all acquired Read Timestamps (RTS). Since writes in \sys are deferred, no other rollback action is necessary.


We briefly discuss some implications of the choice of Read Quorum Size (RQS). Following cases may be distinguished: \one \textbf{$RQS = 1$} A replica may read from just 1 replica at the risk of reading maliciously stale data. If a Client trusts a local replica \textbf{and} the replica is not lagging behind, this can reduce execution latency and consequently minimize conflicting interleavings. \two \textbf{$RQS \geq f+1$} \sys{}'s recommended mode of operation. While side-stepping maliciously stale reads, it is still possible to read (arbitrarily) stale data, either due to inconsistency caused by asynchrony or a byzantine client not fully replicating its transaction. 
\three \textbf{$RQS \geq \frac{n+f+1}{2}$} When reading from a Quorum of sufficient size to overlap with any Validation Quorum (ref section Validation) in at least one honest replica, a client guarantees, that no more \textbf{additional}, beyond previously observed, conflicting writes can be admitted, since such a Quorum of acquired RTS acts as a read-lock. 

\iffalse
Trusting only $f+1$ matching prepared writes has the beneficiary side effect of allowing us to include only transaction identifiers as dependencies, rather than the full transaction, since it is guaranteed that at least one honest replica has stored the transaction. Thus, in the failure free scenario, where clients need not complete transactions $in Dep$, \sys minimizes the meta-data overhead that the recovery protocol imposes.
\fi

We remark, that a byzantine client is not required to follow any of the protocol steps, with the exception of claiming dependencies. By our definition of Byzantine-Isolation, read integrity is only required for honest clients; Byzantine clients may \textit{choose} whether to read legal, or even real data. We do, however, require that no client can fabricate dependencies in order to preclude byzantine clients from indetectibly stalling their own transactions and consequently obstructing liveness for consecutive descendant (second degree...) dependencies.
Trusting only $f+1$ matching prepared writes allows us to only include only transaction identifiers as dependencies, rather than the full transaction, thereby minimizing the common path overhead. When the full transaction information is required to complete potentially stalled transactions, it is guaranteed to be obtainable from an honest replica. We discuss how to complete stalled or slow transactions in section Y (Granting Liveness).





%%-------------------------------------------------------------------------------
\subsection{Validation Phase}
%-------------------------------------------------------------------------------

In order to facilitate atomic committment across all shards involved in a transaction \sys clients invoke a Prepare request, inquiring each shard to validate the transaction for conflicts and cast a 2PC vote. Shards in \sys are replicated for fault tolerance.

\sys comes in two different flavors, \sys{}3 and \sys{}5 respectively, that rely on varying replication degrees, but implement the same design. \sys{}3 requires $n=3f+1$ replicas \fs{, the minimum bound necessary for BFT SMR?,} per shard to guarantee consistency in the presence of $\leq f$ byzantine replicas. \sys{}5 reduces both latencies during failure free execution and complexity during recovery. \footnote{We believe that consortiums with high performance requirements or high replication degrees are respectively comfortable with paying for additional replicas or tolerating a lower fraction (1/3 vs 1/5) of failures}
For the simplicity of exposition we discuss \sys{}5 for the remainder of the paper, and defer to section X \fs{and/or TR} to describe differences in \sys{}3. In the following, we outline \sys 's execution (unaffected by replication degree), validation and writeback protocols.

The goal of the validation phase is straightforward:
Shards need to cast a single durable vote, in the following we refer to this as the shard-decision. It is computed on the basis of different replica votes.
The vote needs to respect isolation guarantees
leaderless and partially ordered (i.e. coordination only when necessary)



The goals of the Validation phase are threefold: \one It must decide on a single, durable vote per-shard that maintains Byzantine Serializability (henceforth we refer to this as a \textit{shard-decision}), \two It should be leaderless, and not enforce unecessary ordering for commutative, non-conflicting transactions, and \three It must preserve independent operability. \fs{needs to refer to the fallback}

Satisfying these goals requires ovecoming several challenges. In order to maximize parallelism and embrace partial ordering, Indicus allows replicas to process requests out of order. Consequently, replicas may temporarily diverge, and hence return different validation results. Such divergence must be reconciled in a way that maintains Isolation, but not overly conservatively in order to maximize the ability to commit successfully and bound the impact of byzantine participants. Further, Indicus designates clients as validation coordinator for its own transactions, thus omitting a dedicated replica leader. The respective protocol must tolerate client failures such as crashes, omission/stalling, equivocation, or replays and allow for consistent recovery. 

The Validation protocol can be broken down into two functionalities: Voting and Logging.
Since \sys replicas may process requests in different order, they must reach consensus on a joint  decision. To do so, they cast a vote for their local validation result, which are subsequently democratically aggregated into a decision. In the case of Indicus, the client acts as the transaction coordinator who aggregates and relays results (along with necessary evidence).
In order to maintain consistency in the presence of failures (no two honest replicas finalize different Commit/Abort decisions), this decision must be durable and unique, guaranteeing that a replay of any Writeback is idempotent. However, since a byzantine coordinator cannot be trusted to durably store a decision, nor could we retain liveness during a crash or partition, \sys demands clients to \textit{log} the decision at replicas before returning.
 

The voting step requires a single round-trip to all replicas, whereas the logging phase requires at most one round-trip \fs{in Indicus5 and at most two round-trips in Indicus3.}. When execution is fault- and contention free transactions can be committed on the \textit{Fast-Path} in a single-round trip as an explicit logging round is not necessary.


%%%%%%%%%% protocol
To describe how the protocol operates in detail we follow a single-shard transaction through the system:

\fbox{\begin{minipage}{23em}
\textbf{(1: C $\rightarrow$ R)}: Client sends Prepare request to all Replicas within the Shard.
\end{minipage}}\\
Upon deciding to Commit in the Execution phase, a Client initiates Validation by sending a message $Phase1 \coloneqq \langle Prepare, TxID, TX \rangle_{\sigma_c}$ to all Replicas.

\fbox{\begin{minipage}{23em}
\textbf{(2: R $\rightarrow$ C)}: Replica receives validation request, processes it and returns vote to Client.
\end{minipage}}\\
A replica validates Timestamp and Dependency integrity of the request. It then evaluates Read and Write Sets for Isolation conflicts against its local state using the MVTSO Concurrency Control Check (CCC), as shown in algorithm 1. It returns a message $Phase1R \coloneqq \langle TxID, vote \rangle_r$, and optionally evidence in case it voted to Abort.

\underline{Additional subtlelties}: 
A replica never changes its Voting decision, because re-execution could leave to different results. Once the MVTSO-Ceck completes (i.e. there are no blocking dependencies), a replica starts a timer to monitor the clients progress.

\fbox{\begin{minipage}{23em}
\textbf{(3: C)}: Client waits for vote replies.
\end{minipage}}\\
A client waits for at least $n-f$ ($4f+1$ in Indicus5) distinct replica votes, or more, up to a system specified timeout. 

\fbox{\begin{minipage}{23em}
\textbf{(3a: C)}: Client receives Threshold of matching votes and returns to application. Proceeds to Writeback
\end{minipage}}\\
In any of the following 3 cases, a client may short-circuit waiting for additional votes and omit a dedicated Logging round:
\begin{enumerate}
\item \textbf{$1$ Abort vote w/ Conflicting TX \& CommitCertificate}: A conflict with a commited transaction. The client validates the integrity of the CommitCertificate and returns the shard decision $(TxID, Abort, \langle ConflTX \rangle_{CC})$. 
\item \textbf{$3f+1$ Abort votes w/ Conflicting TX}: A conflict with a prepared, but not yet committed transaction. The client returns the shard decision $(TxID, Abort, \{\langle AbstainVote\rangle_r\})$. 
\item \textbf{$5f+1$ Commit votes}: No conflicts. The client returns the shard decision $(TxID, Commit, \{\langle CommitVote \rangle_r\}$
\end{enumerate}
Any of such Quorums forms a \textit{Shard-Certificate} and proves the decision. A client uses this Certificate to return to its application and issue the Writeback.

\fbox{\begin{minipage}{23em}
\textbf{(3b: C $\rightarrow$ R)}: Client receives divergent results and suggests a consistent decision to Replicas for Logging
\end{minipage}}\\
If a client does not receive the necessary thresholds of votes to return, it must continue on the \textit{Slow-Path}. To do so, it aggregates the votes according to following decison rule:
If there exists a $CommitQuorum \coloneqq \frac{n+f+1}{2}$ of Commit Votes, the Slow-Path decision is Commit, otherwise it is Abort.
A client broadcasts a message $Phase2 \coloneqq (TxID, decision)_c, \{\langle votes \rangle_r\}$.

\underline{Additional subtlelties}: A client forwards a Quorum of $\geq n-f$ votes to the replicas in order to prove the Slow-Path decision is consistent with Isolation guarantees. Note, that a byzantine client may equivocate the decision by relaying different Quorums.

\fbox{\begin{minipage}{23em}
\textbf{(4: R $\rightarrow$ C)}: Replicas receive, validate and echo decision
\end{minipage}}\\
A replica confirms that the Decision matches the Quorum by evaluating the decision rule itself and adopting the decision. It then returns the decision to the client by sending $Phase2R \coloneqq \langle TxID, decision \rangle_r$. Importantly, a replica never changes its decision.

\fbox{\begin{minipage}{23em}
\textbf{(5: C)}: Client returns shard-decision to application and proceeds to Writeback
\end{minipage}}\\
A client waits for a Quorum of $n-f$ matching $Phase2R$ messages. Such a Quorum forms a \textit{Shard-Certificate} and proves the decision. A client uses this Certificate to return to its application and issue the Writeback.

\underline{Additional subtlelties}: If a client equivocated, it will never receive a Shard-Certificate. An honest client however, is guaranteed to receive matching Phase2 replies. 

We consider a decision (Commit, Abort) to be \textit{logged} when it is possible for some Shard-Certificate to exist, i.e. as soon as the necessary certificate Quorums exists at some Replicas.
Figure \ref{fig:FigureSP} summarizes the relevant nomenclature.

\begin{figure}
\begin{center}
\includegraphics[width= 0.5\textwidth]{./figures/Nom2.png}
\end{center}
\caption{Validation Nomenclature, Slow-Path. Note, that a byzantine client may equivocate Phase2 decisions by including Commit and Abort Quorums respectively. Byantine replicas may store multiple votes and decisions.}
\label{fig:FigureSP}
\end{figure}

\subsubsection{Correctness}
We show, that a \textit{logged} decision is final:
\begin{theorem}[Saf]
A logged decision is durable, and there can ever exist \textbf{at most one} logged decision.
\end{theorem}
\begin{proof}
See TR.
\end{proof} 

Note, that since replicas never change their decision, it is possible for there to never be any logged decision if a byzantine client equivocated its Slow-Path Quorums. In order to reconcile this, we design and discuss a recovery mechanism in section X which relaxes the requirement on persisting a decision.  


\begin{theorem} 
Indicus maintains \textit{Byzantine-Serializability}.
\end{theorem}
To prove that this is the case, we show that for any two conflicting transactions, at most one can be committed.
\begin{proof}
See TR.
\end{proof}

\begin{theorem} 
Indicus maintains Byzantine Independence in the absence of network adversary.
\end{theorem}

We show, that once a Client submits a transaction for validation, the result cannot be unilaterally decided by any byzantine participant, be it client or replica.
\begin{proof}
See TR.
\end{proof}
%
\subsection{Validation Check}

Algorithm \ref{mvtso} shows the necessary validation check to preserve Byzantine-Serializability. 
Given a transactions prepare request the validation check returns an Abort vote if a conflict has been detected, and Commit otherwise. 
When execution results match the timestamp order, there are no conflicts (Fig. \ref{fig:MVTSOEX}: 4 top). For each read, a replica verifies that it has not voted to commit a conflicting write (Algorithm \ref{mvtso}, line 3-7). Conversely, for each write, a replica confirms that there exist no previously accepted reads (line 8-10), and no ongoing read transactions (line 11-12) that conflict. Fig. \ref{fig:MVTSOEX}: 4 shows both non-conflicting and conflicting interleavings.
If there are no conflicts, a replica tentatively \textit{prepares} a transaction, making its writes visible and evaluating future transactions against it for conflicts. Regardless of the outcome, a replica garbage collects all Read Timestmaps (RTS) associated with the transactions reads.
It then waits for necessary dependencies (uncommited writes that were read) to be resolved (Figure \ref{fig:MVTSOEX}: 5). We remark, that the concurrency control check is serialized and executed atomically for each transaction.

\begin{theorem}
The set of transactions for which the MVTSO-Check returns Commit is Byzantine-Serializable. 
\end{theorem}
\begin{proof}
See TR.
\end{proof}

\begin{algorithm}
\caption{MVTSO-Check(TX, TS)}\label{mvtso}
\begin{algorithmic}[1]
\If{\textit{$TS > localClock + \delta$}} %  || $TS < lowWM$ || $\exists d \in dep: d.TS < lowWM$} } dont mention garbage collection part here, it only confuses
\State \Return Abort
\EndIf

\For{\textit{$\forall key,version \in \textit{TX.RS}$}}
        \If{$ \exists TX2 \in Committed \cup Prepared: key \in \textit{TX2.WS} $ \newline
        \hspace*{2em} $\land \, version < \textit{TX2.TS} < TS$}  
          \State  \Return Abort, \textit{TX2, (TX2.CommitProof)}  
         \EndIf  
\EndFor

\For{\textit{$\forall key \in \textit{TX.WS}$}}
        \If{$\exists TX2 \in Committed \cup Prepared:$ \newline 
        \hspace*{2em} $\textit{TX2.RS[key].version} < TS < TX2.TS$} 
          \State  \Return Abort, \textit{TX2, (TX2.CommitProof)}
         
        \EndIf
        \If{$\exists RTS \in key.RTS: RTS > TS$} 
          \State  \Return Abort
       \EndIf
\EndFor
\State Prepared.add(TX) 

\While{$\exists d \in dep: d \notin CommitLog \cup AbortLog $)}
\State Suspend
\EndWhile

%structure it in a way that is better
\For{\textit{$\forall d \in dep$}}
		\If{$ d \in AbortLog $}
		\State	Prepare.remove(TX)
		\State \Return Abort, \textit{(TX2.AbortProof)}
		\EndIf
\EndFor
\State \Return Commit
\end{algorithmic}

\end{algorithm}

In order to perform the MVTSO-check, a replica maintains several data strucutres: \one It stores read timestamps, read versions alongside the multiversioned write stores for committed and tenative transactions respectively, to provide efficient evaluation of conflicts.
\two In order to confirm dependency outcomes, replicas log proofs for completed transactions in respective Commit and Abort Log sets, that together induce the ledger of all processed transactions. 
\three To avoid busy waiting when dependencies are not yet resolved, a replica temporarily suspends the MVTSO-check for the current transaction, allowing it to process other transactions pending validation. To facilitate this, it keeps track of an additonal transaction to dependents mapping that allows to identify and resume all suspended MVTSO-checks associated with dependents of a completing transaction.

\iffalse

\textit{Aside:} Consistent with our definition of Byzantine-Isolation, byzantine clients may issue ficticious Read-Sets comprised of arbitrary read versions and values. However, these have only limited external effect on concurrent writes. We distinguish two extreme cases: 1) $read.version \rightarrow 0$: This case is equivalent to simply reading stale data, and effectively reduces MVTSO to TSO as conlflicts are evaluated only on basis of the transaction timestamps (i.e. Abort write if: write.TS < read.TS). 2) $read.version \rightarrow read.TS$: In this case there are no conflicts as a write is never "missed" by a previous read.
\fi

In the following section we will show how to design a replicated validation scheme that upholds Isolation guarantees and reaches a single shard decision, even when replicas within a shard validate in different orders.

%%-------------------------------------------------------------------------------
\subsection{Consistent logging.}
%-------------------------------------------------------------------------------
Principles and challenges

protocol overview: pic

%%-------------------------------------------------------------------------------
\subsection{Writeback}
%-------------------------------------------------------------------------------

Validation occurs on every Shard that a transaction spans. The goal of the Writeback phase is to aggregate all relevant shard-decisions and to inform replicas of finalized Commit or Abort decisions. This is necessary in order for replicas to be able to garbage collect meta-data of ongoing transactions, and to allow consecutive transactions to reliably observe the updated state. Any finalized decision must respect all shard-decisions relevant to a transaction. Thus, all decisions are aggregated according to standard two-phase commit. \sys follows a simple 2PC protocol: Only if all shards agree that a transaction may commit (i.e. there exist commit certificates for every shard), then a transaction may commit. A single abort shart-decision suffices to abort a transaction respectively.

\textbf{1.} A coordinator waits for all necessary shard decisions, including certificates\\
\textbf{2.} The coordinator broadcasts a $Writeback \coloneqq (TX, decision, \{certificates_S \} )$ message to all replicas in all relevant shards.\\
\textbf{3.} A replica validates whether the certificates match the decision, and commits/aborts the transaction by applying the relevant store updates, resuming potentially suspended MVTSO-Checks, and garbage collecting all ongoing transaction state. Replicas store the Writeback message which is used as Commit proof to service reads and justify conflicts, or Abort proof to justify dependency aborts.

We point out, that the Writeback coordinator need not be the client issuing the transaction, but can in fact be an arbitrary party (client or replica) that is interested in completing the Writeback. This follows straightforwardly from Theorem Y: Any certified shard-decision implies the existence of a logged decision, and hence the Writeback phase is idempotent.
We utilize this to drive the recovery protocol outlined next. 


%\subsection{Multi-sharding}

and Multi-shard 2pc

- Optimization: Single shard logging

%%-------------------------------------------------------------------------------
\subsection{Failures}
%-------------------------------------------------------------------------------
- Fallback: election (only starts if not waiting on another dep to avoid early eviction), views, resolution, subtelties with mvtso (block because of dep), necessity even without dependencies. Interested clients, write-back multishard. garbage collection
- Fallback requires an extra round in order to learn about current views to start viewchange, but thats ok: Its co-function with learning about full TX, and checking for existing certificates. Timeout invocation is concurrent with p1 message.

%%-------------------------------------------------------------------------------
\subsection{Low Cost mode}
%-------------------------------------------------------------------------------
3f+1 if not defending against byz colluders
OCC if not worried about reads aborting


%-------------------------------------------------------------------------------
\section{Protocol}
%-------------------------------------------------------------------------------
Indicus comes in two different flavors, Indicus3 and Indicus5 respectively, relying on varying replication degrees. Indicus3 requires 3f+1 replicas per shard to guarantee consistency, the minimum bound necessary for BFT SMR. Indicus5 uses a higher replication degree of 5f+1 replicas, but in return brings down both gracious execution latency and complexity during uncivil executions. We argue, that unlike past system settings where a single authority strives to maintain the fewest amount of replicas necessary for cost considerations, consortium systems with naturally higher replication degree are willing to pay the additional price for performance. Moreover, rather than trying to reach the threshold of replicas to tolerate some number of faults, these settings may start with a fixed number of replicas and consider the ratio of faults to be tolerable. When argueing from this perspective, the perceived difference between tolerating 1/3 and 1/5 of replica failures may be negligible.
For the simplicity of exposition we outline Indicus5 in detail below. In section X we describe the differences in Indicus3.


%%%%%%%%%%%%%%%%%%%%%%%%%%%%%Figs
\begin{figure}[t]
  \begin{mdframed}[roundcorner=10pt]
 	\textbf{TX}
 	\begin{itemize}
 	\item ClientID
 	\item ClientSeqNo
 	\item ReadSet = \{(key, version)\}
 	\item WriteSet = \{(key, value)\}
 	\item dependencies = \{$\langle (key, version, \{TxID'\})_{f+1 \sigma} \rangle$\}
 	\item TxID = H(TX)
 	\item Timestamp = (Time, ClientID)  optional:, TxID) this is just deterministic tie breaker when a client misbehaves.
 	\end{itemize}
  \end{mdframed}
  \caption{Transaction}
  \label{fig:TX}
\end{figure}

\newcommand{\SubItem}[1]{
    {\setlength\itemindent{15pt} \item[-] #1}
}

\begin{figure}[t]
  \begin{mdframed}[roundcorner=10pt]
 	\textbf{Replica }
 	\begin{itemize}
 	\item ReplicaID
 	\item ShardNo
 	\item LocalClock
 	\item RTS = \{(key, \{(TS, \{dep\})\})\}
 	\item Prepared:
 	\subitem PreparedTX = \{(TxID, [TX, state])\}
 	\subitem PreparedDB = \{(key, [writes:\{(val, version, TxID)\},  reads: \{TS, rversion, TxID\}])\}
 	\subitem WaitingDeps = \{(TxID, dependents\}
 	\subitem Deps = \{(TxID, dependencies\}
 	\item Database = \{(key, [w: \{(val, ver)\}, r: \{(TS, rv)\}])\}
 	\subitem CommitLog = \{(TxID, CommitCert)\}
 	\subitem AbortLog = \{(TxID, AbortCert)\}
	 	
 	
 	\end{itemize}
  \end{mdframed}
  \caption{Replica State and Datastructures}
  \label{fig:RS}
\end{figure}

NEED FIg with replica state. Need to use it to explain what "Prepared"/Committable means
%-------------------------------------------------------------------------------
\subsection{Execution}
%-------------------------------------------------------------------------------
Clients in Indicus both execute and submit their own Transactions. As previously defined a Transaction TX is a sequence of read and write requests that is ultimately terminated by a Commit or Abort decision. A TX object, as shown in Figure \ref{fig:TX} records the execution state necessary for Validation. The execution protocol has three goals: 1) Honest Clients should read valid data, i.e. experience read integrity, 2) Honest Clients should read fresh data, i.e. minimize staleness and hence maximize commit chance, and 3) avoid expensive coordination as much as possible. This is complicated by the presence of byzantine replicas as well as concurrent transactions. Byzantine replicas may provide invalid or arbitrarily stale data and honest replicas may be temporarily out of sync.

We avoid invalid reads by requiring replicas to provide a proof of validity, i.e. a Quorum of Writeback signatures confirming the committment, or alternatively, trusting only $f+1$ matching replies from discrete replicas. In order to minimize coordination only read requests incur a network rountrip, while writes are buffered locally until Validation. 
Since together Validation and Writeback together incur several Wide Area Network (WAN) message delays, a committing transaction is invisible to concurrent transactions for the time-being, yet results in isolation conflicts that need be resolved. In order to minimize this window, we allow transactions to speculatively read proposed, yet uncommitted values. Similarly, since Execution can span multiple read requests that require WAN message delays, there remains a potentially large period in which reads are vulnerable to conflict. To mitigate aborts due to write conflicts, we a) let reads be sequenced at a pre-defined timestamp (rather than always reading the most recent) and b) let reads acquire implicit read-locks that disallow conflicting writes.



Client execution conducts as follows:\\

\begin{enumerate}
\item \textbf{Write(key, value)}. A Client executes a request Write(key, value) by locally buffering (key, value) and returning. Concretely: $WriteSet = WriteSet \cup (key, value)$

%%%%%%%%%Read protocol%%%%%%%%%%%

\item \textbf{Read(key, TS, RQS)} 

\fbox{\begin{minipage}{20em}
\textbf{1: C} $\rightarrow$ \textbf{R}: Client sends read request to Replicas
\end{minipage}}

Given hyperparameter Read Quorum Size (RQS), a Client performs a read on given key at timestamp TS by reading from RQS different replicas. To do so, a Client sends $\langle(CID, key, TS)\rangle_{\sigma_c}$ \fs{technically CID is part of TS} to the respective replicas.\\

\fbox{\begin{minipage}{20em}
\textbf{2: R} $\rightarrow$ \textbf{C}: Replica processes Client read and returns reply
\end{minipage}}

A replica returns a signed pair \text{$\langle \textit{(Committed, Prepared)} \rangle _{\sigma_r}$}, where Committed is the write with largest committed write version prior to the specified Timestamp and Prepared is the respective largest uncommitted write that might be committed and would make Committed outdated. 

\underline{Additional subtlelties}: If the timestamp is beyond a Highwater mark ($HW = localClock + \delta$) replicas ignore the requests. If the request is serviced, a Replica adds the timestamp to its Read Timestamp Set (RTSS): $RTSS(key) += (TS)$. \text{$Committed \coloneqq (value, version)_{cert}$} such that $ (value, version) \in R.CommitLog$, $version = max(q) : (key, val, q) \in R.CommitLog \land v < TS \}$ and $cert$ is a certificate proving (value, version) was legally committed (see Section Writeback) and 
 $Prepared \coloneqq (value, version, dep)$ such that $(value, version) \in R.PreparedSet$, $version = max(q) : (key, val, q) \in R.PreparedSet \land Committed < v < TS \}$ and $dep$ is a DAG of Transaction IDs, starting from the TX that wrote $(value, version)$ and extending to its own dependencies.



\fbox{\begin{minipage}{20em}
\textbf{3: C} ($\rightarrow$ \textbf{R}): Client receives read replies and asynchronously dissipates dependencies
\end{minipage}}

A client waits for RQS read replies and validates the integrity of any $Committed$ it receives.  It adds the biggest $Committed$ or $Prepared$ (iff enough matching) version seen to its Read Set, and claims a dependency if it was a $Prepared$ version. If it claims a dependency, it asynchronously forwards this dependency to all replicas. 



\underline{Additional subtlelties}: If $RQS > n-f$ a Client waits only up to a application set timeout beyond the first $n-f$ received replies. If $f+1$ matching $Committed$ are received they need to be validated. If $f+1$ matching $Pepared$ are received from disjoint replicas, and
 $Prepared.version > max(\{Committed.version\}, \{\langle Prepared.versions \rangle_{f+1 \sigma_r}\})$ the client adds Prepared to its Read Set and claims a dependency by including the $f+1$ signatures: $ReadSet += (key, Prepared.version)$ and $dependencies += \langle Prepared.dep \rangle_{f+1 \sigma_r}$. 

\fs{does the value need to be included?} Further, a Client sends $\langle (key, TS, dep)\rangle_{\sigma_c}$ to all replicas. If there exists no such Prepared, it adds the largest valid Committed, i.e. $ReadSet += (key, max(Committed.version)$. 



\fbox{\begin{minipage}{20em}
\textbf{(4: R)} : Replica receives dependency and adds exception
\end{minipage}}
Upon reception of $\langle (key, TS, dep)\rangle_{\sigma_c}$ a replica adds an exception to its Read Timestamp Set. Concretely, $RTSS(key)(TS) += dep$.



Implications: 
1. Reading from 1 Replica: could abort because reading arbitrarily stale.
Alternatively: Read from f+1 always to reduce proof/memory costs: Could not get matching and hence fail read. Still possible to abort by reading stale data 
2. Reading f+1 matching Prepared that are larger than the commit --> choose that and add dependency (in 5f+1 might only want to do it if read 2f+1, not necessary for safety, but higher commit chance?)
Guaranteed to see 1 thats not intentionally stale. But still arbitrarily stale if unlucky.
3. If read from 3f+1/2f+1 then its effectively a read lock. 
If 1 prepared read was enough then this would be sufficient to see the newest "potential" value. However that is not possible for liveness reasons.
Note, could still have missed a prepared write and have to abort because we required f+1 matching. Why f+1? so it comes from 1 honest: we dont need proofs, and we know it is in the system for recovery.
Still no guarantee that newest commit was seen, but some protection from other writes.



(Note: the TS used here is not the final one, its just the tuple of time and clientId. (The triple later is just necessary to differentiate two TX by a client that were assigned the same Time
Add Read set decision rule, put in correlation to RQS: \fs{could only accept reads if f+1 matching always, then no proofs necessary, but then reads can fail}
Add exceptions etc.
)


\item \textbf{Commit} A Client finalizes its execution, computes the final Timestamp and submits for validation.

\item \textbf{Abort} A client terminates execution, broadcasts a read-release for all potentially acquired RTS and returns.
\fs{A byz client may not release the RTS. To circumvent this, RTS are just leases. Moreover, at some point writes will overtake them}

\end{enumerate}
 


- Read proofs required for honest client correctness. Alternatively one can read from f+1 only, but that can result in failed/older reads
- Do not need to include read proofs in the prepare. According to Isolation definition byzantine Clients can read whatever they want. Moreover, Reads only have very limited external effect. The value does not matter for the CC check. The version has bounded effect: If it goes towards 0, then it is just a check between timestamps as normally. If it goes towards the TS, then it will never abort.
- Fallback not just useful for dependency cleanup, but also "unclaimable" dependencies that need to finish.
%-------------------------------------------------------------------------------
\subsection{Validation}
%-------------------------------------------------------------------------------

%-------------------------------------------------------------------------------
\subsection{Concurrency Control}
%-------------------------------------------------------------------------------
\begin{algorithm}
\caption{MVTSO-Check(TX, TS)}\label{euclid}
\begin{algorithmic}[1]
\If{\textit{$TS > localClock + \delta$} }
\State pass
\EndIf

\For{\textit{$\forall key,version \in \textit{TX.read-set}$}}
        \If{$ \exists TX2 \in CommitLog: key \in \textit{TX2.write-set} \land version < \textit{TX2.TS} < TS$}  
          \State  \Return ABORT, \textit{TX2, TX2.CommitCert}
        \EndIf
         \If{$ \exists TX2 \in Prepared: key \in \textit{TX2.write-set} \land version < \textit{TX2.TS} < TS$}   
          \State  \Return ABSTAIN, \textit{TX2}
        \EndIf
        
   %     \If{$ \textit{dep[key]} == null \land version \notin prepared-reads[key] \cup  CommitLog \cup AbortLog $}   
    %      \State  \Return $\textit{Proof-of-Misbehavior}$
    %    \EndIf
        
        
\EndFor

\For{\textit{$\forall key \in \textit{TX.write-set}$}}
        \If{$\exists TX2 \in Prepared/CommitLog \land TX \notin TX2.dep: \textit{TX2.read-set[key].version} < TS < TX2.TS$}
          \State  \Return RETRY, %$\textit{TS_{new} | \forall TS_{new} > TX2.TS}$ // 
        \EndIf
        \If{$\exists RTS \in key.RTS: RTS > TS \land TX \notin RTS.dep$}
          \State  \Return RETRY, %$\textit{TS_{new} | \forall TS_{new} > TX2.TS}$ // 
        \EndIf

\EndFor
\State Prepared.add(TX) 
\While{$\exists d \in dep: d \notin CommitLog \cup AbortLog $)}
\State Wait
\EndWhile

\For{\textit{$\forall d \in dep$}}
        \If{$ d \in CommitLog $}
        	\If{$d.TS > TS$}
        	\State \Return ABORT, d.CommitCert
          	\EndIf
       
		\Else 
		\State \Return ABORT, d.AbortCert
		\EndIf
\EndFor


\State \Return COMMIT


\end{algorithmic}
\end{algorithm}


 - Optimization: retries - heights
 - Dependency resolution tree
 		- Equivocation not possible if TXidentifier a function with dep as argument
 		- Cannot claim dep if not f+1 times (If you want to, you would require proofs again, which we try to avoid because dep trees can grow exponentially). More reads also more likely to commit --> f+1 guarantees 1 honest thinks it is legit.
 - Exception for depedency and early async read response to inform of exceptions
 - Read leases instead of unlimited locks - in practice only grant to timely clients (not a safety measure, but an increased progress guarantee)
 - 
%-------------------------------------------------------------------------------
\subsection{Consistent logging.}
%-------------------------------------------------------------------------------
\begin{figure}
\begin{center}
\includegraphics[width= 0.4\textwidth]{./figures/AB.png}
\end{center}
\caption{Atomic Broadcast}
\label{fig:Figure1}
\end{figure}


\begin{figure}
\begin{center}
\includegraphics[width= 0.4\textwidth]{./figures/5f+1.png}
\end{center}
\caption{Logging}
\label{fig:Figure1}
\end{figure}

Principles and challenges

protocol overview: pic


%-------------------------------------------------------------------------------
\subsection{Writeback and Multi-shard 2pc}
%-------------------------------------------------------------------------------

%-------------------------------------------------------------------------------
\subsection{Failures}
%-------------------------------------------------------------------------------
- Fallback: election (only starts if not waiting on another dep to avoid early eviction), views, resolution, subtelties with mvtso (block because of dep), necessity even without dependencies. Interested clients, write-back multishard. garbage collection
- Fallback requires an extra round in order to learn about current views to start viewchange, but thats ok: Its co-function with learning about full TX, and checking for existing certificates. Timeout invocation is concurrent with p1 message.

%-------------------------------------------------------------------------------
\subsection{Optimizations}
%-------------------------------------------------------------------------------
- Retries
- single shard logging
\begin{figure*}
\begin{center}
\includegraphics[width= \textwidth]{./figures/SingleShard.png}
\end{center}
\caption{Single Shard Optimization}
\label{fig:Figure1}
\end{figure*}


- (Read Locks ; remove can be optimization for writes)
- OCC instead of mvtso structures if disallowing prepared writes to be visible. OCC if not worried about reads aborting

%-------------------------------------------------------------------------------
\subsection{Garbage Collection}
%-------------------------------------------------------------------------------
- Prepares get removed upon commit/abort. Eventually replica might need to become "interested" client.

%-------------------------------------------------------------------------------
\subsection{Indicus3}
%-------------------------------------------------------------------------------
3f+1 if not defending against byz colluders as much

- no fast path
- Commits in 2 rounds, Aborts in 3 (Alternatively symmetric version)
- fallback quorums and bounds
- proofs necessary for recovery
- recovery rules

\begin{figure}
\begin{center}
\includegraphics[width= 0.4\textwidth]{./figures/3f+1.png}
\end{center}
\caption{Logging}
\label{fig:Figure1}
\end{figure}



%%-------------------------------------------------------------------------------
\section{Discussion}
%-------------------------------------------------------------------------------
1. Explain or prove how we satisfy byzantine independence and why we want it:

byz participants cannot reactively make aborts happen if they do not control the network. 
This is stronger than a leader based setting.  A byzantine leader can always reactively create a conflicting TX. An honest leader would not do this. 
Thus we are equivalent to an honest leader.

(If you know the keys beforehand it is possible to create mutually aborting transactions to form dependencies. But this requires additional knowledge; arguably dependent on the network too).


2. Can use witnesses instead of full replicas to reduce replication cost. Those need to only contain key/version and no values.






\input{sections/Evaluation.tex}

%-------------------------------------------------------------------------------
\section{Limitations}
%-------------------------------------------------------------------------------
As is inherent to any Optimistic Execution and Concurrency Control, Indicus is vulnerable to highly congested workloads. When contention on select objects is high, concurrent execution of Transactions must yield the abort of some Transactions during Validation in order to maintain the Database Isolation guarantees. Note however, that when clients are in charge of execution, a pessimistic concurrency control solution such as two-phase-locking would incur an equal amount of deadlocks which would require resolution. The observation to make is that any system that conducts execution at the client application side speculates on concurrency. This however we stipulate, is unavoidable when trying to scale a system to the number of users rather than replica processing power. The traditonal ways to avoid the abort rate conundrum is to either restrict the transaction model, which in turn weakens the general applicability of the protocol, or to delegate execution to replicas and utilize State Machine Replication to serialize Transactions. SMR protocols with a single leader do not inquire any congestion based aborts as a single sequencer naturally eliminates concurrency.
Indicus does not make these concessions in order to offer interactive Transactions and remain scalable. A workload that exhibits low commutativity and high contention should therefore refrain from adopting our system.

Similarly, as is the case in any transaction protocol, Indicus is vulnerable to ddos attacks by byzantine participants. A byzantine clients only opportunity at subverting progress for honest users is to artificially increase congestion. When such a client has unrestricted access control it may do so strategically iff it has control over the network. If it does not, it cannot reliably gain knowledge about concurrent transactions before they pass the validation step and must resort to flooding based attacks. Defense against such attacks is out of scope in our work, but is disincentivised as participants can be held accountable for their actions in a closed membership setting.


\fs{technical limitations: We require more replicas to avoid certificate signature overheads. Since in 3f+1 one decision needs to be enough to recover. byzantine replicas/clients have more power to vote arbitrarily, because they cannot be held accountable for it. In total order protocols (SMR), deviation from the "common" vote signals misbehavior, whereas for us that is not the case. --> this should go as a challenge somewhere.}

\fs{Clients are more heavyweight. Not suitable for settings where clients just have minimal processing capacity. In practice we envision clients to be dedicated transaction maangers.
Also, Clients need to be registered in system with a sig - necessary to enforce access control in any system however}


%-------------------------------------------------------------------------------
\section{Related Work/Comparison}  
%-------------------------------------------------------------------------------

\paragraph{Consistent Replication}
Indicus offers a replicated, byzantine fault tolerant database that leverages concurrency control and quorum based agreement to maintain isolation guarantees despite its unordered/inconsistent replication layer. 
The majority of prior work instead maintains consistent replication by means of the State Machine Replication (SMR) approach \cite{schneider1990implementing}. Existing solutions for the crash failure fault model (Paxos based: \cite{Li2007, Lampson2001, Lamport98Paxos, Lamport2005a, Lamport2005, Lamport01Paxos, Chandra2007} , Zab: \cite{junqueira2011zab}  VR: \cite{oki1988viewstampeda, liskov2012viewstamped}, \cite{van2014vive}) as well as the byzantine fault model \cite{lamport2011byzantizing, pires2018generalized, Guerraoui08Next, Kotla04High, bessani2014state, liskov2010viewstamped} (most notably PBFT \cite{castro1999practical}) achieve consistency by achieving consensus on a totally ordered log of requests. 

The seminal \textit{practical} Byzantine Fault Tolerance (PBFT) protocol and its adaptations form the cornerstone of most Permissioned Blockchains \cite{Hyperledger, EthereumQuorum, buchman2016tendermint, al2017chainspace, kokoris2018omniledger,  gilad2017algorand, baudet2019state} (Algorand and Omniledger both not permissioned, but lots of PoS blockchains elect validator set that runs pbft basically, mention some traditional blockchains? Bitcoibn, Ehtereum, also total order).

\fs{mention zyzzyva explicitly because of Fast path concept and speculative execution of replicas. Difference: still leaderbased and clients EXPECT replicas to process in order.}


\paragraph{Enforcing Partial Order}
Totally ordered ledgers solve the problem of consistent transaction replication but enforce unecessary coordination/serialization between commutative operations. 
Approaches to reconciling this scalability limitation can be broadly categorized in three ways:
\textbf{1. Fine-grained ordering}: One class of approaches attempts to only order conflicting operations by examining transaction semantics. Existing work however (cite all CF) targets almost exlusively the crash failure model. The the best of our knowledge, Bilbos/Clairvoyant is the only BFT system that offers SMR for commutative transactions, but limits TX model and is "blocking" (i.e. no progress until all concurrent - potentially conflicting - TX that could be ordered before are finished). Other Quorum based systems exist (Q/U, H/Q, Liskov rendering harmless) but they do not offer transactions. Q/U resembles Logging in Indicus - but has no liveness on its own.

\textbf{2. Sharding} Omniledger, Chainspace explore sharding for BFT. We point out that sharding just for partial order is a hack.
Tapir/Janus points out that layering sharding on top of rpelication is inefficient in Crash Failure. Indicus is first to do this for BFT.

\textbf{3. DAGs} Iota, Byteball, Ava, all do DAGs which allows to parallelize some agreement. Existing work is all for permissionless systems. Unclear how to map this to permissioned and also whether it is practical.

All of these improve upon total order by inducing only a partial order. Indicus does so too.

-Commutativity approaches (CF and BFT - Bilbos, Q/U,)
-Sharding; replication and cc/atomic commit are layered. 
-DAGs

\paragraph{Byzantine Databases}
In indicus we argue that blockchain functionality should be though of as DB, transactional store. 
Series of work that offer explicite Byz DBs: Byzantium, Augustus, Callinicios, HRDB, BFT Dur,
Databasify Fabric


\paragraph{Byzantine Clients}
As part of reformulating the Blockchain problem we reformulate correctness guarantees in a byzantine setting. Specifically we address byzantine clients since Indicus aims to be leaderless. Other systems that explore byz clients (quorum systems): \cite{Liskov06Tolerating}, The DBS above only talk about how to tolerate.
Other work that explores fairness/correctness (Herlihy paper)


\iffalse


For ex: Both CF SMR [cite a gazillion] and BFT SMR [cite a gazillion] focus on maintaining a totally ordered log of requests
Permission blockchains, as they rely primarily on BFT constructs similarly maintain this total order
[cite a gazillion]

There is some work that recognizes that this is not scalable and instead tries to only order conflicting ops [Epaxos, Generalised Paxos, Ava, Iota, CPPaxos or whatever it's called], or add sharding [OmniLedger, Herlihy's paper]
In the CF context, layering sharding on top of  replication has been shown to be inefficient [Tapir, Janus]
We argue that we should think of blockchains as a DB. There is some work that does that already [the series of five papers that we talked about]. (one sentence for each on how they build on one other). We expand on that research to be 1) leaderless, scalable, etc
And then maybe after this, as part of reformulating this as a DB problem also need to reformulate the correctness guarantees (cite all work that talks about byzantine clients) 

\fi







\subsection{details}


Indicus is the first to do blablabla. This section describes related work...

\textbf{Distributed Transactions for benign faults}
A lot of recent effort has gone into designing high throughput and low latency databases that leverage synergies between transaction and replication layer to improve performance. The recently proposed TAPIR transaction protocol leverages redundancy between transaction ordering and replication ordering to reduce total roundtrips, thus reducing latency in wide area networks. TAPIR shares several similarities with our system, most notably the absence of a leader and the resulting unordered validation structure. Like Indicus, TAPIR utilizes optimistic concurrency control to allow for concurrency among commutative transactions while minimizing coordination. When congestion is high however, throughput and tail latency worsen as abort rate grows. Janus avoids transaction aborts by dynamically re-ordering conflicting transactions \cite{mu2016consolidating}. This however, is made possible by assuming one-shot transactions, i.e. fixed read/write keys, and thus reduces the application generality. Another key-value store, Carousel, similarly assumes a restricted Transaction model in order to parallelize execution and validation \cite{yan2018carousel}. CURP too, leverages commutativity in order to reduces replication costs by making requests durable first, and only ordering them later \cite{park2019exploiting}. 

While all of these databases offer low latency replication, they are fundamentally limited to tolerating crash-failures. This strong failure assumption makes them less robust and not suitable for mission-critical services \cite{Abdollah2007}, finance or otherwise highly sensitive data.

\textbf{Byzantine Fault Tolerant Replication} 
\fs{Really not necessary to explain all the different SMR protocols that exists. Rather: Should explain how they could be used to solve the same problem with total order?}
 The problem of tolerating arbitrary failures was originally formalized as \textit{Byzantine Generals Problem} \cite{lamport2019byzantine}, paving the way for countless Byzantine Fault Tolerant (BFT) protocols and implementations \cite{castro1999practical, martin2006fast, kotla2007zyzzyva, pires2018generalized, bessani2014state, lamport2011byzantizing, arun2019ezbft, malkhi2019flexible, duan2014hbft, yin2003separating}. Such protocols have a reputation of being  both notoriously difficult to understand and design correctly \cite{abraham2017revisiting, abraham2018revisiting, shrestha2019revisiting}, and in practice often impose significant overhead compared to its crash-failure tolerant counterparts, hence limiting frequent industrial adoption.
Recently however, with the surge in Blockchain interest, BFT protocols are experiencing a second Spring. While originally developed to tolerate arbitrary bugs, these protocols find increasing importance in settings where participants are untrusted or malicious. Permissioned Blockchains, organized by a consortium of registered participants can use traditional BFT State Machine Replication (SMR) protocols in order to achieve agreement. 
\fs{mention basic BFT first? pBFT is the first "practical", not the first BFT}
The seminal practical BFT (PBFT) protocol \cite{castro1999practical} implements a primary backup scheme that achieves consensus in three broadcast rounds, maintaining safety as long as the number of faulty participants is $<1/3$ (i.e. $n\geq 3f+1$), and offering liveness during synchronous operation. When primaries misbehave, a dedicated \textit{view-change protocol} allows for safe reconciliation. There have been numerous adaptations that aim to speed up common case and failure free execution such as FaB \cite{martin2006fast}, and most notably Zyzzyva \cite{kotla2007zyzzyva} which leverages speculative execution and a semi-client driven protocol to reduce latency. Akin to Indicus, replicas in Zyzzyva speculatively execute requests out-of-order, and rely on honest clients for reconciliation. Zyzzyva replicas however, \textit{expect} to process requests in-order, and diverge only temporarily as result of a byzantine leader. hBFT builds on Zyzzyva and attempts to improve faulty case behavior by returning the client responsibility of detecting inconsistencies to a dedicated replica checkpointing protocol. 
UpRight explores efforts to tolerate both byzantine and crash failures in order to make the overhead of deplyoing a BFT system practical.
SBFT \cite{gueta2018sbft} pursues the goal of scaling to scale to large replication degrees, while offering light clients and high throughput under benign faults. It adopts the UpRight replication model, and modifies PBFT by utilizing collectors, threshold signatures and a fast path akin to Zyzzyva. 
Aardvark \cite{clement2009making} emphasizes the importance of robust performance under byzantine failures by adding increased client and replica checks and imposing frequent leader rotation by steadily increasing throughput requirements. Tendermint \cite{buchman2016tendermint} pushes this idea to the extreme by rotating leaders continuously, at the cost of assuming a synchronous network. Herlihy and Mor attempt to rigorize fairness properties for permissioned Blockchains and propose techniques to increase accountability for Tendermint specifically \cite{herlihy2016enhancing}.

HotStuff too, explores the use of rotating leaders by exploiting the symmetry in PBFTs phases in order to pipeline transaction processing \cite{yin2019hotstuff}. 
Nevertheless, all protocols derived from the PBFT family suffer from the leader bottleneck as well as enforcing a total order even on commutative operations. BFT-Mir \cite{stathakopoulou2019mir} identifies the proposal speed of a single leader as the practical bottleneck in most implementations and improves upon this by allowing all replicas to act as proposers.
Biblos \cite{bazzi2018clairvoyant} achieves leaderless SMR by leveraging a non-skipping timestamp protocol. It furthermore allows for commutative transactions to be executed in parallel at the cost of requiring read/write key sets to be known in advance. In order to preserve liveness however, Bilbos falls back to a PBFT resolution. \fs{Uses all to all, predefined TX, pbft fallback, timestamp phase and proofs required}
The Quorum/Update (Q/U) protocol too, offers leaderless agreement by leveraging Qurorums for a agreement in a limited read/write interface, but fails to terminate under contention \cite{abd2005fault}. To reconcile this shortocming, the Hybrid Quorum protocol (HQ) explores a hybrid solution, extending upon Q/U by adding a PBFT fallback path under contention \cite{cowling2006hq}. In \cite{liskov2006tolerating}, Liskov et Al explore improved byzantine Quorum protocols for a single Read/Write interface, giving special attention to bounding the effects of byzantine clients. 
\fs{BFT protocols under fire: \cite{singh2008bft} that one-size-fits-all protocols may be hard if not impossible to design in practice}
All of the above systems, including Indicus, guarantee Liveness only under the assumption of some degree of synchrony. This is consistent with the well known FLP impossibility result stating that no deterministic consensus solution may exist in the presence of asynchrony \cite{fischer1985impossibility}. Protocols such as Ben-Or's algorithm \cite{ben1983another}, HoneybadgerBFT \cite{miller2016honey} or BEAT \cite{duan2018beat} sidestep this result by introducing randomization, thus enabling probabilistic agreement even during asynchrony.

\textbf{BFT distributed transactions}
While the literature on BFT state machine replication is extensive, the efforts to offer a transactional interface for BFT are scarce. SMR itself can be utilized as a straw-man system to implement pre-defined Transactions (one-shot or stored procedures) by enforcing a common total order in which replicas will execute the transactions. \fs{Byz agreement apps for DB \cite{garcia1986applications}}
HRDB offers a dedicated BFT Database, but assumes a trusted shepheard layer, thus not being truly BF resilient \cite{vandiver2007tolerating}.
Byantium \fs{middleware stuff} offers Snapshot Isolation for Transactions by re-purposing PBFT as atomic broadcast. It executes requests only at a primary and uses replicas to validate results in the total order defined by the SMR protocol \cite{garcia2011efficient}. Augustus implements short-transactions, a limited, comparison/query/update transaction model that declares all operations before execution \cite{padilha2013augustus}. It offers scalability via partitioning and achieves consistency within partitions by utilizing atomic broadcast. While Augustus assumes an optimistic execution model that allows for aborts under concurrency, its follow up work Callinicos implements a locking scheme in order to avoid Transaction aborts \cite{padilha2016callinicos}. To do so efficiently it resorts to a limited transaction model that requires knowledge of read/write sets and otherwise locks an entire partition in order to guarantee mutual exclusion, thus voiding any concurrency. BFT Deferred Update Replication adopts an interactive OCC transaction model comparable to ours, allowing execution to be speculative at clients \cite{pedone2012byzantine}. However, it uses PBFT as atomic broadcast implementation to enforce SMR on validation, thus resorting to both a leader, and a total order on all requests. It moreover does not extend to multiple shards.

\textbf{Blockchains and Sharding}
In the Blockchain world, Chainspace \cite{al2017chainspace} and Omniledger \cite{kokoris2018omniledger} implement sharding by layering 2PC and atomic broadcast (within shards) for the UTXO transaction model. Rapidchain \cite{zamani2018rapidchain} offers efficient sharding for a permissionless system (what else? didnt read much, not so relevant). Hyperledger provides an enterprise-ready permissioned blockchain solution that simulates optimistic transaction processing by letting replicas speculatively execute Smart Contracts. While Hyperledger \cite{Hyperledger} describes itself as a shared database, it relies on a dedicated SMR ordering service that enforces a total order, and, in practice, only tolerates crash failures (Hyperledger uses Raft \cite{ongaro2014search}, a consensus protocol for the Crash-Failure model,  but can be extended to use PBFT). Sharma et al. attempt to \textit{databaseify} Hyperledger by adding database techniques such as transaction re-ordering in order to reduce aborts, but fundamentally maintains the totally ordered (Hyper-) ledger design \cite{sharma2018databasify}. \fs{Databasify paper: tries to add some bandaid db techniques to hyperledger. Transaction re-ordering (still total order), and early aborts.} 
 \fs{(Mention other Permissioned Blockchain systems: Ethereum Quorum? (uses Raft or IstanbulBFT = implementation of pbft)}

Directed Acylcic Graph (DAG) based consensus architectures \cite{pervez2018comparative}(IoT-chain, IOTA, Byteball - all are blockless) have been explored  in order to parallelize agreement and execution instances. While these approaches increase throughput by inducing a partial order, they target permissionelss settings (based on Proof of Work/Stake), rather than permissioned BFT-based solutions. To the best of our knowledge there exists no such system to this date, and it is unclear how to merge these approaches. Designing a BFT protocol that replicates a DAG instead of a totally ordered ledger is an interesting avenue for future work.






\fs{How does all this related work compare to Indicus?}




%-------------------------------------------------------------------------------
\section{Conclusion}
%-------------------------------------------------------------------------------

summary of Indicus

Our experience designing Indicus.

opportunities for future systems? what kind of systems maintain the defs?



\iffalse
%-------------------------------------------------------------------------------
\section*{Acknowledgments}
%-------------------------------------------------------------------------------
%see ack.txt

%-------------------------------------------------------------------------------
\section*{Availability}
%-------------------------------------------------------------------------------

USENIX program committees give extra points to submissions that are
backed by artifacts that are publicly available. If you made your code
or data available, it's worth mentioning this fact in a dedicated
section.
\fi
%-------------------------------------------------------------------------------
\bibliographystyle{plain}
%\bibliography{ref}
\bibliography{refs}

\tr{
\appendix
%-------------------------------------------------------------------------------
\section{Proofs}
%-------------------------------------------------------------------------------

\begin{theorem}[1]
The set of transactions for which the MVTSO-Check returns Commit is Byzantine-Serializable. 
\end{theorem}
\begin{proof}

We first show, that the MVTSO-Check never returns Commit for two conflicting transactions TX1 and TX2.
WLOG, assume that TX1 is validated before TX2 and MVTSO-Check(TX1, TX1.TS) returns Commit. This implies that all reads and writes of TX1 are \textit{Prepared}.
We show by case distinction that TX2's validation cannot return Commit:
\begin{itemize}

\item \textbf{TX1.TS < TX2.TS} By timestamp order, the excution results of TX1 and TX2 should be equivalent to a serial schedule in which TX1 happened \textit{before} TX2. A conflict only arises, if TX2 performed a read that should have seen TX1's write, i.e. if TX2's $read.version < TX1.TS$. By lines 3-7 TX2 must Abstain/Abort since TX1 $\in Prepared$. 

\item \textbf{TX1.TS > TX2.TS:} By timestamp order, the excution results of TX1 and TX2 should be equivalent to a serial schedule in which TX1 happened \textit{after} TX2. A conflict only arises, if TX2 attempts to write a version that TX1 should have observed, i.e. if TX1's $read.version < TX2.TS$. By lines 8-10 TX2 must Abstain/Abort since TX1 $\in Prepared$. 
\end{itemize}

We now show, that every honest clients transaction is \textit{legal}, i.e. its reads are based on committed writes. This follow straightforward from Execution protocol: An honest client only reads values that are either committed, or optimistically reads by claiming a dependency on uncommitted writes. By lines 14-18, a TX with dependencies only Commits if all of its dependencies Commit, implying that it read committed writes.

Consequently, all honest clients experience Serializability, and thus, the MVTSO-check maintains Byzantine-Serializability.
\end{proof}

\fs{should split this between Slow and Fast Path. Logged decisions are only slow path. Fast Path implies that the only logged decisions that are possible will be consistent}
\fs{redefine logged decision: 3f+1 correct replicas have adopted a decision. Define shard-certificate: set of decisions that implies existance of a logged decision.}
\begin{theorem}[Saf]
A logged decision is durable, and there can ever exist \textbf{at most one} logged decision.
\fs{Clarify the point of this: We preserve consistency of writeback commit/abort. That in turn follows from the logged decision durability.}
\end{theorem}
\begin{proof}

We show this by case distinction over Slow- and Fast-Path execution. \textbf{Slow-Path}: A Slow-Path decision is \textit{logged} if $\frac{n+f+1}{2} = 3f+1$ ($LoggedQuorum$) honest replicas have adopted the decision, since $n-f = 4f+1$ votes suffice to form a Shard-Certificate and $f$ byzantine participants may decide arbitrarily. Thus it is impossible for two Slow-Path logged decisions to co-exist, as any two $LoggedQuorums$ must intersect in $f+1$ honest replicas that will only ever accept one decision - a contradiction. Furthermore, honest replicas do not change their decision, and hence a slow-path logged decision is durable. \textbf{Fast-Path}: We distinguish three sub-cases. \one A Fast-Path commit shard-certificate requires the existance of $4f+1$ Phase1 honest commit votes. This implies that any Slow-Path execution must result in a commit decision, since there cannot exist $f+1$ Abort votes and client waiting for up to a Vote Set Quorum ($n-f = 4f+1$) is bound to receive $3f+1$ commit votes. \two Vice versa, by Quorum intersection, the existance of $3f+1$ Phase1 abort votes implies the impossibility of any Slow-Path commit decision. 
Trivially, cases \one and \two  mutually exclude each other. 
\three Lastly, 1 Abort vote with valid proof of conflict against a committed transaction implies the existance of a logged commit decision for the conflicting transaction. \fs{Defer this proof to the next theorem... redundant}
By Induction, the conflicting transaction has stored a commit vote or decision at $\geq 3f+1$ correct replicas. Thus, $\geq 3f+1$ correct replicas will vote to Abort the ongoing transaction, hence precluding the existance of $\geq 3f+1$ commit votes and the possibility to ever log a commit decision for the ongoing Transaction.


\end{proof} 

Note, that since replicas never change their decision, it is possible for there to never be any logged decision if a byzantine client equivocated its Slow-Path Quorums. In order to reconcile this, we design and discuss a recovery mechanism in section X which relaxes the requirement for durability of decisions.  


\begin{theorem} 
Indicus maintains \textit{Byzantine-Serializability}.
\end{theorem}
To prove that this is the case, we show that for any two conflicting transactions, at most one can be committed.
\begin{proof}

Let $TX_1$ be a transaction with logged decision \textit{Commit}, i.e. $TX_1$ has either already committed at or is bound to commit at all honest replicas. Let $TX_2$ be a conflicting transaction, that if committed would violate Byzantine-Serializability. Assume $TX_2$ too, managed to log a commit decision. By the protocol, at least  $\frac{n+f+1}{2} = 3f+1$ commit votes are required to log a commit decision, and no honest replica changes its vote. By Quorum intersection, at least one honest replica must have voted commit for both $TX_1$ and $TX_2$. WLOG, this replica received $TX_1$ before $TX_2$. By the correctness of the MVTSO-check (\ref{Theorem1}, must have voted Abort for TX2. A contradiction.


\end{proof}

\begin{theorem} 
Indicus maintains Byzantine Independence in the absence of network adversary.
\end{theorem}

We show, that once a Client submits a transaction for validation, the result cannot be unilaterally decided by any group of (colluding) byzantine participant, be it client or replica.
\begin{proof}

First, we observe that a client may never autonomously determine a result itself, and may only influence the decision value by choice of Fast- or Slow-Path Quorum. Specifically, a byzantine client cannot single-handedly decide to abort its own transaction once submitted for validation, and consequently, cannot single-handedly force potentially dependent transactions to abort as well. 
Second, any Quorum decision requires at least one honest replicas vote. In particular, $f$ byzantine replicas may arbitrarily vote to abort (e.g. by reactively generating and injecting a ficticious, conflciting transaction), but at least one additional correct replicas' abort vote is necessary to result in an abort decision. 
Thus, in order to artificially cause transactions to abort, a conflicting transaction must be generated (artificial congestion). However, to do so strategically and reliably \fs{deterministically}, the adversary must control the network in order to guarantee the artifical transactions arrival at honest replicas \textit{before} the original transaction. 

Likewise, colluding clients may attempt to abort each other in order to cause a correct clients' dependency to abort. For example, they may attempt to schedule the arrival of two conflicting transactions out of order across replicas. To preemptively setup such a targeted trap however, colluding clients must a) know the to-be-victim transactions read keys, and b) race the respective transaction for arrival order. Doing so deterministically requires control of the network.

Thus, when the network is not adversarial, validation decisions are \textit{Byzantine Independent}.

\fs{need to add the case of dependency trying to get aborted by its own dependent. this too is up to the network: A byz replica does not know that there is a dependent until the exceptions or prepares are being issued. needs to have network control in order to still abort. if there are multiple levels then the colluders could already pre-abort each other: example: out of order at 3 replicas each. any TX coming after that claims a dep is doomed. But this is not deterministic: it requires preemtive setup, but keys are not known}


\end{proof}


\begin{theorem}[Liv] 
Every transaction that an honest client is \textit{interested} in eventually completes.
\fs{Clarify: Exactly one shard-decision exists. I.e. Safety is maintained, but there will be one decision. }
\end{theorem}

\begin{proof}
First, we note, that a timely client can trivially complete all of its own transactions that have no dependencies. However, if a client is slow, or its transaction has dependencies, it may lose autonomy over its own transaction. For a given client c, we define the set \textit{Interested$_c$} to include its own tansactions and all their dependencies, as well as any other arbitrary transactions whose completion a client is interested in. 

We distinguish two cases for each $TX \in Interested_c$ that has timed out on its orignial client: 
1) An interested client manages to receive a shard-certificate by either re-issuing a $Phase1$ message and receiving a Fast-Path Threshold of $Phase1R$ messages, or by issuing a new $Phase2$ message and receiving a Slow-Path Quorum of $Phase2R$ messages. In this case, a client is able to complete the transaction as any client may issue the Writeback. Saf is maintained as this case follows the normal-case protocol operation.\\

2) An interested client cannot obtain shard-certificate and starts a Fallback invocation. By Lemma1, given synchrony, a correct Fallback replica will be elected after at most $f+1$ election rounds. Such a Fallback replica will reconcile a consistent decision across all replicas, thus allowing the interested client to receive a Slow-Path Quorum of $Phase2R$ messages (matching decisions, matching views) as shard-certificate, allowing it to complete the Writeback phase. By Lemma2, this decision is consistent with any potential past shard-certificate that may have existed.

\end{proof}



\begin{lemma}
Given weak synchrony, Fallback election is live and non-skipping.
\end{lemma}

\underline{\textbf{View Change Rules:}} \one Replicas only adopt a view $v+1$ if the view set includes $3f+1$ votes from view $v$. \textit{Vote subsumtion:} A view $v$ may count as a vote for all $v' \leq v$. \two Replicas that lag behind, may safely skip ahead to the maximum view $v$ present $f+1$ times, since, by induction, $\geq 3f+1$ replicas must have claimed to be in a view $\geq v-1$. \\
We note, that replicas enforce exponential time-outs on each new tenure: A replica will not adopt a new view and start a new election, until the previous view leader (client or Fallback replica) has elapsed its time-out.

\begin{proof}
A client must provide $3f+1$ matching view responses in order for replicas to adopt the next view and start an election. This implies, that for each correct replica in view $v$, there must exist at least $2f+1$ correct replicas in view $\geq v-1$ (A).
If a client cannot receive such a Quorum, e.g. due to temorary view inconsistency, it must reconcile the views first. This is possible in a single step: By (A) any set of $4f+1$ replica responses must contain $\geq f+1$ correct replicas' votes for a view $v' \geq max(correct.views) -1 $. 
Thus, in the presence on a correct interested clients, and when time-outs grow large enough to enforce synchrony, Byzantine clients cannot stop the successful election of a new Fallback by continuously invoking a view-change. Moreover, election is non-skipping, as a correct client will broadcast a new-view invocation to all replicas. 
\end{proof}

\underline{Practical optimizations:}  In absence of honest interested clients, byzantine clients may invoke fallback elections at a subset of replicas, thus inhibiting true election and skipping select replicas' terms. To avoid artificially increased timeouts and non-skipping candidates, replicas may forward the $ElectFB$ message to all other replicas. Replicas that receive $f+1$ forwarded messages, adopt larger views, and forward the $ElectFB$ message themselves, thus ensuring that all correct replicas adopt each new view, and a Fallback replica is successfully endorsed.
This optimization is not necessary for "theoretical" liveness as shown by Lemma1. To avoid unecessary all-to-all communication, it may only be enforced for views $v > T$, where T is a system hyperparameter.\\

%%%%%%%%%%%%%%%%




\begin{lemma}
The Decision Reconciliation Rule maintains both Saf and Byz-Serializability.
\end{lemma}

The Fallback mechanism extends the requirment for shard-certificates to match in both decisions and views. Consequently, a logged decision is a pair of (decision, view) that, at one point in time, is adopted by $\geq 3f+1$  correct replicas.

\textbf{Decision Reconciliation Rule:} \textit{dec$_{new}$ $=$ maj(\{Elect.decision\})}. Note, that for decision reconciliation, the associated view of a decision is explicitly irrelevant.

\begin{proof} If a logged decision exists, i.e. $3f+1$ correct replicas have adopted the same decision, any Quorum of $4f+1$ Elect messages is guaranteed to contain $2f+1$ matching decisions (a majority). Thus, by the Decision Reconciliation Rule, the only new decision possible is the previously logged decision. By induction, this hols for all consecutive views: If there ever existed a logged decision $d$ in view $v$, then all $dec_{v' > v} = d$. Consequently, Elect messages for future, higher views, do not need not match in their decision views.

If no logged decision existed, then any majority decision qualifies. This is consistent with Byz-Serializability since the majority decision contains $\geq f+1$ correct replicas, which would only have agreed to a decision on the basis of a matching Quorum of $Phase1R$ messages.
\end{proof}

\textit{Aside:} If shard-certificates were not required to match in views, Saf would be violated. Consider an example in which for a view $v$ half of the replicas have adopted decisions Commit/Abort respectively. Let $P_1$ (Commit) and $P_2$ (Abort) be the partitions of replicas with matching votes. A Byzantine Fallback may use two Elect Quorums with different majorities to send $dec_{v+1}$ = Abort to $P_1$, and $dec_{v+1}$ = Commit to $P_2$. Then shard-certificates for both Commit/Abort could exist by using $P_1$ from view $v$ and $P_2$ from view $v+1$, and vice versa, violating Saf. 
}{}

%%%%%%%%%%%%%%%%%%%%%%%%%%%%%%%%%%%%%%%%%%%%%%%%%%%%%%%%%%%%%%%%%%%%%%%%%%%%%%%%
\end{document}
%%%%%%%%%%%%%%%%%%%%%%%%%%%%%%%%%%%%%%%%%%%%%%%%%%%%%%%%%%%%%%%%%%%%%%%%%%%%%%%%

%%  LocalWords:  endnotes includegraphics fread ptr nobj noindent
%%  LocalWords:  pdflatex acks
