%%-------------------------------------------------------------------------------
\subsection{Execution}
%-------------------------------------------------------------------------------
The goals of \sys{}'s execution protocol are threefold: 1) to offer clients an interactive transaction interface, 2) to be scalable, and 3) to offer honest clients independent operability.

\sys{} achieves all three of these goals though the combination of optimistic client-side execution, and an aggressive, but byzantine resilient, concurrency control (CC) scheme. Rather than submitting stored procedures that must be ordered and executed by replica servers, \sys clients may execute arbitrary transactions by relying on RPC's to conduct transactions themselves. This places computational responsibility with the clients, and allows commutative transactions to execute in parallel. Since \sys clients execute optimistically concurrently, \sys must enforce CC in order to reconcile Isolation conflicts might arise.
By relying on optimistic rather than pessimistic CC schemes such as 2-phase-locking (2PL), \sys{} forgoes costly coordination to acquire locks, and sidesteps the concern of byzantine clients refusing to relinquish locks. 

Before we detail \sys{}'s transaction processing and validation mechanism, we first discuss the CC that \sys implements as the latter are functions of the precise CC requirements. 
 
\subsubsection{Concurrency Control}

%%% Explain MVTSO itself only briefly. Then explain how we make it for for byz
% 1) bound timestamps, 2) make writes visible late only, 3) only allow f+1 matchin uncommitted writes, 4) read timestamps
%Afterwards: explain execution interface. After that explain validation check. Then have overarching example
\iffalse
% we need to change a few things:
0) timestmap bound
1) Read validity
2) deferred writes
3) read uncommitted
4) byz read timestamps (read lock implies you had access control, at that point it is no different than issuing a tx and completing. But we do not want to give the power to abort for no reason. It should be traceable.
\fi

Figure \ref{fig:MVTSOEX} shows an example trace of \sys operational behaviors under different request processing orders (assume f=1) that we use to guide the remaining outline.

\begin{figure*}
\begin{center}
\includegraphics[width= \textwidth]{./figures/MVTSOLargeFont.png}
\end{center}
\caption{\emph{MVTSO behavior for different replica processing orders}. $r_x(C : a_y ,P : a_z)$ denotes that transaction $T_x$ (x being the timestamp of a transaction in this example) reads the version $y$ of object a written by committed transaction $T_y$ and version $z$ from tentative prepared transaction $T_z$. $P_x(RS,WS,DEP)$ denotes a transaction $T_x$'s prepare request (i.e. the validation check) for respective ReadSet (RS), WriteSet (WS) and Dependenc Set (DEP), and $\rightarrow C / A$ denotes the local replica validation outcome (Commit/Abort).} 
\label{fig:MVTSOEX}
\end{figure*}

Our starting point is a Multiversioned Timestamp Ordering scheme (MVTSO) which prescribes a serialization order by assigning a speculative timestamp \textit{before} Execution \cite{bernstein1983multiversion, reed1983implementing, su2017tebaldi}. This allows us to a) only classify concurent transactions as conflicting if their execution results violate the timestamp order (Fig. \ref{fig:MVTSOEX}: 4 bottom), and b) avoid write/write conflicts alltogether by applying them in order. 
When speculative execution results match the pre-defined timestamp order, no aborts are necessary (Fig. \ref{fig:MVTSOEX}: 4 top). \sys challenge is therefore to a) assign appropriate timestamps and b) coordinate execution in a way that maximizes such coherence; the presence of byzantine participants (clients/replicas) complicates this.


In MVTSO, reads operations return the latest written version smaller than the readers' timestamps, while writers attempt to create new versions at their timestamp, but must abort if a higher timestamped read would have "missed the write" by reading a prior version. A transaction may commit only, if all read-write dependencies have committed respectively.

Read operations in \sys have the following sub-goals: \one Honest clients should read valid data, i.e. experience read integrity, \two Honest clients should read fresh data, i.e. minimize staleness and hence maximize commit chance, and \three Reads must provide the context necessary to potentially complete observed write state. 
Clients validate the integrity of reads by requiring replicas to provide a proof of validity, i.e. a set of signatures confirming the committment, or if not yet existant, trusting  only $f+1$ matching replies from discrete replicas. Moreover, in order to guarantee that clients do not read maliciously stale data from their local replica, \sys encourages clients to always read from at least $f+1$ replicas and choose the freshest version (Fig. \ref{fig:MVTSOEX}: 1,2).
Avoiding byzantine influence comes at a cost: Read operations require a synchronous, potentially WAN, rountrip. 

MVTSO intuitively synergizes well with \sys{}'s potenially WAN remote reads as reading from a fixed timestamp helps speculative readers observe consistent snapshots, even when execution is long and consequently interleavings are frequent (Fig. \ref{fig:MVTSOEX}: 6). \fs{at the price of experiencing serializability instead of strict serializability - that is only the case if your timestamps are outdated}.

While we assume that clocks are loosely synchronized across honest participants, byzantine participants may diverge arbitrarily in undesirable fashion. To side-step the additional overhead that a dedicated, bounded timestamping phase \cite{Clairvoyant} incurs, we compromise by allowing clients to optimistically select their own timestamp, but rejecting request timestamps above a Threshold at all honest replicas, thus incentivising clients to select real-time timestamps. In order to facilitate a total serialization order across all clients, we define Timestamps to be tuples of the form $(Time, CID)$. 

In traditional MVTSO, writes become visible to successive reads with higher timestamps immediately. In the presence of byzantine clients this however, is undesirable, as it allows clients to issue singular write operations without the intention of ever committing a transaction. Thus, any honest clients' read that observes, and consequently depends on, a byzantine clients write may be blocked indefinitely, thus compromising its liveness.
Allowing other clients to preemtively commit outstanding write operations is infeasible, as the remaining transaction procedure is known only to the issuing client, while conceiding pessimistic abort permission empowers byzantine participants to obstruct any honest clients' writes. 
To reconcile this dilemma, \sys{} defers all database updates until execution has finalized. Concretely, \sys clients buffer all write operations, and submit them only when attempting to commit the transaction, thereby enabling other \sys client to orderly complete Validation and Writeback steps. 

Clients in \sys distinguish explicitly between \textit{committed} and tentatively, but unconfirmed \textit{prepared} writes: While clients trust and accept every verified \textit{committed} write versions, they accept tentative \textit{prepared} versions only when observing $f+1$ matching results. This ensures that a) at least one honest replica believes that the write can commit, and hence is worth observing, and b) that byzantine replicas cannot violate byzantine independence by reactively inventing transactions that are guaranteed to abort.

Finally, \sys replicas abort all writes that higher timestamped reads "miss" by storing a Read Timestamps (RTS) for each locally processed read (Fig. \ref{fig:MVTSOEX}: 3). While this allows in-execution reads to elicit external effects, much like outstanding write operations discussed above, these are only limited to concurrent transactions with smaller timestamps and hence are straightforward to bypass by re-trying a transaction. To nonetheless limit repeatable abuse , we discuss a practical mechanism to limit byzantine influence in section Y.z (Optional Modifc. personalized read leases).



\subsubsection{Execution interface}
Client applications execute transactions via the following interface. A TX object \textit{TXObj $\coloneqq$ (SeqNo, ClientID, InvolvedShards, ReadSet, WriteSet, Dependencies)}, records the state necessary for Validation.

\iffalse
\begin{figure}
\begin{center}
\includegraphics[width= 0.5\textwidth]{./figures/TxState.png}
\end{center}
\caption{Transaction execution state}
\label{fig:Txstate}
\end{figure}
\fi

\textbf{Begin()} A client begins a transaction by optimistically choosing a timestamp \textit{TS $\coloneqq$ (Time, ClientID)}. \\
\textbf{Write(key, value)}. A client buffers the write: \textit{WriteSet = WriteSet $\cup$ (key, value)}\\
%%%%%%%%%Read protocol%%%%%%%%%%%
\textbf{Read(key, TS, RQS)} 
If \textit{key $\in$ WriteSet} a client returns the buffered write value. Otherwise, the client conducts a remote read:

\fbox{\begin{minipage}{22em}

\textbf{1: C} $\rightarrow$ \textbf{R}: Client sends read request to Replicas
\end{minipage}}\\
Given a hyperparameter Read Quorum Size (RQS), a client sends $m = (Read, key, TS)_c$  to $RQS$ different replicas. Note, that in order to guarantee $\geq RQS$ replies a client might need to send up to $f$ additional requests to compensate for unresponsive/faulty participants ($max(|Replies|) \leq n-f$). 

\fbox{\begin{minipage}{22em}
\textbf{2: R} $\rightarrow$ \textbf{C}: Replica processes client read and replies
\end{minipage}}\\
A replica authenticates the client\footnote{Byzantine replica may ignore read access control. Solving this problem is beyond the scope of this work; we defer to existing solutions \cite{basu2019efficient}.}, and whether the timestamp is within a local highwater mark ($HW = localClock + \delta$). It returns a signed message \text{$\langle \textit{ReadReply, Committed, Prepared} \rangle _R$}, where $Committed \coloneqq (value, version, proof)$ represents the value-version pair with largest committed write version smaller than TS and a proof of commitment,  and $Prepared \coloneqq (value, version, TxID', deps)$ the respective largest uncommitted value-version pair, the associated transaction ID, and the latter transactions potentially uncommitted read-write dependencies. Moreover, a Replica stores a new read timestamp (RTS) for the key: $RTS(key) = RTS(key) \cup TS$ (Fig. \ref{fig:MVTSOEX}: 3). 
\fs{a replica may include a set of prepared values to increase likelihood of client receiving f+1 matching}

\fbox{\begin{minipage}{22em}
\textbf{3: C} ($\rightarrow$ \textbf{R}): Client receives read replies 
\end{minipage}}\\
A client waits for RQS read replies and chooses the biggest valid result \textit{(value, version) $= max_{valid}$(\{Committed\},\{Prepared\}} (Figure \ref{fig:MVTSOEX}: 1,2) . A \textit{Committed} tuple is valid, if the proof confirms commitment, wheras a \textit{Prepared} is valid iff there exist $f+1$ matching \textit{Prepared$_r$} signed by different replicas. The client adds the version to its read set \textit{ReadSet = ReadSet $\cup$ (key, version)} and additionally claims a dependency if it was a \textit{Prepared} version: \textit{Dep = Dep $\cup$ f+1 $\times$ (Prepared$_r$)} . 

\textbf{Commit()} A Client terminates its execution, and computes a unique transaction identifier based on final execution object and its timestamp: $TxID \coloneqq (H(TxObj, TS)$, thus preculding byzantine participants from equivocating transaction contents. It then begins the Validation Phase by issuing a 2PC-Prepare requests to each relevant shard and waiting for all votes.

\textbf{Abort()} A client terminates execution, and broadcasts a request to release all acquired Read Timestamps (RTS). Since writes in \sys are deferred, no other rollback action is necessary.


We briefly discuss some implications of the choice of Read Quorum Size (RQS). Following cases may be distinguished: \one \textbf{$RQS = 1$} A replica may read from just 1 replica at the risk of reading maliciously stale data. If a Client trusts a local replica \textbf{and} the replica is not lagging behind, this can reduce execution latency and consequently minimize conflicting interleavings. \two \textbf{$RQS \geq f+1$} \sys{}'s recommended mode of operation. While side-stepping maliciously stale reads, it is still possible to read (arbitrarily) stale data, either due to inconsistency caused by asynchrony or a byzantine client not fully replicating its transaction. 
\three \textbf{$RQS \geq \frac{n+f+1}{2}$} When reading from a Quorum of sufficient size to overlap with any Validation Quorum (ref section Validation) in at least one honest replica, a client guarantees, that no more \textbf{additional}, beyond previously observed, conflicting writes can be admitted, since such a Quorum of acquired RTS acts as a read-lock. 

\iffalse
Trusting only $f+1$ matching prepared writes has the beneficiary side effect of allowing us to include only transaction identifiers as dependencies, rather than the full transaction, since it is guaranteed that at least one honest replica has stored the transaction. Thus, in the failure free scenario, where clients need not complete transactions $in Dep$, \sys minimizes the meta-data overhead that the recovery protocol imposes.
\fi

We remark, that a byzantine client is not required to follow any of the protocol steps, with the exception of claiming dependencies. By our definition of Byzantine-Isolation, read integrity is only required for honest clients; Byzantine clients may \textit{choose} whether to read legal, or even real data. We do, however, require that no client can fabricate dependencies in order to preclude byzantine clients from indetectibly stalling their own transactions and consequently obstructing liveness for consecutive descendant (second degree...) dependencies.
Trusting only $f+1$ matching prepared writes allows us to only include only transaction identifiers as dependencies, rather than the full transaction, thereby minimizing the common path overhead. When the full transaction information is required to complete potentially stalled transactions, it is guaranteed to be obtainable from an honest replica. We discuss how to complete stalled or slow transactions in section Y (Granting Liveness).





%%-------------------------------------------------------------------------------
\subsection{Validation Phase}
%-------------------------------------------------------------------------------

In order to facilitate atomic committment across all shards involved in a transaction \sys clients invoke a Prepare request, inquiring each shard to validate the transaction for conflicts and cast a 2PC vote. Shards in \sys are replicated for fault tolerance.

\sys comes in two different flavors, \sys{}3 and \sys{}5 respectively, that rely on varying replication degrees, but implement the same design. \sys{}3 requires $n=3f+1$ replicas \fs{, the minimum bound necessary for BFT SMR?,} per shard to guarantee consistency in the presence of $\leq f$ byzantine replicas. \sys{}5 reduces both latencies during failure free execution and complexity during recovery. \footnote{We believe that consortiums with high performance requirements or high replication degrees are respectively comfortable with paying for additional replicas or tolerating a lower fraction (1/3 vs 1/5) of failures}
For the simplicity of exposition we discuss \sys{}5 for the remainder of the paper, and defer to section X \fs{and/or TR} to describe differences in \sys{}3. In the following, we outline \sys 's execution (unaffected by replication degree), validation and writeback protocols.

The goal of the validation phase is straightforward:
Shards need to cast a single durable vote, in the following we refer to this as the shard-decision. It is computed on the basis of different replica votes.
The vote needs to respect isolation guarantees
leaderless and partially ordered (i.e. coordination only when necessary)



The goals of the Validation phase are threefold: \one It must decide on a single, durable vote per-shard that maintains Byzantine Serializability (henceforth we refer to this as a \textit{shard-decision}), \two It should be leaderless, and not enforce unecessary ordering for commutative, non-conflicting transactions, and \three It must preserve independent operability. \fs{needs to refer to the fallback}

Satisfying these goals requires ovecoming several challenges. In order to maximize parallelism and embrace partial ordering, Indicus allows replicas to process requests out of order. Consequently, replicas may temporarily diverge, and hence return different validation results. Such divergence must be reconciled in a way that maintains Isolation, but not overly conservatively in order to maximize the ability to commit successfully and bound the impact of byzantine participants. Further, Indicus designates clients as validation coordinator for its own transactions, thus omitting a dedicated replica leader. The respective protocol must tolerate client failures such as crashes, omission/stalling, equivocation, or replays and allow for consistent recovery. 

The Validation protocol can be broken down into two functionalities: Voting and Logging.
Since \sys replicas may process requests in different order, they must reach consensus on a joint  decision. To do so, they cast a vote for their local validation result, which are subsequently democratically aggregated into a decision. In the case of Indicus, the client acts as the transaction coordinator who aggregates and relays results (along with necessary evidence).
In order to maintain consistency in the presence of failures (no two honest replicas finalize different Commit/Abort decisions), this decision must be durable and unique, guaranteeing that a replay of any Writeback is idempotent. However, since a byzantine coordinator cannot be trusted to durably store a decision, nor could we retain liveness during a crash or partition, \sys demands clients to \textit{log} the decision at replicas before returning.
 

The voting step requires a single round-trip to all replicas, whereas the logging phase requires at most one round-trip \fs{in Indicus5 and at most two round-trips in Indicus3.}. When execution is fault- and contention free transactions can be committed on the \textit{Fast-Path} in a single-round trip as an explicit logging round is not necessary.


%%%%%%%%%% protocol
To describe how the protocol operates in detail we follow a single-shard transaction through the system:

\fbox{\begin{minipage}{23em}
\textbf{(1: C $\rightarrow$ R)}: Client sends Prepare request to all Replicas within the Shard.
\end{minipage}}\\
Upon deciding to Commit in the Execution phase, a Client initiates Validation by sending a message $Phase1 \coloneqq \langle Prepare, TxID, TX \rangle_{\sigma_c}$ to all Replicas.

\fbox{\begin{minipage}{23em}
\textbf{(2: R $\rightarrow$ C)}: Replica receives validation request, processes it and returns vote to Client.
\end{minipage}}\\
A replica validates Timestamp and Dependency integrity of the request. It then evaluates Read and Write Sets for Isolation conflicts against its local state using the MVTSO Concurrency Control Check (CCC), as shown in algorithm 1. It returns a message $Phase1R \coloneqq \langle TxID, vote \rangle_r$, and optionally evidence in case it voted to Abort.

\underline{Additional subtlelties}: 
A replica never changes its Voting decision, because re-execution could leave to different results. Once the MVTSO-Ceck completes (i.e. there are no blocking dependencies), a replica starts a timer to monitor the clients progress.

\fbox{\begin{minipage}{23em}
\textbf{(3: C)}: Client waits for vote replies.
\end{minipage}}\\
A client waits for at least $n-f$ ($4f+1$ in Indicus5) distinct replica votes, or more, up to a system specified timeout. 

\fbox{\begin{minipage}{23em}
\textbf{(3a: C)}: Client receives Threshold of matching votes and returns to application. Proceeds to Writeback
\end{minipage}}\\
In any of the following 3 cases, a client may short-circuit waiting for additional votes and omit a dedicated Logging round:
\begin{enumerate}
\item \textbf{$1$ Abort vote w/ Conflicting TX \& CommitCertificate}: A conflict with a commited transaction. The client validates the integrity of the CommitCertificate and returns the shard decision $(TxID, Abort, \langle ConflTX \rangle_{CC})$. 
\item \textbf{$3f+1$ Abort votes w/ Conflicting TX}: A conflict with a prepared, but not yet committed transaction. The client returns the shard decision $(TxID, Abort, \{\langle AbstainVote\rangle_r\})$. 
\item \textbf{$5f+1$ Commit votes}: No conflicts. The client returns the shard decision $(TxID, Commit, \{\langle CommitVote \rangle_r\}$
\end{enumerate}
Any of such Quorums forms a \textit{Shard-Certificate} and proves the decision. A client uses this Certificate to return to its application and issue the Writeback.

\fbox{\begin{minipage}{23em}
\textbf{(3b: C $\rightarrow$ R)}: Client receives divergent results and suggests a consistent decision to Replicas for Logging
\end{minipage}}\\
If a client does not receive the necessary thresholds of votes to return, it must continue on the \textit{Slow-Path}. To do so, it aggregates the votes according to following decison rule:
If there exists a $CommitQuorum \coloneqq \frac{n+f+1}{2}$ of Commit Votes, the Slow-Path decision is Commit, otherwise it is Abort.
A client broadcasts a message $Phase2 \coloneqq (TxID, decision)_c, \{\langle votes \rangle_r\}$.

\underline{Additional subtlelties}: A client forwards a Quorum of $\geq n-f$ votes to the replicas in order to prove the Slow-Path decision is consistent with Isolation guarantees. Note, that a byzantine client may equivocate the decision by relaying different Quorums.

\fbox{\begin{minipage}{23em}
\textbf{(4: R $\rightarrow$ C)}: Replicas receive, validate and echo decision
\end{minipage}}\\
A replica confirms that the Decision matches the Quorum by evaluating the decision rule itself and adopting the decision. It then returns the decision to the client by sending $Phase2R \coloneqq \langle TxID, decision \rangle_r$. Importantly, a replica never changes its decision.

\fbox{\begin{minipage}{23em}
\textbf{(5: C)}: Client returns shard-decision to application and proceeds to Writeback
\end{minipage}}\\
A client waits for a Quorum of $n-f$ matching $Phase2R$ messages. Such a Quorum forms a \textit{Shard-Certificate} and proves the decision. A client uses this Certificate to return to its application and issue the Writeback.

\underline{Additional subtlelties}: If a client equivocated, it will never receive a Shard-Certificate. An honest client however, is guaranteed to receive matching Phase2 replies. 

We consider a decision (Commit, Abort) to be \textit{logged} when it is possible for some Shard-Certificate to exist, i.e. as soon as the necessary certificate Quorums exists at some Replicas.
Figure \ref{fig:FigureSP} summarizes the relevant nomenclature.

\begin{figure}
\begin{center}
\includegraphics[width= 0.5\textwidth]{./figures/Nom2.png}
\end{center}
\caption{Validation Nomenclature, Slow-Path. Note, that a byzantine client may equivocate Phase2 decisions by including Commit and Abort Quorums respectively. Byantine replicas may store multiple votes and decisions.}
\label{fig:FigureSP}
\end{figure}

\subsubsection{Correctness}
We show, that a \textit{logged} decision is final:
\begin{theorem}[Saf]
A logged decision is durable, and there can ever exist \textbf{at most one} logged decision.
\end{theorem}
\begin{proof}
See TR.
\end{proof} 

Note, that since replicas never change their decision, it is possible for there to never be any logged decision if a byzantine client equivocated its Slow-Path Quorums. In order to reconcile this, we design and discuss a recovery mechanism in section X which relaxes the requirement on persisting a decision.  


\begin{theorem} 
Indicus maintains \textit{Byzantine-Serializability}.
\end{theorem}
To prove that this is the case, we show that for any two conflicting transactions, at most one can be committed.
\begin{proof}
See TR.
\end{proof}

\begin{theorem} 
Indicus maintains Byzantine Independence in the absence of network adversary.
\end{theorem}

We show, that once a Client submits a transaction for validation, the result cannot be unilaterally decided by any byzantine participant, be it client or replica.
\begin{proof}
See TR.
\end{proof}
%
\subsection{Validation Check}

Algorithm \ref{mvtso} shows the necessary validation check to preserve Byzantine-Serializability. 
Given a transactions prepare request the validation check returns an Abort vote if a conflict has been detected, and Commit otherwise. 
When execution results match the timestamp order, there are no conflicts (Fig. \ref{fig:MVTSOEX}: 4 top). For each read, a replica verifies that it has not voted to commit a conflicting write (Algorithm \ref{mvtso}, line 3-7). Conversely, for each write, a replica confirms that there exist no previously accepted reads (line 8-10), and no ongoing read transactions (line 11-12) that conflict. Fig. \ref{fig:MVTSOEX}: 4 shows both non-conflicting and conflicting interleavings.
If there are no conflicts, a replica tentatively \textit{prepares} a transaction, making its writes visible and evaluating future transactions against it for conflicts. Regardless of the outcome, a replica garbage collects all Read Timestmaps (RTS) associated with the transactions reads.
It then waits for necessary dependencies (uncommited writes that were read) to be resolved (Figure \ref{fig:MVTSOEX}: 5). We remark, that the concurrency control check is serialized and executed atomically for each transaction.

\begin{theorem}
The set of transactions for which the MVTSO-Check returns Commit is Byzantine-Serializable. 
\end{theorem}
\begin{proof}
See TR.
\end{proof}

\begin{algorithm}
\caption{MVTSO-Check(TX, TS)}\label{mvtso}
\begin{algorithmic}[1]
\If{\textit{$TS > localClock + \delta$}} %  || $TS < lowWM$ || $\exists d \in dep: d.TS < lowWM$} } dont mention garbage collection part here, it only confuses
\State \Return Abort
\EndIf

\For{\textit{$\forall key,version \in \textit{TX.RS}$}}
        \If{$ \exists TX2 \in Committed \cup Prepared: key \in \textit{TX2.WS} $ \newline
        \hspace*{2em} $\land \, version < \textit{TX2.TS} < TS$}  
          \State  \Return Abort, \textit{TX2, (TX2.CommitProof)}  
         \EndIf  
\EndFor

\For{\textit{$\forall key \in \textit{TX.WS}$}}
        \If{$\exists TX2 \in Committed \cup Prepared:$ \newline 
        \hspace*{2em} $\textit{TX2.RS[key].version} < TS < TX2.TS$} 
          \State  \Return Abort, \textit{TX2, (TX2.CommitProof)}
         
        \EndIf
        \If{$\exists RTS \in key.RTS: RTS > TS$} 
          \State  \Return Abort
       \EndIf
\EndFor
\State Prepared.add(TX) 

\While{$\exists d \in dep: d \notin CommitLog \cup AbortLog $)}
\State Suspend
\EndWhile

%structure it in a way that is better
\For{\textit{$\forall d \in dep$}}
		\If{$ d \in AbortLog $}
		\State	Prepare.remove(TX)
		\State \Return Abort, \textit{(TX2.AbortProof)}
		\EndIf
\EndFor
\State \Return Commit
\end{algorithmic}

\end{algorithm}

In order to perform the MVTSO-check, a replica maintains several data strucutres: \one It stores read timestamps, read versions alongside the multiversioned write stores for committed and tenative transactions respectively, to provide efficient evaluation of conflicts.
\two In order to confirm dependency outcomes, replicas log proofs for completed transactions in respective Commit and Abort Log sets, that together induce the ledger of all processed transactions. 
\three To avoid busy waiting when dependencies are not yet resolved, a replica temporarily suspends the MVTSO-check for the current transaction, allowing it to process other transactions pending validation. To facilitate this, it keeps track of an additonal transaction to dependents mapping that allows to identify and resume all suspended MVTSO-checks associated with dependents of a completing transaction.

\iffalse

\textit{Aside:} Consistent with our definition of Byzantine-Isolation, byzantine clients may issue ficticious Read-Sets comprised of arbitrary read versions and values. However, these have only limited external effect on concurrent writes. We distinguish two extreme cases: 1) $read.version \rightarrow 0$: This case is equivalent to simply reading stale data, and effectively reduces MVTSO to TSO as conlflicts are evaluated only on basis of the transaction timestamps (i.e. Abort write if: write.TS < read.TS). 2) $read.version \rightarrow read.TS$: In this case there are no conflicts as a write is never "missed" by a previous read.
\fi

In the following section we will show how to design a replicated validation scheme that upholds Isolation guarantees and reaches a single shard decision, even when replicas within a shard validate in different orders.

%%-------------------------------------------------------------------------------
\subsection{Consistent logging.}
%-------------------------------------------------------------------------------
Principles and challenges

protocol overview: pic

%%-------------------------------------------------------------------------------
\subsection{Writeback}
%-------------------------------------------------------------------------------

Validation occurs on every Shard that a transaction spans. The goal of the Writeback phase is to aggregate all relevant shard-decisions and to inform replicas of finalized Commit or Abort decisions. This is necessary in order for replicas to be able to garbage collect meta-data of ongoing transactions, and to allow consecutive transactions to reliably observe the updated state. Any finalized decision must respect all shard-decisions relevant to a transaction. Thus, all decisions are aggregated according to standard two-phase commit. \sys follows a simple 2PC protocol: Only if all shards agree that a transaction may commit (i.e. there exist commit certificates for every shard), then a transaction may commit. A single abort shart-decision suffices to abort a transaction respectively.

\textbf{1.} A coordinator waits for all necessary shard decisions, including certificates\\
\textbf{2.} The coordinator broadcasts a $Writeback \coloneqq (TX, decision, \{certificates_S \} )$ message to all replicas in all relevant shards.\\
\textbf{3.} A replica validates whether the certificates match the decision, and commits/aborts the transaction by applying the relevant store updates, resuming potentially suspended MVTSO-Checks, and garbage collecting all ongoing transaction state. Replicas store the Writeback message which is used as Commit proof to service reads and justify conflicts, or Abort proof to justify dependency aborts.

We point out, that the Writeback coordinator need not be the client issuing the transaction, but can in fact be an arbitrary party (client or replica) that is interested in completing the Writeback. This follows straightforwardly from Theorem Y: Any certified shard-decision implies the existence of a logged decision, and hence the Writeback phase is idempotent.
We utilize this to drive the recovery protocol outlined next. 


%\subsection{Multi-sharding}

and Multi-shard 2pc

- Optimization: Single shard logging

%%-------------------------------------------------------------------------------
\subsection{Failures}
%-------------------------------------------------------------------------------
- Fallback: election (only starts if not waiting on another dep to avoid early eviction), views, resolution, subtelties with mvtso (block because of dep), necessity even without dependencies. Interested clients, write-back multishard. garbage collection
- Fallback requires an extra round in order to learn about current views to start viewchange, but thats ok: Its co-function with learning about full TX, and checking for existing certificates. Timeout invocation is concurrent with p1 message.

%%-------------------------------------------------------------------------------
\subsection{Low Cost mode}
%-------------------------------------------------------------------------------
3f+1 if not defending against byz colluders
OCC if not worried about reads aborting


%-------------------------------------------------------------------------------
\section{Protocol}
%-------------------------------------------------------------------------------
Indicus comes in two different flavors, Indicus3 and Indicus5 respectively, relying on varying replication degrees. Indicus3 requires 3f+1 replicas per shard to guarantee consistency, the minimum bound necessary for BFT SMR. Indicus5 uses a higher replication degree of 5f+1 replicas, but in return brings down both gracious execution latency and complexity during uncivil executions. We argue, that unlike past system settings where a single authority strives to maintain the fewest amount of replicas necessary for cost considerations, consortium systems with naturally higher replication degree are willing to pay the additional price for performance. Moreover, rather than trying to reach the threshold of replicas to tolerate some number of faults, these settings may start with a fixed number of replicas and consider the ratio of faults to be tolerable. When argueing from this perspective, the perceived difference between tolerating 1/3 and 1/5 of replica failures may be negligible.
For the simplicity of exposition we outline Indicus5 in detail below. In section X we describe the differences in Indicus3.


%%%%%%%%%%%%%%%%%%%%%%%%%%%%%Figs
\begin{figure}[t]
  \begin{mdframed}[roundcorner=10pt]
 	\textbf{TX Exec State}
 	\begin{itemize}
 	\item ClientID
 	\item ClientSeqNo
 	\item InvolvedShards = \{S\}
 	\item ReadSet = \{(key, version)\}
 	\item WriteSet = \{(key, value)\}
 	\item dependencies = \{$\langle (key, version, \{TxID'\})_{f+1 \sigma} \rangle$\}
 	\item TxID = H(TX)
 	\item Timestamp = (Time, ClientID)  optional:, TxID) this is just deterministic tie breaker when a client misbehaves.
 	\end{itemize}
  \end{mdframed}
  \caption{Transaction Execution state}
  \label{fig:TX}
\end{figure}

\newcommand{\SubItem}[1]{
    {\setlength\itemindent{15pt} \item[-] #1}
}

\begin{figure}[t]
  \begin{mdframed}[roundcorner=10pt]
 	\textbf{Replica }
 	\begin{itemize}
 	\item ReplicaID
 	\item ShardNo
 	\item LocalClock
 	\item RTS = \{(key, \{(TS, \{dep\})\})\}
 	\item Ongoing:
 	\subitem OngoingTX = \{(TxID, [TX, state])\}
 	\subitem PreparedDB = \{(key, [writes:\{(val, version, TxID)\},  reads: \{TS, rversion, TxID\}])\}
 	\subitem WaitingDeps = \{(TxID, dependents\}
 	\subitem Deps = \{(TxID, dependencies\}
 	\item Database = \{(key, [w: \{(val, ver)\}, r: \{(TS, rv)\}])\}
 	\subitem CommitLog = \{(TxID, CommitCert)\}
 	\subitem AbortLog = \{(TxID, AbortCert)\}
	 	
 	
 	\end{itemize}
  \end{mdframed}
  \caption{Replica State and Datastructures}
  \label{fig:RS}
\end{figure}

NEED FIg with replica state. Need to use it to explain what "Prepared"/Committable means
%-------------------------------------------------------------------------------
\subsection{Execution}
%-------------------------------------------------------------------------------
Clients in Indicus both execute and submit their own Transactions. As previously defined a Transaction TX is a sequence of read and write requests that is ultimately terminated by a Commit or Abort decision. A TX object, as shown in Figure \ref{fig:TX} records the execution state necessary for Validation. The execution protocol has three goals: 1) Honest Clients should read valid data, i.e. experience read integrity, 2) Honest Clients should read fresh data, i.e. minimize staleness and hence maximize commit chance, and 3) avoid expensive coordination as much as possible. This is complicated by the presence of byzantine replicas as well as concurrent transactions. Byzantine replicas may provide invalid or arbitrarily stale data and honest replicas may be temporarily out of sync.

We avoid invalid reads by requiring replicas to provide a proof of validity, i.e. a Quorum of Writeback signatures confirming the committment, or alternatively, trusting only $f+1$ matching replies from discrete replicas. In order to minimize coordination only read requests incur a network rountrip, while writes are buffered locally until Validation. 
Since together Validation and Writeback together incur several Wide Area Network (WAN) message delays, a committing transaction is invisible to concurrent transactions for the time-being, yet results in isolation conflicts that need be resolved. In order to minimize this window, we allow transactions to speculatively read proposed, yet uncommitted values. Similarly, since Execution can span multiple read requests that require WAN message delays, there remains a potentially large period in which reads are vulnerable to conflict. To mitigate aborts due to write conflicts, we a) let reads be sequenced at a pre-defined timestamp (rather than always reading the most recent) and b) let reads acquire implicit read-locks that disallow conflicting writes. 

\fs{lead over to MVTSO - what I just described are the three base techniques}



Client execution conducts as follows:

\begin{enumerate}
\item \textbf{Begin} A client begins a Transaction by optimistically choosing a timestamp $TS \coloneqq (Time, Client ID)$. 
\item \textbf{Write(key, value)}. A Client executes a request Write(key, value) by locally buffering (key, value) and returning. Concretely: $WriteSet = WriteSet \cup (key, value)$

%%%%%%%%%Read protocol%%%%%%%%%%%

\item \textbf{Read(key, TS, RQS)} 
\fbox{Client does blabla}
\fbox{\begin{minipage}{21em}
\textbf{1: C} $\rightarrow$ \textbf{R}: Client sends read request to Replicas
\end{minipage}}

Given hyperparameter Read Quorum Size (RQS), a Client performs a read on given key at timestamp TS by reading from RQS different replicas. To do so, a Client sends $\langle(CID, key, TS)\rangle_{\sigma_c}$ \fs{technically CID is part of TS} to the respective replicas.\\

\fbox{\begin{minipage}{21em}
\textbf{2: R} $\rightarrow$ \textbf{C}: Replica processes Client read and returns reply
\end{minipage}}

A replica returns a signed pair \text{$\langle \textit{(Committed, Prepared)} \rangle _{\sigma_r}$}, where Committed is the write with largest committed write version prior to the specified Timestamp and Prepared is the respective largest uncommitted write that might be committed and would make Committed outdated. 

\underline{Additional subtlelties}: If the timestamp is beyond a Highwater mark ($HW = localClock + \delta$) replicas ignore the requests. If the request is serviced, a Replica adds the timestamp to its Read Timestamp Set (RTSS): $RTSS(key) += (TS)$. \text{$Committed \coloneqq (value, version)_{cert}$} such that $ (value, version) \in R.CommitLog$, $version = max(q) : (key, val, q) \in R.CommitLog \land v < TS \}$ and $cert$ is a certificate proving (value, version) was legally committed (see Section Writeback) and 
 $Prepared \coloneqq (value, version, dep)$ such that $(value, version) \in R.PreparedSet$, $version = max(q) : (key, val, q) \in R.PreparedSet \land Committed < v < TS \}$ and $dep$ is a DAG of Transaction IDs, starting from the TX that wrote $(value, version)$ and extending to its own dependencies.
\fs{a replica does not return a prepared val, if that is still waiting on its own dependencies. Want to avoid chains. In this case one does not need a DAG, because its only direct deps}

A replica only does this, if the client has access control for the value. (limits ability of byz clients to claim dependencies)


\fbox{\begin{minipage}{21em}
\textbf{3: C} ($\rightarrow$ \textbf{R}): Client receives read replies and asynchronously dissipates dependencies
\end{minipage}}

A client waits for RQS read replies and validates the integrity of any $Committed$ it receives.  It adds the biggest $Committed$ or $Prepared$ (iff enough matching) version seen to its Read Set, and claims a dependency if it was a $Prepared$ version. If it claims a dependency, it asynchronously forwards this dependency to all replicas. 
\{send dependency set, does not need to be proven.\}


\underline{Additional subtlelties}: If $RQS > n-f$ a Client waits only up to a application set timeout beyond the first $n-f$ received replies. If $f+1$ matching $Committed$ are received they need to be validated. If $f+1$ matching $Pepared$ are received from disjoint replicas, and
 $Prepared.version > max(\{Committed.version\}, \{\langle Prepared.versions \rangle_{f+1 \sigma_r}\})$ the client adds Prepared to its Read Set and claims a dependency by including the $f+1$ signatures: $ReadSet += (key, Prepared.version)$ and $dependencies += \langle Prepared.dep \rangle_{f+1 \sigma_r}$. 

\fs{does the value need to be included?} Further, a Client sends $\langle (key, TS, dep)\rangle_{\sigma_c}$ to all replicas. If there exists no such Prepared, it adds the largest valid Committed, i.e. $ReadSet += (key, max(Committed.version)$. 



\fbox{\begin{minipage}{21em}
\textbf{(4: R)} : Replica receives dependencies and adds exceptions
\end{minipage}}
Upon reception of $\langle (key, TS, dep)\rangle_{\sigma_c}$ a replica adds an exception to its Read Timestamp Set. Concretely, $RTSS(key)(TS) += dep$.

\underline{Additional subtlelties}: the dep set received does not need to have proofs, since a client is only trying to gain additional protection by allowing dependencies to be committed. If this set is "fake", it will only cause the own read to abort later, because more writes than desired were allowed to pass. "It just weakens the power of the read lock", which is not relevant for safety, but only for commit rate.
(does one need a check in MVTSO that deps are correct? If max depth =1, then one does not need to send deps anyways, because byz can always lie; or is it still necessary to finish with fallback)
A: byz client only includes dependencies to avoid direct fallback eviction. Only by including proof for deps can it be validated, that these deps are real and can be traced in order to induce a fallback on them)



\item \textbf{Commit} A Client finalizes its execution, computes the final $TxID \coloneqq H(TX)$  and submits the Transaction for validation.

\item \textbf{Abort} A client terminates execution, broadcasts a read-release for all potentially acquired Read Timestamps (RTS) and returns.
\fs{A byz client may not release the RTS. To circumvent this, RTS are just leases. Moreover, at some point writes will overtake them}

\end{enumerate}

\fs{Need to talk about MVTSO here first}
 
We briefly discuss some implications of the choice of Read Quorum Size (RQS) as well as the general MVTSO design.
 
A client may choose to read from any number of replicas. Following cases may be distinguished:
\begin{itemize}
\item \textbf{$RQS = 1$} A replica may read from just 1 replica at the risk of later aborting due to reading maliciously stale data (i.e. replica claiming no write ever existed). Note, that an honest client is still guaranteed to read valid data, as only Committed reads are processed. If a Client trusts a local replica \textbf{and} the replica is not lagging behind, this is a viable option to reduce execution time and hence improve both latency and conflict likelihood.
\item \textbf{$RQS \geq f+1$} When reading from $f+1$ or more replicas a client is not susceptible to maliciously stale reads, yet it is still possible to read (arbitrarily) stale data, either due to inconsistency caused by asynchrony or a byzantine client not fully replicating its transaction. Furthermore, it is always possible to miss a concurrent TX in the pipeline.
\item \textbf{$RQS \geq \frac{n+f+1}{2}$} When reading from $3f+1$ (in Indicus5, or $2f+1$ respectively in Indicus3), a client guarantees, that no more \textbf{additional} conflicting writes can be admitted, since such a Quorum acts as a read-lock. If a single Prepared write would suffice to form a dependency, then it would be guaranteed, that a read cannot abort, since the freshest possible value would have been read. This however, overly empowers byzantine replicas:  Depending on such transactions is undesirable, as there exists neither confidence that this transaction does not violate Isolation, nor would \textit{Byzantine Independence} be upheld as byzantine replicas could reactively fabricate Transactions. Thus, 

- also has the effect of not needing to store the full TX always, because one honest is guaranteed to have it and we can get it later.
- Byz clients could fabricate their own dependencies 
\end{itemize}

1. Reading from 1 Replica: could abort because reading arbitrarily stale.
Alternatively: Read from f+1 always to reduce proof/memory costs: Could not get matching and hence fail read. Still possible to abort by reading stale data 
2. Reading f+1 matching Prepared that are larger than the commit --> choose that and add dependency (in 5f+1 might only want to do it if read 2f+1, not necessary for safety, but higher commit chance?)
Guaranteed to see 1 thats not intentionally stale. But still arbitrarily stale if unlucky.
3. If read from 3f+1/2f+1 then its effectively a read lock. 
If 1 prepared read was enough then this would be sufficient to see the newest "potential" value. However that is not possible for liveness reasons.
Note, could still have missed a prepared write and have to abort because we required f+1 matching. Why f+1? so it comes from 1 honest: we dont need proofs, and we know it is in the system for recovery.
Still no guarantee that newest commit was seen, but some protection from other writes.
4. Exceptions: granting exceptions is protection for oneself: minimize the window of dependent writes aborting against oneself. \fs{Should not rely on deps beyond depth 1, because those could be both byz and not claim exceptions and abort themselves always}

(Note: the TS used here is not the final one, its just the tuple of time and clientId. (The triple later is just necessary to differentiate two TX by a client that were assigned the same Time
Add Read set decision rule, put in correlation to RQS: \fs{could only accept reads if f+1 matching always, then no proofs necessary, but then reads can fail}
Add exceptions etc.
)



- Read proofs required for honest client correctness. Alternatively one can read from f+1 only, but that can result in failed/older reads
- Notice, that a byz client does not follow this protocol. It can do whatever reads it wants, but it cannot claim non-existant dependencies. Do not need to include read proofs in the prepare. According to Isolation definition byzantine Clients can read whatever they want. Moreover, Reads only have very limited external effect. The value does not matter for the CC check. The version has bounded effect: If it goes towards 0, then it is just a check between timestamps as normally. If it goes towards the TS, then it will never abort.
- Fallback not just useful for dependency cleanup, but also "unclaimable" concurrent TX that need to finish.

\fs{
1. read access control not possible since 1 byz can return write. This requires Multi party computation, shared secrets (verifiable secret sharing), beyond the scope of our work (reference soumyas paper)
2. need to be clearer on the byz independence violation when single read deps are allowed. Byz clients could fabricate their own dependencies 
}
THEOREM:
Our design accomplishes Byzantine Independence in the absence of an adverserial network.
%-------------------------------------------------------------------------------
\subsection{Validation}
%-------------------------------------------------------------------------------
The goals of the Validation design are threefold: i) It needs to preserve Isolation guarantees between transactions, ii) It should embrace partial ordering, while minimizing commit latency and  maximizing scalability, and iii) it should be \textit{leaderless}. Satisfying these goals requires ovecoming several challenges. In order to maximize parallelism and embrace partial ordering, Indicus allows replicas to process requests out of order. Consequently, replicas may temporarily be out of sync, and hence return different results. Such divergence must be reconciled in a way that maintains Isolation, but not overly conservatively in order to maximize the ability to commit successfully and bound the impact of byzantine participants. Further, Indicus designates clients as validation coordinator for its own transactions. The respective protocol must tolerate client failures such as crashes, omission/stalling, equivocation, or replays and allow for consistent recovery. 

The Validation protocol can be broken down into two functionalities: Voting and Logging.
Since Indicus aims to prescribe a partial order, Replicas need to agree on results, rather than an Ordering. To do so in a leaderless fashion, they vote. As in any democratic decision, these votes are then aggregated into a decision. In the case of Indicus, the client acts as the transaction coordinator who aggregates and relays results (along with necessary evidence).
 In order to maintain consistency in the presence of failures, this decision must be logged prior to returning, guaranteeing that a replay of any Writeback is idempotent. \fs{This two step decision protocol resembles Two-Phase Commit.} However, since we cannot trust a byzantine coordinator to persist a decision, nor could we retain liveness during a crash, we delegate the decision logging to the replicas.

The voting phase requires a single round-trip to all replicas, whereas the logging phase requires at most one round-trip \fs{in Indicus5 and at most two round-trips in Indicus3.}. When execution is \textit{gracious} Transactions can be committed on the \textit{Fast-Path} in a single-round trip as an explicit logging round is not necessary.
To describe how the protocol operates in detail we follow a single-shard transaction through the system:

\fbox{\begin{minipage}{21em}
\textbf{(1: C $\rightarrow$ R)}: Client sends Prepare request to all Replicas within the Shard.
\end{minipage}}

Upon deciding to Commit in the Execution phase, a Client initiates Validation by sending a message $Phase1 \coloneqq \langle Prepare, TX, (TS) \rangle_{\sigma_c}$ to all Replicas.

\underline{Additional subtlelties}: A client may not equivocate this request. Any change to the TX state changes the TXiD and hence is treated as seperate TX. 

\fbox{\begin{minipage}{21em}
\textbf{(2: R $\rightarrow$ C)}: Replica receives validation request, processes it and returns vote to Client.
\end{minipage}}
A replica validates Timestamp and Dependency integrity of the request. It then evaluates Read and Write Sets for Isolation conflicts using the MVTSO Concurrency Control Check (CCC), as shown in algorithm 1. This concurrency check is entirely based on local knowledge of previously committed, and concurrently ongoing transactions. If successful, a replica updates its set of potenially committable transactions (Prepared) and returns a Commit Vote to the client. If unsuccessful, a replica returns an Abort/Abstain Vote to the client, along with evidence justifying the decision. It sends a message $Phase1R \coloneqq (\langle TxID, result \rangle_r, optional: TX, optional: \{proof\})$.



\underline{Additional subtlelties}: A replica validates whether all entries in the dependency set are legal, i.e. whether f+1 matching signatures exist. If this is not the case, a replica rejects the request and submits a Proof of Misbheavior (PoM) to expel the client from the system. A byzantine client may neither match read set and dependency set, nor match read set to the involved shards. This is tolerated, as a byzantine client chooses whether to experience Isolation and Atomicity. Illegal reads, or un-committed writes do not have external effect, but are auditable in the committed ledger to trace misbehavior.
\fs{ In effect, any claimed deps implies a client did those reads, but he does not necessarily have to include them} 
A replica updates its $OngoingTX$ set by adding a pair $(TxID, state)$, where $state$ is an object containing relevant protocol metadata, such as the Client Phase1 request or the MVTSO Check decision (to service replays and avoid re-execution). Importantly, a replica never changes its Voting decision, because re-execution could leave to different results. In section X we discuss an optimization to allow this behavior. If voting Abort, a replica returns a CommitCertificate for the Transaction causing the conflict. If voting Abstain, a replica returns the signed Phase1 request of the conflicting Transaction.
%%%%%%%%%%%%%
Access control is beyond the scope of this work, but a replica would additionally check for all writes whether access control exists and reject the transaction otherwise.


\fbox{\begin{minipage}{21em}
\textbf{(3: C)}: Client waits for vote replies.
\end{minipage}}
A client waits for at least $n-f$ ($4f+1$ in Indicus5) distinct replica votes, or more, up to a system specified timeout. 

\fbox{\begin{minipage}{21em}
\textbf{(3a: C)}: Client receives Threshold of matching votes and returns to application. Proceeds to Writeback
\end{minipage}}
In any of the following 3 cases, a client may short-circuit waiting for additional votes and omit a dedicated Logging round:
\begin{enumerate}
\item \textbf{$1$ Abort vote w/ Conflicting TX \& CommitCertificate}: The client validates the integrity of the CommitCertificate (CC) and returns the shard decision $(TxID, Abort, \langle ConflTX \rangle_{CC})$.
\item \textbf{$3f+1$ Abstain votes w/ Conflicting TX}: The client returns the shard decision $(TxID, Abort, \{\langle AbstainVote\rangle_r\})$. 
\underline{Additional subtlelties}: The client temporarily stores the conflicting TX Phase1 requests, in order to be able to aid termination if necessary (see Section X).
\item \textbf{$5f+1$ Commit votes}: The client returns the shard decision $(TxID, Commit, \{\langle CommitVote \rangle_r\}$
\end{enumerate}
Any of such Quorums forms a \textit{Shard-Certificate} and proves the decision. A client uses this Certificate to return to its application and issue the Writeback.
\fs{need to distinguish with Indicus3: can still abort FP, but not commit.}

\fbox{\begin{minipage}{21em}
\textbf{(3b: C $\rightarrow$ R)}: Client receives inconsistent results and sends decision to Replicas for Logging
\end{minipage}}
If a client does not receive the necessary thresholds of votes to return, it must continue on the \textit{Slow-Path}. To do so, it aggregates the votes according to a conservative decison rule:
If there exists a $CommitQuorum \coloneqq \frac{n+f+1}{2}$ ($3f+1$/$2f+1$ in Indicus5 and Indicus3 respectively) of Commit Votes, the Slow-Path decision is Commit, otherwise it is Abort.
A client broadcasts a message $Phase2 \coloneqq (TxID, decision, \{\langle votes \rangle_r\}$.

\underline{Additional subtlelties}: A client forwards a Quorum of $\geq n-f$ votes to the replicas. This is necessary to prove the Slow-Path decision is consistent with Isolation guarantees. Effectively, replicas make the decision for themselves. Note, that a byzantine client may equivocate the decision by relaying different Quorums.

\fbox{\begin{minipage}{21em}
\textbf{(4: R $\rightarrow$ C)}: Replicas receive, validate and echo decision
\end{minipage}}
A replica confirms that the Decision matches the Quorum, by evaluating the decision rule itself. It then returns the decision to the client by sending $Phase2R \coloneqq \langle TxID, decision \rangle_r$. \fs{with retries this needs to include the timestamp}

\fbox{\begin{minipage}{21em}
\textbf{(5: C)}: Client returns shard-decision to application and proceeds to Writeback
\end{minipage}}
A client waits for a Quorum of $n-f$ matching Phase2 Replies. Such a Quorum forms a \textit{Shard-Certificate} and proves the decision. A client uses this Certificate to return to its application and issue the Writeback.

\underline{Additional subtlelties}: If a client equivocated, it will never receive a Shard-Certificate. This will never happen to an honest client.

%%%Optional for Indicus3
\iffalse
\fbox{\begin{minipage}{21em}
\textbf{(5a: C)}: Client returns Commit to application
\end{minipage}}

\fbox{\begin{minipage}{21em}
\textbf{(5b: C $\rightarrow$ R)}: Client sends consistent echo to Replicas
\end{minipage}}

\fbox{\begin{minipage}{21em}
\textbf{(6 : R $\rightarrow$ C)}: Replicas echo 
\end{minipage}}
\fi

\fs{should perhaps come after Writeback (which is brief) in order to frame the proofs. But then we need to show Atomicity also. Durability. Conistency. }
We now briefly allude to the correctness of respective decision rules and Quorum sizes.



We consider a decision (Commit, Abort) to be \textit{logged} when it is possible for some Shard-Certificate to exist, i.e. as soon as the necessary certificate Quorums exists at some Replicas. We show, that a \textit{logged} decision is final:
\begin{theorem} 
A logged decision is persistant, and exists at most one logged decision.
\end{theorem}
\underline{Proof sketch:} We show this by case distinction. Slow-Path: A Slow-Path decision is \textit{logged} if $\frac{n+f+1}{2} = 3f+1$ ($LoggedQuorum$) honest replicas have adopted the decision, since $n-f = 4f+1$ votes suffice to form a Shard-Certificate and $f$ byzantine participants may decide arbitrarily. Thus it is impossible for two logged decisions to co-exist, as any two $LoggedQuorums$ must intersect in $f+1$ honest replicas. Furthermore, honest replicas do not change their decision, and hence a slow-path logged decision persists. Fast-Path: We distinguish three sub-cases. The existance of $4f+1$ Phase1 honest commit votes, implies that any Slow-Path decision must result in a commit decision, since any Quorum ($n-f = 4f+1$) is bound to include $3f+1$ commit votes. Vice versa, the existance of $2f+1$ honest Phase1 abstain votes, implies the impossibility of any Slow-Path commit decision. Moreover, both the above cases mutually exclude each other. Lastly, 1 valid Abort vote implies the existance of a logged decision for the conflicting Transaction. By Induction \fs{that TXs logged decision never changes}, and Quorum intersection, this implies that at least $3f+1$ honest replicas will vote to Abstain the ongoing transaction and hence it is impossible to ever log a commit decision for the ongoing Transaction.

Note, that since replicas never change their decision, it is possible for there to never be any logged decision if a byzantine client equivocated its Slow-Path Quorums. In order to reconcile this, we design and discuss a recovery mechanism in section X which relaxes the requirement on persisting a decision.  


\begin{theorem} 
Indicus maintains \textit{Byzantine-Serializability}.
\end{theorem}
To prove that this is the case, we show that for any two conflicting transactions, at most one can be committed.
\underline{Proof sketch:} Let TX1 be a transaction with logged decision Commit \fs{, i.e. TX1 has either already committed at or is bound to commit at all honest replicas}. Let TX2 be a conflicting transaction, that if committed, would violate Byzantine-Serializability. Assume TX2 too, managed to log a commit decision. By the protocol (and proof of Theorem Y -the above theorem), at least  $\frac{n+f+1}{2} = 3f+1$ commit votes are required to log a decision, and no honest replica changes its vote. By Quorum intersection, at least one honest replica must have voted commit for both TX1 and TX2. WLOG, this replica received TX1 before TX2, and, by the correctness of the MVTSO-check, must have voted Abort or Abstain for TX2. A contradiction.





\begin{theorem} 
Indicus maintains Byzantine Independence in the absence of network adversary.
\end{theorem}
\underline{Proof sketch:}
We show, that once a Client submits a transaction for validation, the result cannot be unilaterally decided by any byzantine participant, be it client or replica.
First, we observe that a client may never choose a result itself, but only implicitly influence a decision by choice of Slow-Path Quorum. Specifically, a byzantine client cannot single-handedly decide to abort its transaction.
Second, any Quorum decision requires at least one honest replicas vote. In particular, a $f$ byzantine replicas may vote to abstain arbitrarily (by always reactively generating a ficticious transaction), but at least one additional honest abort vote is necessary to result in an abort (at least $f+1$ additional honest votes if the network is synchronous with regards to the timeout $\delta$). Thus, in order to artificially cause transactions to abort, a conflicting transaction must be generated (artificial congestion). However, to do so strategically and reliably, the adversary must control the network in order to guarantee the artifical transactions arrival at honest replicas \textit{before} the original transaction. Thus, when the network is not adversarial, validation decisions are \textit{Byzantine Independent}.

We note, that Indicus3 does not have this property: Byzantine Clients can always abort themselves, which is undesirable when wanting to allow dependencies.


%-------------------------------------------------------------------------------
\subsection{Concurrency Control}
%-------------------------------------------------------------------------------
\begin{algorithm}
\caption{MVTSO-Check(TX, TS)}\label{euclid}
\begin{algorithmic}[1]
\If{\textit{$TS > localClock + \delta$} }
\State pass
\EndIf

\For{\textit{$\forall key,version \in \textit{TX.read-set}$}}
        \If{$ \exists TX2 \in CommitLog: key \in \textit{TX2.write-set} \land version < \textit{TX2.TS} < TS$}  
          \State  \Return ABORT, \textit{TX2, TX2.CommitCert}
        \EndIf
         \If{$ \exists TX2 \in Prepared: key \in \textit{TX2.write-set} \land version < \textit{TX2.TS} < TS$}   
          \State  \Return ABSTAIN, \textit{TX2}
        \EndIf
          
   %     \If{$ \textit{dep[key]} == null \land version \notin prepared-reads[key] \cup  CommitLog \cup AbortLog $}   
    %      \State  \Return $\textit{Proof-of-Misbehavior}$
    %    \EndIf
		   
        
        
\EndFor

\For{\textit{$\forall key \in \textit{TX.write-set}$}}
        \If{$\exists TX2 \in Prepared/CommitLog \land TX \notin TX2.dep: \textit{TX2.read-set[key].version} < TS < TX2.TS$}
          \State  \Return RETRY, %$\textit{TS_{new} | \forall TS_{new} > TX2.TS}$ // 
        \EndIf
        \If{$\exists RTS \in key.RTS: RTS > TS \land TX \notin RTS.dep$}
          \State  \Return RETRY, %$\textit{TS_{new} | \forall TS_{new} > TX2.TS}$ // 
        \EndIf

\EndFor
\State Prepared.add(TX) 
\While{$\exists d \in dep: d \notin CommitLog \cup AbortLog $)}
\State Wait
\EndWhile

\For{\textit{$\forall d \in dep$}}
        \If{$ d \in CommitLog $}
        	\If{$d.TS > TS$}
        	\State \Return ABORT, d.CommitCert
          	\EndIf
       
		\Else 
		\State \Return ABORT, d.AbortCert
		\EndIf
\EndFor


\State \Return COMMIT


\end{algorithmic}
\end{algorithm}

Datastructures..

Replica updates them how..

Ties broken between identical timestmaps.
A replica updates its internal state.. Ongoing TX set, prepared set. Checks whether dependencies are resolved already, otherwise puts that thread to sleep. 
\fs{edit Replica state to include ONGOING set: has p1, and p2 info, view, dependencies, proofs, etc., has field that says prepared or not - remove Prepared set.)}
A client may however choose identical timestamps for two different transactions, these ties are broken by TXID order.


 - Optimization: retries - heights
 - Dependency resolution tree
 		- Equivocation not possible if TXidentifier a function with dep as argument
 		- Cannot claim dep if not f+1 times (If you want to, you would require proofs again, which we try to avoid because dep trees can grow exponentially). More reads also more likely to commit --> f+1 guarantees 1 honest thinks it is legit.
 - Exception for depedency and early async read response to inform of exceptions
 - Read leases instead of unlimited locks - in practice only grant to timely clients (not a safety measure, but an increased progress guarantee)
 - 
%-------------------------------------------------------------------------------
\subsection{Consistent logging.}
%-------------------------------------------------------------------------------
\begin{figure}
\begin{center}
\includegraphics[width= 0.4\textwidth]{./figures/AB.png}
\end{center}
\caption{Atomic Broadcast}
\label{fig:Figure1}
\end{figure}


\begin{figure}
\begin{center}
\includegraphics[width= 0.4\textwidth]{./figures/5f+1.png}
\end{center}
\caption{Logging}
\label{fig:Figure1}
\end{figure}

Principles and challenges

protocol overview: pic


%-------------------------------------------------------------------------------
\subsection{Writeback and Multi-shard 2pc}
%-------------------------------------------------------------------------------
Validation occurs on every Shard that a transaction spans. A byzantine client may choose to mismatch.

Client puts in the TX which shards are involved, if that does not match the read/write set, then its not atomic. A shard will not accept reads/writes for which there exist no shard proof.
Client waits for all shard results. Then issues Writeback.
\fbox{\begin{minipage}{21em}
\textbf{(1: C $\rightarrow$ R)}: Client sends Commit/Abort decision to the Replicas.
\end{minipage}}
%-------------------------------------------------------------------------------
\subsection{Failures}
%-------------------------------------------------------------------------------
- Fallback: election (only starts if not waiting on another dep to avoid early eviction), views, resolution, subtelties with mvtso (block because of dep), necessity even without dependencies. Interested clients, write-back multishard. garbage collection
- Fallback requires an extra round in order to learn about current views to start viewchange, but thats ok: Its co-function with learning about full TX, and checking for existing certificates. Timeout invocation is concurrent with p1 message.
\fs{think carefully about how to write Fallback down. Decision rules for view consistency vs all to all. In 5f+1 vs 3f+1}

\fs{why not elect a client? could be byzantine, unbounded amount --> no guarantee that we ever finish. Replicas are bounded.}
%% Gathering info
\fbox{\begin{minipage}{21em}
\textbf{(1: C $\rightarrow$ R)}: Client sends Fallback invocation.
Asks for state, views, acts as p1 if not received yet. OR ASKS for p1 if havent received yet... 
\end{minipage}}

\fbox{\begin{minipage}{21em}
\textbf{(1: R $\rightarrow$ C)}:  process p1 if they have not yet, and reply (can go fp right here to Writeback if everybody cooperates. Needs to do this on every shard for p1, only on 1 shard for p2)
the above is WRONG. Correct: they just reply with p2 state, views, and full TX (if they have)

Replicas mark Client as interested
\end{minipage}}


%%Fallback start
\fbox{\begin{minipage}{21em}
\textbf{(1: C $\rightarrow$ R)}:  sends Fallback elect msg for view x (uses view update rules as described) (maybe only p1 here)
\end{minipage}}

\fbox{\begin{minipage}{21em}
\textbf{(1: R $\rightarrow$ R)}: Replicas send elect msg to fallback if timeout expires, blacklist old client (implicitly by increasing view), send p1 to client  (potential fastpath)


addional pedantics:  replicas send all to all to start election. First (few)view change happens without this. after that always all to all, to avoid timeouts growing exponentially because byz clients keep trying to view change without true election having happened.

\fs{election only works if the logging shard can elect solo. Any client should be able to re-do the p1, if no p2 exists yet (only after the timeout ofc). }
\end{minipage}}

\fbox{\begin{minipage}{21em}
\textbf{(1: R(FB) $\rightarrow$ R)}:  sends p2 decision
\end{minipage}}

\fbox{\begin{minipage}{21em}
\textbf{(1: R $\rightarrow$ C, FB)}:  echo decision to interested client, and back to fallback. 
\end{minipage}}

%%optional 
\fbox{\begin{minipage}{21em}
\textbf{(1: FB $\rightarrow$ R)}:  FB sends p3 as final certificates. Can be writeback if only single shard involved, or if replicas know about all other shards.
\end{minipage}}

So in total if it goes badly: 1 rr voting, 3 rr FB, 1 rr logging.


View changes are per-TX. they dont affect liveness of all other TX. If a client is responsible for allowing up to X view changes it can be expelled for not being timely enough for the system.
Not just byz clients, but poor clients get excluded.
%-------------------------------------------------------------------------------
\subsection{Optimizations}
%-------------------------------------------------------------------------------
- Retries: How it works (just change timestamps, only accept if p2 doesnt exist already; fallback needs to resolve). Retries can be per client choice (have a flag that says whether fast path or retires available).
- single shard logging
\begin{figure*}
\begin{center}
\includegraphics[width= \textwidth]{./figures/SingleShard.png}
\end{center}
\caption{Single Shard Optimization}
\label{fig:Figure1}
\end{figure*}


- (Read Locks ; remove can be optimization for writes)
- OCC instead of mvtso structures if disallowing prepared writes to be visible. OCC if not worried about reads aborting

%-------------------------------------------------------------------------------
\subsection{Garbage Collection}
%-------------------------------------------------------------------------------
- Prepares get removed upon commit/abort. Eventually replica might need to become "interested" client.

%-------------------------------------------------------------------------------
\subsection{Indicus3}
%-------------------------------------------------------------------------------
3f+1 if not defending against byz colluders as much

- no fast path
- Commits in 2 rounds, Aborts in 3 (Alternatively symmetric version)
- fallback quorums and bounds
- proofs necessary for recovery
- recovery rules

\begin{figure}
\begin{center}
\includegraphics[width= 0.4\textwidth]{./figures/3f+1.png}
\end{center}
\caption{Logging}
\label{fig:Figure1}
\end{figure}


