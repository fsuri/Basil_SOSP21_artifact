
%-------------------------------------------------------------------------------
%\section{Protocol}
%-------------------------------------------------------------------------------
\begin{figure*}[!th]
\begin{center}
\includegraphics[width= \textwidth]{./figures/Archi.png}
\end{center}
\caption{{\em Transaction Lifecycle}. Clients execute remote reads (1) and buffer writes (2). For Committment, all involved shards verify isolation (3). If there are conflicting transactions (TX'), replicas in a shard (B) vote to Abort. A client persists a decision (4) that serves as Two-Phase-Commit Vote for each shard (5), and Commits a transaction if all shards vote to commit (6).}
\label{fig:Figure1}
\end{figure*}
\fs{cut figure if it is not useful} \nc{I don't figure hurts, but I don't think it helps much either}\la{I would tentatively remove it: we are probably going to need the space}

\sys is is a sharded and replicated transactional key-value store designed to be scalable and leaderless, and our architecture reflects this ethos. 

\par \textbf{Transaction Execution} Transaction execution is driven by clients (removing costly all-to-all communications amongst replicas) and consists of three phases. First, in an \textit{execution phase}, clients execute individual transactional operations. As is standard in optimistic databases, reads are submitted to remote replicas while writes are buffered locally. \sys{} supports \textit{interactive} and cross-shard transactions: clients can issue new operations based on the results of past operations to any shard in the system. \sys{} must additionally ensure that these read operations do not violate Byzantine independence. In a second \textit{validation phase}, invidual shards in \sys{} must validate whether committing the transaction would violate serializability. For performance, \sys{} allows invidiual replicas within a shard to process requests out of order. \sys{} must additionally ensure that Byzantine actors cannot cause spurious aborts. Finally, \sys{} aggregates each shard decision in a \textit{commit phase} \fs{decision phase? must rename writeback section} to determine the outcome of the transaction, notifies both application and replicas in the system of the decision, and, if the decision was to commit, makes the buffered writes persistent in the data store.  Importantly, the decision of whether each transaction commits or aborts must be preserved across failures, reconfigurations \fs{sounds nice, but we never talk about reconfigs}, and Byzantine attacks. We describe each of these phases in turn in Section~\ref{section:exec}.

\par \textbf{Transaction Recovery} A Byzantine actor could begin executing a transaction, start the validation phase, but intentionally never reveal its decision. Without care,
such behavior would prevent the system from making progress, violating Byzantine independence \fs{I think this refers to independent operability? Byz independence talks about result outcomes}. To ensure progress, \sys{} thus implements a fallback recovery mechanism that can terminate stalled transactions while maintaining \fs{Byz-} serializability. We describe this mechanism in Section~\ref{section:rec}.




\nc{Florian's version in comments}
\iffalse
\sys is designed to be scalable and leaderless. Our architecture reflects this ethos. We briefly summarise it here before going into more detail in the later sections. 
In \sys, clients drive the entire transaction life cycle which can be broken down into three stages as shown in Figure \ref{fig:Figure1}: i) Execution, ii) Validation, and iii) Writeback. 
i) Clients \textit{speculatively execute} transactions themselves, invoking only remote read procedure calls (1) and buffering writes locally (2). \two Clients validate initiate a two-phase commit vote to validate speculative execution results for byzantine-serializability (3).
For every involved shard, a client queries potentially inconsistent replicas for their commitment vote, reconciles divergent votes into a single per-shard decision that maintains Isolation, and make this decision durable to avoid replay of contradictory decisions (4). 
iii) Lastly, clients aggregate all shard decisions (5) for atomic commit, return to the application and asynchronously, \textit{writes back} decisions and database updates (6).

Next, we outline the protocols for Execution, Validation and Writeback respectively. 
\fs{longer version in Architecture.tex }
\fi




%%%%%%%%%%%%%%%%%%%%%%%%%%%%%%%%%%%%%%%%%%%%%%%%%%%%%%%%%%%%%%%%%%%%%%%%%%%%%%%%%%%%%%%%%%%%%%%%%%%%
\iffalse
\fs{since this only affect validation, maybe move it there.}
\sys comes in two different flavors, \sys{}3 and \sys{}5 respectively, that rely on varying replication degrees, but implement the same design. \sys{}3 requires $n=3f+1$ replicas \fs{, the minimum bound necessary for BFT SMR?,} per shard to guarantee consistency in the presence of $\leq f$ byzantine replicas. \sys{}5 reduces both latencies during failure free execution and complexity during recovery. \footnote{We believe that consortiums with high performance requirements or high replication degrees are respectively comfortable with paying for additional replicas or tolerating a lower fraction (1/3 vs 1/5) of failures}
For the simplicity of exposition we discuss \sys{}5 for the remainder of the paper, and defer to section X \fs{and/or TR} to describe differences in \sys{}3. In the following, we outline \sys 's execution (unaffected by replication degree), validation and writeback protocols.

\fi



%%-------------------------------------------------------------------------------
\subsection{Execution}
%-------------------------------------------------------------------------------
The goals of \sys{}'s execution protocol are threefold: 1) to offer clients an interactive transaction interface, 2) to be scalable, and 3) to offer honest clients independent operability.

\sys{} achieves all three of these goals though the combination of optimistic client-side execution, and an aggressive, but byzantine resilient, concurrency control (CC) scheme. Rather than submitting stored procedures that must be ordered and executed by replica servers, \sys clients may execute arbitrary transactions by relying on RPC's to conduct transactions themselves. This places computational responsibility with the clients, and allows commutative transactions to execute in parallel. Since \sys clients execute optimistically concurrently, \sys must enforce CC in order to reconcile Isolation conflicts might arise.
By relying on optimistic rather than pessimistic CC schemes such as 2-phase-locking (2PL), \sys{} forgoes costly coordination to acquire locks, and sidesteps the concern of byzantine clients refusing to relinquish locks. 

Before we detail \sys{}'s transaction processing and validation mechanism, we first discuss the CC that \sys implements as the latter are functions of the precise CC requirements. 
 
\subsubsection{Concurrency Control}

%%% Explain MVTSO itself only briefly. Then explain how we make it for for byz
% 1) bound timestamps, 2) make writes visible late only, 3) only allow f+1 matchin uncommitted writes, 4) read timestamps
%Afterwards: explain execution interface. After that explain validation check. Then have overarching example
\iffalse
% we need to change a few things:
0) timestmap bound
1) Read validity
2) deferred writes
3) read uncommitted
4) byz read timestamps (read lock implies you had access control, at that point it is no different than issuing a tx and completing. But we do not want to give the power to abort for no reason. It should be traceable.
\fi

Figure \ref{fig:MVTSOEX} shows an example trace of \sys operational behaviors under different request processing orders (assume f=1) that we use to guide the remaining outline.

\begin{figure*}
\begin{center}
\includegraphics[width= \textwidth]{./figures/MVTSOLargeFont.png}
\end{center}
\caption{\emph{MVTSO behavior for different replica processing orders}. $r_x(C : a_y ,P : a_z)$ denotes that transaction $T_x$ (x being the timestamp of a transaction in this example) reads the version $y$ of object a written by committed transaction $T_y$ and version $z$ from tentative prepared transaction $T_z$. $P_x(RS,WS,DEP)$ denotes a transaction $T_x$'s prepare request (i.e. the validation check) for respective ReadSet (RS), WriteSet (WS) and Dependenc Set (DEP), and $\rightarrow C / A$ denotes the local replica validation outcome (Commit/Abort).} 
\label{fig:MVTSOEX}
\end{figure*}

Our starting point is a Multiversioned Timestamp Ordering scheme (MVTSO) which prescribes a serialization order by assigning a speculative timestamp \textit{before} Execution \cite{bernstein1983multiversion, reed1983implementing, su2017tebaldi}. This allows us to a) only classify concurent transactions as conflicting if their execution results violate the timestamp order (Fig. \ref{fig:MVTSOEX}: 4 bottom), and b) avoid write/write conflicts alltogether by applying them in order. 
When speculative execution results match the pre-defined timestamp order, no aborts are necessary (Fig. \ref{fig:MVTSOEX}: 4 top). \sys challenge is therefore to a) assign appropriate timestamps and b) coordinate execution in a way that maximizes such coherence; the presence of byzantine participants (clients/replicas) complicates this.


In MVTSO, reads operations return the latest written version smaller than the readers' timestamps, while writers attempt to create new versions at their timestamp, but must abort if a higher timestamped read would have "missed the write" by reading a prior version. A transaction may commit only, if all read-write dependencies have committed respectively.

Read operations in \sys have the following sub-goals: \one Honest clients should read valid data, i.e. experience read integrity, \two Honest clients should read fresh data, i.e. minimize staleness and hence maximize commit chance, and \three Reads must provide the context necessary to potentially complete observed write state. 
Clients validate the integrity of reads by requiring replicas to provide a proof of validity, i.e. a set of signatures confirming the committment, or if not yet existant, trusting  only $f+1$ matching replies from discrete replicas. Moreover, in order to guarantee that clients do not read maliciously stale data from their local replica, \sys encourages clients to always read from at least $f+1$ replicas and choose the freshest version (Fig. \ref{fig:MVTSOEX}: 1,2).
Avoiding byzantine influence comes at a cost: Read operations require a synchronous, potentially WAN, rountrip. 

MVTSO intuitively synergizes well with \sys{}'s potenially WAN remote reads as reading from a fixed timestamp helps speculative readers observe consistent snapshots, even when execution is long and consequently interleavings are frequent (Fig. \ref{fig:MVTSOEX}: 6). \fs{at the price of experiencing serializability instead of strict serializability - that is only the case if your timestamps are outdated}.

While we assume that clocks are loosely synchronized across honest participants, byzantine participants may diverge arbitrarily in undesirable fashion. To side-step the additional overhead that a dedicated, bounded timestamping phase \cite{Clairvoyant} incurs, we compromise by allowing clients to optimistically select their own timestamp, but rejecting request timestamps above a Threshold at all honest replicas, thus incentivising clients to select real-time timestamps. In order to facilitate a total serialization order across all clients, we define Timestamps to be tuples of the form $(Time, CID)$. 

In traditional MVTSO, writes become visible to successive reads with higher timestamps immediately. In the presence of byzantine clients this however, is undesirable, as it allows clients to issue singular write operations without the intention of ever committing a transaction. Thus, any honest clients' read that observes, and consequently depends on, a byzantine clients write may be blocked indefinitely, thus compromising its liveness.
Allowing other clients to preemtively commit outstanding write operations is infeasible, as the remaining transaction procedure is known only to the issuing client, while conceiding pessimistic abort permission empowers byzantine participants to obstruct any honest clients' writes. 
To reconcile this dilemma, \sys{} defers all database updates until execution has finalized. Concretely, \sys clients buffer all write operations, and submit them only when attempting to commit the transaction, thereby enabling other \sys client to orderly complete Validation and Writeback steps. 

Clients in \sys distinguish explicitly between \textit{committed} and tentatively, but unconfirmed \textit{prepared} writes: While clients trust and accept every verified \textit{committed} write versions, they accept tentative \textit{prepared} versions only when observing $f+1$ matching results. This ensures that a) at least one honest replica believes that the write can commit, and hence is worth observing, and b) that byzantine replicas cannot violate byzantine independence by reactively inventing transactions that are guaranteed to abort.

Finally, \sys replicas abort all writes that higher timestamped reads "miss" by storing a Read Timestamps (RTS) for each locally processed read (Fig. \ref{fig:MVTSOEX}: 3). While this allows in-execution reads to elicit external effects, much like outstanding write operations discussed above, these are only limited to concurrent transactions with smaller timestamps and hence are straightforward to bypass by re-trying a transaction. To nonetheless limit repeatable abuse , we discuss a practical mechanism to limit byzantine influence in section Y.z (Optional Modifc. personalized read leases).



\subsubsection{Execution interface}
Client applications execute transactions via the following interface. A TX object \textit{TXObj $\coloneqq$ (SeqNo, ClientID, InvolvedShards, ReadSet, WriteSet, Dependencies)}, records the state necessary for Validation.

\iffalse
\begin{figure}
\begin{center}
\includegraphics[width= 0.5\textwidth]{./figures/TxState.png}
\end{center}
\caption{Transaction execution state}
\label{fig:Txstate}
\end{figure}
\fi

\textbf{Begin()} A client begins a transaction by optimistically choosing a timestamp \textit{TS $\coloneqq$ (Time, ClientID)}. \\
\textbf{Write(key, value)}. A client buffers the write: \textit{WriteSet = WriteSet $\cup$ (key, value)}\\
%%%%%%%%%Read protocol%%%%%%%%%%%
\textbf{Read(key, TS, RQS)} 
If \textit{key $\in$ WriteSet} a client returns the buffered write value. Otherwise, the client conducts a remote read:

\fbox{\begin{minipage}{22em}

\textbf{1: C} $\rightarrow$ \textbf{R}: Client sends read request to Replicas
\end{minipage}}\\
Given a hyperparameter Read Quorum Size (RQS), a client sends $m = (Read, key, TS)_c$  to $RQS$ different replicas. Note, that in order to guarantee $\geq RQS$ replies a client might need to send up to $f$ additional requests to compensate for unresponsive/faulty participants ($max(|Replies|) \leq n-f$). 

\fbox{\begin{minipage}{22em}
\textbf{2: R} $\rightarrow$ \textbf{C}: Replica processes client read and replies
\end{minipage}}\\
A replica authenticates the client\footnote{Byzantine replica may ignore read access control. Solving this problem is beyond the scope of this work; we defer to existing solutions \cite{basu2019efficient}.}, and whether the timestamp is within a local highwater mark ($HW = localClock + \delta$). It returns a signed message \text{$\langle \textit{ReadReply, Committed, Prepared} \rangle _R$}, where $Committed \coloneqq (value, version, proof)$ represents the value-version pair with largest committed write version smaller than TS and a proof of commitment,  and $Prepared \coloneqq (value, version, TxID', deps)$ the respective largest uncommitted value-version pair, the associated transaction ID, and the latter transactions potentially uncommitted read-write dependencies. Moreover, a Replica stores a new read timestamp (RTS) for the key: $RTS(key) = RTS(key) \cup TS$ (Fig. \ref{fig:MVTSOEX}: 3). 
\fs{a replica may include a set of prepared values to increase likelihood of client receiving f+1 matching}

\fbox{\begin{minipage}{22em}
\textbf{3: C} ($\rightarrow$ \textbf{R}): Client receives read replies 
\end{minipage}}\\
A client waits for RQS read replies and chooses the biggest valid result \textit{(value, version) $= max_{valid}$(\{Committed\},\{Prepared\}} (Figure \ref{fig:MVTSOEX}: 1,2) . A \textit{Committed} tuple is valid, if the proof confirms commitment, wheras a \textit{Prepared} is valid iff there exist $f+1$ matching \textit{Prepared$_r$} signed by different replicas. The client adds the version to its read set \textit{ReadSet = ReadSet $\cup$ (key, version)} and additionally claims a dependency if it was a \textit{Prepared} version: \textit{Dep = Dep $\cup$ f+1 $\times$ (Prepared$_r$)} . 

\textbf{Commit()} A Client terminates its execution, and computes a unique transaction identifier based on final execution object and its timestamp: $TxID \coloneqq (H(TxObj, TS)$, thus preculding byzantine participants from equivocating transaction contents. It then begins the Validation Phase by issuing a 2PC-Prepare requests to each relevant shard and waiting for all votes.

\textbf{Abort()} A client terminates execution, and broadcasts a request to release all acquired Read Timestamps (RTS). Since writes in \sys are deferred, no other rollback action is necessary.


We briefly discuss some implications of the choice of Read Quorum Size (RQS). Following cases may be distinguished: \one \textbf{$RQS = 1$} A replica may read from just 1 replica at the risk of reading maliciously stale data. If a Client trusts a local replica \textbf{and} the replica is not lagging behind, this can reduce execution latency and consequently minimize conflicting interleavings. \two \textbf{$RQS \geq f+1$} \sys{}'s recommended mode of operation. While side-stepping maliciously stale reads, it is still possible to read (arbitrarily) stale data, either due to inconsistency caused by asynchrony or a byzantine client not fully replicating its transaction. 
\three \textbf{$RQS \geq \frac{n+f+1}{2}$} When reading from a Quorum of sufficient size to overlap with any Validation Quorum (ref section Validation) in at least one honest replica, a client guarantees, that no more \textbf{additional}, beyond previously observed, conflicting writes can be admitted, since such a Quorum of acquired RTS acts as a read-lock. 

\iffalse
Trusting only $f+1$ matching prepared writes has the beneficiary side effect of allowing us to include only transaction identifiers as dependencies, rather than the full transaction, since it is guaranteed that at least one honest replica has stored the transaction. Thus, in the failure free scenario, where clients need not complete transactions $in Dep$, \sys minimizes the meta-data overhead that the recovery protocol imposes.
\fi

We remark, that a byzantine client is not required to follow any of the protocol steps, with the exception of claiming dependencies. By our definition of Byzantine-Isolation, read integrity is only required for honest clients; Byzantine clients may \textit{choose} whether to read legal, or even real data. We do, however, require that no client can fabricate dependencies in order to preclude byzantine clients from indetectibly stalling their own transactions and consequently obstructing liveness for consecutive descendant (second degree...) dependencies.
Trusting only $f+1$ matching prepared writes allows us to only include only transaction identifiers as dependencies, rather than the full transaction, thereby minimizing the common path overhead. When the full transaction information is required to complete potentially stalled transactions, it is guaranteed to be obtainable from an honest replica. We discuss how to complete stalled or slow transactions in section Y (Granting Liveness).





%
\subsection{Validation Check}

Algorithm \ref{mvtso} shows the necessary validation check to preserve Byzantine-Serializability. 
Given a transactions prepare request the validation check returns an Abort vote if a conflict has been detected, and Commit otherwise. 
When execution results match the timestamp order, there are no conflicts (Fig. \ref{fig:MVTSOEX}: 4 top). For each read, a replica verifies that it has not voted to commit a conflicting write (Algorithm \ref{mvtso}, line 3-7). Conversely, for each write, a replica confirms that there exist no previously accepted reads (line 8-10), and no ongoing read transactions (line 11-12) that conflict. Fig. \ref{fig:MVTSOEX}: 4 shows both non-conflicting and conflicting interleavings.
If there are no conflicts, a replica tentatively \textit{prepares} a transaction, making its writes visible and evaluating future transactions against it for conflicts. Regardless of the outcome, a replica garbage collects all Read Timestmaps (RTS) associated with the transactions reads.
It then waits for necessary dependencies (uncommited writes that were read) to be resolved (Figure \ref{fig:MVTSOEX}: 5). We remark, that the concurrency control check is serialized and executed atomically for each transaction.

\begin{theorem}
The set of transactions for which the MVTSO-Check returns Commit is Byzantine-Serializable. 
\end{theorem}
\begin{proof}
See TR.
\end{proof}

\begin{algorithm}
\caption{MVTSO-Check(TX, TS)}\label{mvtso}
\begin{algorithmic}[1]
\If{\textit{$TS > localClock + \delta$}} %  || $TS < lowWM$ || $\exists d \in dep: d.TS < lowWM$} } dont mention garbage collection part here, it only confuses
\State \Return Abort
\EndIf

\For{\textit{$\forall key,version \in \textit{TX.RS}$}}
        \If{$ \exists TX2 \in Committed \cup Prepared: key \in \textit{TX2.WS} $ \newline
        \hspace*{2em} $\land \, version < \textit{TX2.TS} < TS$}  
          \State  \Return Abort, \textit{TX2, (TX2.CommitProof)}  
         \EndIf  
\EndFor

\For{\textit{$\forall key \in \textit{TX.WS}$}}
        \If{$\exists TX2 \in Committed \cup Prepared:$ \newline 
        \hspace*{2em} $\textit{TX2.RS[key].version} < TS < TX2.TS$} 
          \State  \Return Abort, \textit{TX2, (TX2.CommitProof)}
         
        \EndIf
        \If{$\exists RTS \in key.RTS: RTS > TS$} 
          \State  \Return Abort
       \EndIf
\EndFor
\State Prepared.add(TX) 

\While{$\exists d \in dep: d \notin CommitLog \cup AbortLog $)}
\State Suspend
\EndWhile

%structure it in a way that is better
\For{\textit{$\forall d \in dep$}}
		\If{$ d \in AbortLog $}
		\State	Prepare.remove(TX)
		\State \Return Abort, \textit{(TX2.AbortProof)}
		\EndIf
\EndFor
\State \Return Commit
\end{algorithmic}

\end{algorithm}

In order to perform the MVTSO-check, a replica maintains several data strucutres: \one It stores read timestamps, read versions alongside the multiversioned write stores for committed and tenative transactions respectively, to provide efficient evaluation of conflicts.
\two In order to confirm dependency outcomes, replicas log proofs for completed transactions in respective Commit and Abort Log sets, that together induce the ledger of all processed transactions. 
\three To avoid busy waiting when dependencies are not yet resolved, a replica temporarily suspends the MVTSO-check for the current transaction, allowing it to process other transactions pending validation. To facilitate this, it keeps track of an additonal transaction to dependents mapping that allows to identify and resume all suspended MVTSO-checks associated with dependents of a completing transaction.

\iffalse

\textit{Aside:} Consistent with our definition of Byzantine-Isolation, byzantine clients may issue ficticious Read-Sets comprised of arbitrary read versions and values. However, these have only limited external effect on concurrent writes. We distinguish two extreme cases: 1) $read.version \rightarrow 0$: This case is equivalent to simply reading stale data, and effectively reduces MVTSO to TSO as conlflicts are evaluated only on basis of the transaction timestamps (i.e. Abort write if: write.TS < read.TS). 2) $read.version \rightarrow read.TS$: In this case there are no conflicts as a write is never "missed" by a previous read.
\fi

In the following section we will show how to design a replicated validation scheme that upholds Isolation guarantees and reaches a single shard decision, even when replicas within a shard validate in different orders.


%%-------------------------------------------------------------------------------
\subsection{Validation Phase}
%-------------------------------------------------------------------------------

In order to facilitate atomic committment across all shards involved in a transaction \sys clients invoke a Prepare request, inquiring each shard to validate the transaction for conflicts and cast a 2PC vote. Shards in \sys are replicated for fault tolerance.

\sys comes in two different flavors, \sys{}3 and \sys{}5 respectively, that rely on varying replication degrees, but implement the same design. \sys{}3 requires $n=3f+1$ replicas \fs{, the minimum bound necessary for BFT SMR?,} per shard to guarantee consistency in the presence of $\leq f$ byzantine replicas. \sys{}5 reduces both latencies during failure free execution and complexity during recovery. \footnote{We believe that consortiums with high performance requirements or high replication degrees are respectively comfortable with paying for additional replicas or tolerating a lower fraction (1/3 vs 1/5) of failures}
For the simplicity of exposition we discuss \sys{}5 for the remainder of the paper, and defer to section X \fs{and/or TR} to describe differences in \sys{}3. In the following, we outline \sys 's execution (unaffected by replication degree), validation and writeback protocols.

The goal of the validation phase is straightforward:
Shards need to cast a single durable vote, in the following we refer to this as the shard-decision. It is computed on the basis of different replica votes.
The vote needs to respect isolation guarantees
leaderless and partially ordered (i.e. coordination only when necessary)



The goals of the Validation phase are threefold: \one It must decide on a single, durable vote per-shard that maintains Byzantine Serializability (henceforth we refer to this as a \textit{shard-decision}), \two It should be leaderless, and not enforce unecessary ordering for commutative, non-conflicting transactions, and \three It must preserve independent operability. \fs{needs to refer to the fallback}

Satisfying these goals requires ovecoming several challenges. In order to maximize parallelism and embrace partial ordering, Indicus allows replicas to process requests out of order. Consequently, replicas may temporarily diverge, and hence return different validation results. Such divergence must be reconciled in a way that maintains Isolation, but not overly conservatively in order to maximize the ability to commit successfully and bound the impact of byzantine participants. Further, Indicus designates clients as validation coordinator for its own transactions, thus omitting a dedicated replica leader. The respective protocol must tolerate client failures such as crashes, omission/stalling, equivocation, or replays and allow for consistent recovery. 

The Validation protocol can be broken down into two functionalities: Voting and Logging.
Since \sys replicas may process requests in different order, they must reach consensus on a joint  decision. To do so, they cast a vote for their local validation result, which are subsequently democratically aggregated into a decision. In the case of Indicus, the client acts as the transaction coordinator who aggregates and relays results (along with necessary evidence).
In order to maintain consistency in the presence of failures (no two honest replicas finalize different Commit/Abort decisions), this decision must be durable and unique, guaranteeing that a replay of any Writeback is idempotent. However, since a byzantine coordinator cannot be trusted to durably store a decision, nor could we retain liveness during a crash or partition, \sys demands clients to \textit{log} the decision at replicas before returning.
 

The voting step requires a single round-trip to all replicas, whereas the logging phase requires at most one round-trip \fs{in Indicus5 and at most two round-trips in Indicus3.}. When execution is fault- and contention free transactions can be committed on the \textit{Fast-Path} in a single-round trip as an explicit logging round is not necessary.


%%%%%%%%%% protocol
To describe how the protocol operates in detail we follow a single-shard transaction through the system:

\fbox{\begin{minipage}{23em}
\textbf{(1: C $\rightarrow$ R)}: Client sends Prepare request to all Replicas within the Shard.
\end{minipage}}\\
Upon deciding to Commit in the Execution phase, a Client initiates Validation by sending a message $Phase1 \coloneqq \langle Prepare, TxID, TX \rangle_{\sigma_c}$ to all Replicas.

\fbox{\begin{minipage}{23em}
\textbf{(2: R $\rightarrow$ C)}: Replica receives validation request, processes it and returns vote to Client.
\end{minipage}}\\
A replica validates Timestamp and Dependency integrity of the request. It then evaluates Read and Write Sets for Isolation conflicts against its local state using the MVTSO Concurrency Control Check (CCC), as shown in algorithm 1. It returns a message $Phase1R \coloneqq \langle TxID, vote \rangle_r$, and optionally evidence in case it voted to Abort.

\underline{Additional subtlelties}: 
A replica never changes its Voting decision, because re-execution could leave to different results. Once the MVTSO-Ceck completes (i.e. there are no blocking dependencies), a replica starts a timer to monitor the clients progress.

\fbox{\begin{minipage}{23em}
\textbf{(3: C)}: Client waits for vote replies.
\end{minipage}}\\
A client waits for at least $n-f$ ($4f+1$ in Indicus5) distinct replica votes, or more, up to a system specified timeout. 

\fbox{\begin{minipage}{23em}
\textbf{(3a: C)}: Client receives Threshold of matching votes and returns to application. Proceeds to Writeback
\end{minipage}}\\
In any of the following 3 cases, a client may short-circuit waiting for additional votes and omit a dedicated Logging round:
\begin{enumerate}
\item \textbf{$1$ Abort vote w/ Conflicting TX \& CommitCertificate}: A conflict with a commited transaction. The client validates the integrity of the CommitCertificate and returns the shard decision $(TxID, Abort, \langle ConflTX \rangle_{CC})$. 
\item \textbf{$3f+1$ Abort votes w/ Conflicting TX}: A conflict with a prepared, but not yet committed transaction. The client returns the shard decision $(TxID, Abort, \{\langle AbstainVote\rangle_r\})$. 
\item \textbf{$5f+1$ Commit votes}: No conflicts. The client returns the shard decision $(TxID, Commit, \{\langle CommitVote \rangle_r\}$
\end{enumerate}
Any of such Quorums forms a \textit{Shard-Certificate} and proves the decision. A client uses this Certificate to return to its application and issue the Writeback.

\fbox{\begin{minipage}{23em}
\textbf{(3b: C $\rightarrow$ R)}: Client receives divergent results and suggests a consistent decision to Replicas for Logging
\end{minipage}}\\
If a client does not receive the necessary thresholds of votes to return, it must continue on the \textit{Slow-Path}. To do so, it aggregates the votes according to following decison rule:
If there exists a $CommitQuorum \coloneqq \frac{n+f+1}{2}$ of Commit Votes, the Slow-Path decision is Commit, otherwise it is Abort.
A client broadcasts a message $Phase2 \coloneqq (TxID, decision)_c, \{\langle votes \rangle_r\}$.

\underline{Additional subtlelties}: A client forwards a Quorum of $\geq n-f$ votes to the replicas in order to prove the Slow-Path decision is consistent with Isolation guarantees. Note, that a byzantine client may equivocate the decision by relaying different Quorums.

\fbox{\begin{minipage}{23em}
\textbf{(4: R $\rightarrow$ C)}: Replicas receive, validate and echo decision
\end{minipage}}\\
A replica confirms that the Decision matches the Quorum by evaluating the decision rule itself and adopting the decision. It then returns the decision to the client by sending $Phase2R \coloneqq \langle TxID, decision \rangle_r$. Importantly, a replica never changes its decision.

\fbox{\begin{minipage}{23em}
\textbf{(5: C)}: Client returns shard-decision to application and proceeds to Writeback
\end{minipage}}\\
A client waits for a Quorum of $n-f$ matching $Phase2R$ messages. Such a Quorum forms a \textit{Shard-Certificate} and proves the decision. A client uses this Certificate to return to its application and issue the Writeback.

\underline{Additional subtlelties}: If a client equivocated, it will never receive a Shard-Certificate. An honest client however, is guaranteed to receive matching Phase2 replies. 

We consider a decision (Commit, Abort) to be \textit{logged} when it is possible for some Shard-Certificate to exist, i.e. as soon as the necessary certificate Quorums exists at some Replicas.
Figure \ref{fig:FigureSP} summarizes the relevant nomenclature.

\begin{figure}
\begin{center}
\includegraphics[width= 0.5\textwidth]{./figures/Nom2.png}
\end{center}
\caption{Validation Nomenclature, Slow-Path. Note, that a byzantine client may equivocate Phase2 decisions by including Commit and Abort Quorums respectively. Byantine replicas may store multiple votes and decisions.}
\label{fig:FigureSP}
\end{figure}

\subsubsection{Correctness}
We show, that a \textit{logged} decision is final:
\begin{theorem}[Saf]
A logged decision is durable, and there can ever exist \textbf{at most one} logged decision.
\end{theorem}
\begin{proof}
See TR.
\end{proof} 

Note, that since replicas never change their decision, it is possible for there to never be any logged decision if a byzantine client equivocated its Slow-Path Quorums. In order to reconcile this, we design and discuss a recovery mechanism in section X which relaxes the requirement on persisting a decision.  


\begin{theorem} 
Indicus maintains \textit{Byzantine-Serializability}.
\end{theorem}
To prove that this is the case, we show that for any two conflicting transactions, at most one can be committed.
\begin{proof}
See TR.
\end{proof}

\begin{theorem} 
Indicus maintains Byzantine Independence in the absence of network adversary.
\end{theorem}

We show, that once a Client submits a transaction for validation, the result cannot be unilaterally decided by any byzantine participant, be it client or replica.
\begin{proof}
See TR.
\end{proof}

%%-------------------------------------------------------------------------------
\subsection{Writeback}
%-------------------------------------------------------------------------------

Validation occurs on every Shard that a transaction spans. The goal of the Writeback phase is to aggregate all relevant shard-decisions and to inform replicas of finalized Commit or Abort decisions. This is necessary in order for replicas to be able to garbage collect meta-data of ongoing transactions, and to allow consecutive transactions to reliably observe the updated state. Any finalized decision must respect all shard-decisions relevant to a transaction. Thus, all decisions are aggregated according to standard two-phase commit. \sys follows a simple 2PC protocol: Only if all shards agree that a transaction may commit (i.e. there exist commit certificates for every shard), then a transaction may commit. A single abort shart-decision suffices to abort a transaction respectively.

\textbf{1.} A coordinator waits for all necessary shard decisions, including certificates\\
\textbf{2.} The coordinator broadcasts a $Writeback \coloneqq (TX, decision, \{certificates_S \} )$ message to all replicas in all relevant shards.\\
\textbf{3.} A replica validates whether the certificates match the decision, and commits/aborts the transaction by applying the relevant store updates, resuming potentially suspended MVTSO-Checks, and garbage collecting all ongoing transaction state. Replicas store the Writeback message which is used as Commit proof to service reads and justify conflicts, or Abort proof to justify dependency aborts.

We point out, that the Writeback coordinator need not be the client issuing the transaction, but can in fact be an arbitrary party (client or replica) that is interested in completing the Writeback. This follows straightforwardly from Theorem Y: Any certified shard-decision implies the existence of a logged decision, and hence the Writeback phase is idempotent.
We utilize this to drive the recovery protocol outlined next. 


%\subsection{Multi-sharding}

and Multi-shard 2pc

- Optimization: Single shard logging


%%-------------------------------------------------------------------------------
\subsection{Failures}
%-------------------------------------------------------------------------------
- Fallback: election (only starts if not waiting on another dep to avoid early eviction), views, resolution, subtelties with mvtso (block because of dep), necessity even without dependencies. Interested clients, write-back multishard. garbage collection
- Fallback requires an extra round in order to learn about current views to start viewchange, but thats ok: Its co-function with learning about full TX, and checking for existing certificates. Timeout invocation is concurrent with p1 message.



%%-------------------------------------------------------------------------------
\subsection{Optimizations}
%-------------------------------------------------------------------------------


\fs{the current writeback might be unecessary to explain, since we implement the below anyways... Perhaps merge writeback and this subsection}
\paragraph{Single Shard Logging}
\fs{This does not work for Atomic Broadcast! In AB, the voting only happens AFTER the tx has already been logged (i.e. the order has been durably replicated)}

When transaction execution touches multiple shards validation can incur redundant explicit logging overhead. When a Slow-Path is necessary to arrive at a logged decision on S different shards, bandwith is wasted. Consider an example in which $S-1$ shards attempt to log the decision Commit, while a single shard attempts to log an Abort decision. If the latter shard succeeds, the effort of the remaining shards was in vain. \fs{Moreover, logging is always bottlenecked by the slowest shard. }
The culprit of this phenomenon is the delayal of Two-Phase-Commit (2PC) until the Writeback phase. By preemptively making a 2PC decision \textbf{before} logging we can avoid this redundancy. We remark, that even when when all shards agree on a decision, this saves redundant coordination.
Concretely, we designate \textbf{one} involved shard as \textit{logging Shard}, while all other shards remain responsible only for Validation. The logging Shard can be determined via a determinsitic function over the \textit{involved Shards}. A simple load balanced solution may select $loggingS = min(TxID \% involvedShard)$. \fs{This does not work for Atomic Broadcast! In AB, the voting only happens AFTER the tx has already been logged (i.e. the order has been durably replicated).Figure \ref{fig:SingleShardOpt} shows a comparison and the revised structure. dont use this figure in actual paper} In order to log a decision, voting quorums from all involved shards are required. We modify step 3 of the Validation protocol accordingly:

\fbox{\begin{minipage}{21em}
\textbf{Validation (3: C)}: Client waits for vote replies from all involved Shards.
\end{minipage}}
A client aggregates a per-shard decision for each shard according to the \textit{CommitQuorum} rule. If all shard-decisions are Commit, it attempts to log a Commit decision by sending $Phase2 \coloneqq (TxID, Commit, S \times \{CommitQuorum\}$ to all replicas in the designated logging Shard. If a single shard-decision is Abort, it stops waiting for other shard-decisions and attempts to log an Abort decision by instead sending $Phase2 \coloneqq (TxID, Abort, AbortQuorum)$. 

\underline{Additional subtlelties:} A client can go Fast-Path and return to the Writeback phase immediately only if Fast-Path Quorums were received for all shards. 

The remaining Validation protocol proceeds identically to the multi-shard version. Notice, that when only a single shard is involved, no adjustments were made. The Writeback phase instead, may proceed with just the single certificate from the logging Shard


The Fallback protocol is adjusted accordingly: A fallback replica need (and can) only be elected on the logging Shard, simplifying reconciliation and reducing the cost for interested clients. 
To further reduce unecessary load, a client may attempt to first inquire whether decisions exists at the logging shard (Fallback protocol step 1 \& 2), before sending $Rec-Phase1$ messages to all shards in order to gather votes itself (Fallback protocol step 1). 

\iffalse
\begin{figure*}
\begin{center}
\includegraphics[width= \textwidth]{./figures/SingleShard.png}
\end{center}
\caption{Single Shard Optimization}
\label{fig:SingleShardOpt}
\end{figure*}
\fi

 
